\documentclass[a4paper,10pt,ngerman]{scrartcl}
\usepackage{babel}
\usepackage[T1]{fontenc}
\usepackage[utf8x]{inputenc}
\usepackage[a4paper,margin=2.5cm,footskip=0.5cm]{geometry}
\newcommand{\Aufgabe}{Allgemeines}
\newcommand{\TeilnahmeId}{67275}
\newcommand{\Name}{Lennart Protte}
\usepackage{scrlayer-scrpage, lastpage}
\setkomafont{pageheadfoot}{\large\textrm}
\lohead{\Aufgabe}
\rohead{Teilnahme-ID: \TeilnahmeId}
\usepackage{listings}
\lstset{escapeinside={<@}{@>}}
\usepackage{color}
\usepackage{titling}
\setlength{\droptitle}{-1.0cm}
\usepackage{hyperref}
\usepackage{cleveref}
\title{\textbf{\Huge\Aufgabe}}
\author{\LARGE Teilnahme-ID: \LARGE \TeilnahmeId \\\\
\LARGE Bearbeiter dieser Aufgabe: \\
\LARGE \Name\\\\}
\date{\LARGE\today}
\begin{document}
    \maketitle
    \tableofcontents
    \vspace{0.5cm}
    \textbf{Vorwort:}
    Verglichen mit der ersten Runde, erscheint mir die Bearbeitung der zweiten Runde deutlich anspruchsvoller.
    Dies mag zum Teil daran liegen, dass die Bearbeitung während der Abiturvorbereitung gutes Zeitmanagement erfordert,
    als auch zum Teil daran, dass ich mir die Sprache C++ neu für die Implementation meiner Algorithmen angeeignet habe.
    \newline
    Die Entscheidung, welche der Aufgaben ich bearbeite, ist mir leicht gefallen, da mir die Aufgabenstellung der ersten beiden
    Aufgaben sofort interessant erschienen ist.
    Allerdings habe ich während der Bearbeitung nicht nur eine Programmiersprache erlernt,
    sondern auch mein Wissen in außerschulischen Inhalten der theoretischen Informatik und der Algorithmik erweitern können.
    \newline
    Mein erster Ansatz für die Lösung der Aufgabe Weniger krumme Touren war ein SAT-Solver.
    Das Ergebnis war zwar lauffähig, aber hatte eine schlechte Laufzeitkomplexität.
    Der Algorithmus hat jede mögliche Permutation auf eine Reihe von Klauseln getestet, welche zuvor aufgestellt wurden.
    Dieser Ansatz ist nicht in der finalen Abgabe enthalten aber dennoch eine Erwähnung wert,
    da ich in diesen Ansatz viel Zeit investiert habe.
    \section{Verwendete Sprachen}\label{sec:verwendete-sprachen}
    Als Sprache habe ich C++ verwendet.
    Ich hatte in dieser keinerlei Vorwissen und habe daher den BWINF-41 als Gelegenheit genutzt C++ zu erlernen.
    Die, für die Bearbeitung der BWINF-Aufgaben relevanten, Vorteile von C++ sind zum einen die
    Plattformunabhängigkeit (da ich auf mehreren verschiedenen Betriebssystemen arbeite),
    als auch Leistungsstärke und Effizienz in der Laufzeit und im Speicher (da eine geringe Laufzeit die Entwicklung vereinfacht),
    so wie Objektorientierung (da ich bereits Java Kenntnisse hatte, war der Umstieg auf C++ realistisch).
    \section{Verwendete Programme}\label{sec:verwendete-programme}
    Als Programme habe ich CLion und Texmaker verwendet.
    Texmaker habe ich für das Schreiben der Dokumentation mit LaTeX verwendet,
    da es mir bereits aus der ersten Runde vertraut war.
    CLion ist eine Jetbrains IDE und da ich bereits gut vertraut mit anderen IDE's, wie Intellij, Webstorm, Golang, etc.,
    war, hat sich der Einstieg in CLion als problemlos herausgestellt.
    \section{Testen}\label{sec:testen}
    Das Projekt enthält insgesamt drei Quellcode-Dateien.
    Um die Quellcode-Dateien zu kompilieren, wird ein C/C++ Compiler benötigt
    (dies ist nicht zwingend notwendig, da bereits ausführbare Programme vorliegen).
    Die ausführbaren Programme wurden unter Ubuntu-22.04.2-LTS kompiliert und getestet.
    Zum Testen der Eingabedatei wenigerkrumm5.txt der Aufgabe ``Weniger krumme Touren''
    muss die Datei von dem Order ``keine\_akzeptable\_zeit'' nach ``Eingabedateien'' verschoben werden.
    Es ist dabei zu beachten, dass der Algorithmus für die Datei keine Lösung in akzeptabler Zeit findet.
    Die Ausgabedateien werden automatisch generiert, sollten diese nicht vorhanden sein.
    Der Compiler kann auf Ubuntu wie folgt installiert und verifiziert werden:
    \begin{lstlisting}[language=bash,label={lst:install}]
        $ sudo apt-get update && sudo apt-get upgrade
        $ sudo apt-get install build-essential manpages-dev
        $ gcc --version
    \end{lstlisting}
    Anschließend können die Quellcode-Dateien wie folgt kompiliert werden:
    \begin{lstlisting}[language=cmd,label={lst:compile}]
        $ g++ -std=c++20 program-header-code.h program-source-code.cpp -o executable-file-name
    \end{lstlisting}
    Die ausführbaren Programme können wie folgt ausgeführt werden:
    \begin{lstlisting}[language=bash,label={lst:run}]
        $ ./executable-file-name
    \end{lstlisting}
    Sollte der Fehler auftreten, dass Eingabe-/Ausgabe-Dateien nicht gefunden werden,
    wie Beispielsweise:
    \begin{lstlisting}[language=bash,label={lst:error},color=blue]
       <@\textcolor{red}{filesystem error: directory iterator cannot open directory: No such file or directory}@>
    \end{lstlisting}
    Befindet sich die kompilierte Datei mit hoher Wahrscheinlichkeit in einem falschen Ordner.
\end{document}