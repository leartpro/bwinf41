%! Author = leartpro
%! Date = 04.01.23

\documentclass[a4paper,10pt,ngerman]{scrartcl}
\usepackage{babel}
\usepackage[T1]{fontenc}
\usepackage[utf8x]{inputenc}
\usepackage[a4paper,margin=2.5cm,footskip=0.5cm]{geometry}

% Die nächsten drei Felder bitte anpassen:
\newcommand{\Aufgabe}{Allgemeines} % Aufgabennummer und Aufgabennamen angeben
\newcommand{\TeilnahmeId}{67275}                 % Teilnahme-ID angeben
\newcommand{\Name}{Lennart Protte}               % Name des Bearbeiter / der Bearbeiterin dieser Aufgabe angeben

% Kopf- und Fußzeilen
\usepackage{scrlayer-scrpage, lastpage}
\setkomafont{pageheadfoot}{\large\textrm}
\lohead{\Aufgabe}
\rohead{Teilnahme-ID: \TeilnahmeId}

% Position des Titels
\usepackage{titling}
\setlength{\droptitle}{-1.0cm}

% Diese beiden Pakete müssen zuletzt geladen werden
\usepackage{hyperref} % Anklickbare Links im Dokument
\usepackage{cleveref}

% Daten für die Titelseite
\title{\textbf{\Huge\Aufgabe}}
\author{\LARGE Teilnahme-ID: \LARGE \TeilnahmeId \\\\
\LARGE Bearbeiter/-in dieser Aufgabe: \\
\LARGE \Name\\\\}
\date{\LARGE\today}

\begin{document}

    \maketitle
    \tableofcontents

    \vspace{0.5cm}

    \textbf{Vorwort:}
    Verglichen mit der ersten Runde, erscheint mir die Bearbeitung der zweiten Runde deutlich anspruchsvoller.
    Dies mag zum Teil daran liegen, dass die Bearbeitung während der Abiturvorbereitung gutes Zeitmanagement erfordert,
    als auch zum Teil daran, dass ich mir die Sprache C++, in welcher meine Programme geschrieben sind, neu angeeignet habe.
    Die Entscheidung, welche der Aufgaben ich bearbeite, ist mir leicht gefallen, da mir die Aufgabenstellung der ersten beiden
    Aufgaben sofort interessant erschienen ist.
    Allerdings habe ich während der Bearbeitung nicht nur eine Programmiersprache erlernt,
    sondern auch mein Wissen in außerschulischen Inhalten der theoretischen Informatik und der Algorithmik erweitern können.


    \section{Verwendete Sprachen}
    Als Sprache habe ich C++ verwendet.
    Ich hatte in dieser keinerlei Vorwissen und habe daher den BWINF-41 als Gelegenheit genutzt C++ zu erlernen.
    Die, für die Bearbeitung der BWINF-Aufgaben relevanten, Vorteile die C++ bietet sind
    Plattformunabhängigkeit (da ich auf mehreren verschiedenen Betriebssystemen arbeite),
    Leistungsstärke und Effizienz in der Laufzeit und im Speicher,
    Objektorientiert (da ich bereits Java Kenntnisse hatte, war der Umstieg auf C++ realistisch),



    \section{Verwendete Programme}
    -Begründung, warum diese gewählt wurden

    \section{Testen}
    -Wie getestet werden kann
    -Wie ausgeführt werden kann

\end{document}
