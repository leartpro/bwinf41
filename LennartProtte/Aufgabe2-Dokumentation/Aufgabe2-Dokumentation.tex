%! Author = leartpro
%! Date = 04.01.23

\documentclass[a4paper,10pt,ngerman]{scrartcl}
\usepackage{babel}
\usepackage[T1]{fontenc}
\usepackage[utf8x]{inputenc}
\usepackage[a4paper,margin=2.5cm,footskip=0.5cm]{geometry}

% Die nächsten drei Felder bitte anpassen:
\newcommand{\Aufgabe}{Aufgabe 2: Alles Käse} % Aufgabennummer und Aufgabennamen angeben
\newcommand{\TeilnahmeId}{67275}                        % Teilnahme-ID angeben
\newcommand{\Name}{Lennart Protte}                      % Name des Bearbeiter / der Bearbeiterin dieser Aufgabe angeben

% Kopf- und Fußzeilen
\usepackage{scrlayer-scrpage, lastpage}
\setkomafont{pageheadfoot}{\large\textrm}
\lohead{\Aufgabe}
\rohead{Teilnahme-ID: \TeilnahmeId}
\cfoot*{\thepage{}/\pageref{LastPage}}

% Position des Titels
\usepackage{titling}
\setlength{\droptitle}{-1.0cm}

% Für mathematische Befehle und Symbole
\usepackage{amsmath}
\usepackage{amssymb}

% Für Bilder
\usepackage{graphicx}

%Für Überschriften
\usepackage[labelformat=empty]{caption}
\captionsetup[algorithm]{labelformat=empty}

% Für Algorithmen
\usepackage{algpseudocode}
\usepackage{adigraph}
\usepackage{algorithm}
\usepackage{algorithmicx}
\usepackage{tikz}

% Für Beispiele
\usepackage{tikz-3dplot}
\usepackage{subcaption}


% Für Quelltext
\usepackage{color}
\usepackage{xcolor}
\usepackage{textcomp}
\usepackage{listings}
\lstset{
    keywordstyle=\color{blue},commentstyle=\color{mygreen},
    stringstyle=\color{mymauve},rulecolor=\color{black},
    basicstyle=\footnotesize\ttfamily,numberstyle=\tiny\color{mygray},
    captionpos=b, % sets the caption-position to bottom
    keepspaces=true, % keeps spaces in text
    numbers=left, numbersep=5pt, showspaces=false,showstringspaces=true,
    showtabs=false, stepnumber=2, tabsize=2, title=\lstname ,
    inputencoding = utf8,  % Input encoding
    extendedchars = true,  % Extended ASCII
    literate      =        % Support additional characters
        {á}{{\'a}}1  {é}{{\'e}}1  {í}{{\'i}}1 {ó}{{\'o}}1  {ú}{{\'u}}1
        {Á}{{\'A}}1  {É}{{\'E}}1  {Í}{{\'I}}1 {Ó}{{\'O}}1  {Ú}{{\'U}}1
        {à}{{\`a}}1  {è}{{\`e}}1  {ì}{{\`i}}1 {ò}{{\`o}}1  {ù}{{\`u}}1
        {À}{{\`A}}1  {È}{{\'E}}1  {Ì}{{\`I}}1 {Ò}{{\`O}}1  {Ù}{{\`U}}1
        {ä}{{\"a}}1  {ë}{{\"e}}1  {ï}{{\"i}}1 {ö}{{\"o}}1  {ü}{{\"u}}1
        {Ä}{{\"A}}1  {Ë}{{\"E}}1  {Ï}{{\"I}}1 {Ö}{{\"O}}1  {Ü}{{\"U}}1
        {â}{{\^a}}1  {ê}{{\^e}}1  {î}{{\^i}}1 {ô}{{\^o}}1  {û}{{\^u}}1
        {Â}{{\^A}}1  {Ê}{{\^E}}1  {Î}{{\^I}}1 {Ô}{{\^O}}1  {Û}{{\^U}}1
        {œ}{{\oe}}1  {Œ}{{\OE}}1  {æ}{{\ae}}1 {Æ}{{\AE}}1  {ß}{{\ss}}1
        {ç}{{\c c}}1 {Ç}{{\c C}}1 {ø}{{\o}}1  {Ø}{{\O}}1   {å}{{\r a}}1
        {Å}{{\r A}}1 {ã}{{\~a}}1  {õ}{{\~o}}1 {Ã}{{\~A}}1  {Õ}{{\~O}}1
        {ñ}{{\~n}}1  {Ñ}{{\~N}}1  {¿}{{?`}}1  {¡}{{!`}}1
        {°}{{\textdegree}}1 {º}{{\textordmasculine}}1 {ª}{{\textordfeminine}}1
}

\usetikzlibrary{3d}          % for 'canvas is...' options
\usetikzlibrary{perspective} % for '3d view' option
\tikzset
{
    linea/.style={draw=red},
    lineb/.style={draw=blue},
}

\newcommand{\simplecube}[8]% origin, dimension x, dimension y, dimension z, style x, style y, style z
{
    \begin{scope}[shift={#1}]
        \fill[gray!40,canvas is yz plane at x=#2, opacity=#8] (0,0) rectangle (#3,#4);
        \fill[gray!10,canvas is xz plane at y=#3, opacity=#8] (0,0) rectangle (#2,#4);
        \fill[white  ,canvas is xy plane at z=#4, opacity=#8] (0,0) rectangle (#2,#3);
        \foreach\i/\j in {0/1, 1/1, 1/0}
            {
            \draw[line#5] (0,#3*\i,#4*\j) --++ (#2,0,0);
            \draw[line#6] (#2*\i,0,#4*\j) --++ (0,#3,0);
            \draw[line#7] (#2*\i,#3*\j,0) --++ (0,0,#4);
        }
    \end{scope}
}

\newcommand{\bigSquare}[4]% origin, a, b, separation
{
    \begin{scope}[shift={#1}]
        \simplecube{(0,     0,      0)}     {1}{4}{2}   {a}{a}{a}   {1}
        \simplecube{(1+#4,   0,      0))}    {1}{4}{2}   {a}{a}{a}   {1}
        \simplecube{(0,     0,      2+#4)}   {2}{4}{1}   {a}{a}{a}   {1}
        \simplecube{(2+2*#4,   0,      0)}     {1}{4}{3}   {a}{a}{a}   {0.8}
        \simplecube{(0,     4+#4,    0)}     {3}{1}{3}   {a}{a}{a}   {0.8}
        \simplecube{(0,     5+2*#4,    0)}     {3}{1}{3}   {a}{a}{a}   {0.5}
        \simplecube{(0,     0,      3+2*#4)}   {3}{6}{1}   {a}{a}{a}   {0.8}
        \simplecube{(3+3*#4,   0,      0))}    {1}{6}{4}   {a}{a}{a}   {0.5}
        \simplecube{(0,     0,      4+3*#4)}   {4}{6}{1}   {a}{a}{a}   {0.5}
        \simplecube{(0,     0,      5+4*#4)}   {4}{6}{1}   {a}{a}{a}   {0.2}
        \simplecube{(4+4*#4,   0,      0))}    {1}{6}{6}   {a}{a}{a}   {0.2}
        \simplecube{(5+5*#4,   0,      0))}    {1}{6}{6}   {a}{a}{a}   {0}
    \end{scope}
}

\newcommand{\smallSquare}[4]% origin, a, b, separation
{
    \begin{scope}[shift={#1}]
        \simplecube{(0,     0,      0)}     {1}{2}{2}   {a}{a}{a}   {1}
        \simplecube{(1+#4,   0,      0))}    {1}{2}{2}   {a}{a}{a}   {1}
    \end{scope}
}

\newcommand{\unsolvedSquare}[4]% origin, a, b, separation
{
    \begin{scope}[shift={#1}]
        \simplecube{(0,     0,      0)}     {1}{2}{2}   {a}{a}{a}   {1}
        \simplecube{(1+#4,   0,      0))}    {1}{2}{2}   {a}{a}{a}   {1}
        \simplecube{(2+2*#4,   0,      0))}    {1}{1}{1}   {a}{a}{a}   {1}
    \end{scope}
}

\newcommand{\threeSquares}[4]% origin, a, b, separation
{
    \begin{scope}[shift={0}]
        \simplecube{(0,     0,      0)}     {1}{1}{1}   {a}{a}{a}   {1}
    \end{scope}
    \begin{scope}[shift={5}]
        \simplecube{(0,     0,      0)}     {1}{3}{3}   {a}{a}{a}   {1}
        \simplecube{(1+#4,   0,      0))}    {1}{3}{3}   {a}{a}{a}   {1}
        \simplecube{(2+2*#4,   0,      0))}    {1}{3}{3}   {a}{a}{a}   {1}
    \end{scope}
    \begin{scope}[shift={7}]
        \simplecube{(0,     0,      0)}     {1}{5}{5}   {a}{a}{a}   {1}
        \simplecube{(1+#4,,     0,      0)}     {1}{5}{5}   {a}{a}{a}   {1}
        \simplecube{(2+2*#4,    0,      0)}     {1}{5}{5}   {a}{a}{a}   {1}
        \simplecube{(3+3*#4,    0,      0)}     {1}{5}{5}   {a}{a}{a}   {1}
        \simplecube{(4+4*#4,    0,      0)}     {1}{5}{5}   {a}{a}{a}   {1}
    \end{scope}
}

% Diese beiden Pakete müssen zuletzt geladen werden
\usepackage{hyperref} % Anklickbare Links im Dokument
\usepackage{cleveref}

% Daten für die Titelseite
\title{\textbf{\Huge\Aufgabe}}
\author{\LARGE Teilnahme-ID: \LARGE \TeilnahmeId \\\\
\LARGE Bearbeiter dieser Aufgabe: \\
\LARGE \Name\\\\}
\date{\LARGE\today}

\begin{document}

    \maketitle
    \tableofcontents
    \vspace{0.5cm}
    \newpage


    \section{Lösungsidee}\label{sec:losungsidee}

    \subsection{Allgemein}\label{subsec:allgemein}

    Der Aufgabenstellung ist entnehmen, dass eine gegebene Menge an Scheiben, wobei zu jede Scheibe durch
    eine Länge und eine Höhe beschreiben wird, geprüft werden soll, ob diese zu einem Quader zusammengesetzt werden können.
    Wenn dies der Fall ist, soll auch die Reihenfolge angegeben werden, in der die Scheiben zu einem, Quader zusammengesetzt werden können.,
    Dabei müssen in der Lösungsreihenfolge alle Scheiben der gegebenen Menge vorkommen.
    Die Aufgabenstellung stellt dabei keine Ansprüche, an die Form des Zielquaders, allerdings erscheint es nicht im Sinne der Aufgabenstellung,
    dass der Zielquader Lücken, beziehungsweise Hohlräume aufweist.

    Beim Zusammensetzten der Scheiben zu einem Quader ist zu beachten, dass diese nicht zwangsläufig alle von der gleichen Seite des Quaders stammen müssen,
    daher müssen alle Dimensionen des Quaders betrachtet werden.
    Aufgrund der mathematischen Eigenschaften eines Quaders, ist es ausreichend, wenn nur drei der sechs Seiten betrachtet werden,
    da parallele Seiten bei einem Quader identisch sind.

    Aus der Menge der gegebenen Scheiben lässt sich das Volumen des Quaders berechnen.
    Aus diesem und der längsten Seite, die eine Scheibe der gegebenen Menge besitzt, lassen sich alle möglichen Varianten des Zielquaders bilden.
    Aufgrund der ganzzahligen Scheibendimensionen, kann angenommen werden, dass auch die Dimensionen des Zielquaders ganzzahlig sind.
    Es müssen daher nur Quader mit ganzzahligen Dimensionen betrachtet werden, welche das berechnete Volumen beinhalten.

    Nun kann für jede mögliche Variante des Zielquaders, mit einem Greedy-Algorithmus, geprüft werden,
    ob sich die gegebene Menge an Scheiben zu dieser Variante zusammenzusetzen lässt.
    Sobald dies der Fall ist, ist eine Lösung gefunden.
    Sollte dies nicht der Fall sein, wird die nächste Variante betrachtet.
    Sollte es für keine der möglichen Varianten eine Lösung geben, ist es nicht möglich die gegebene Menge an Scheiben
    zu einem Quader zusammenzusetzen.

    Um zu prüfen, ob eine Variante in die gegebene Menge an Schieben zerlegt werden kann, wird jede
    Scheibe in drei möglichen Dimensionen des Quaders (Vorne, Oben, Seite) getestet.
    Wenn eine der beiden Seitenlängen der Scheibe mit einer der Seitenlängen des Quaders übereinstimmt,
    kann die Scheibe in der entsprechenden Dimension entfernt werden.
    Wenn dies der Fall ist, wird sie zur Lösungsmenge $O$ hinzugefügt und aus der Liste $S$ entfernt.
    Der Quader wird aktualisiert, indem die entsprechende Seitenlänge reduziert wird.
    Wenn es eine Lösung mit der aktuellen Scheibe gibt, gibt die Funktion ``true'' zurück.
    Andernfalls werden die Änderungen rückgängig gemacht, indem die Scheibe wieder zur Liste $S$ hinzugefügt wird
    und die letzte Scheibe aus der Lösungsmenge $O$ entfernt wird.
    Wenn keine der noch nicht betrachteten Scheiben entfernt werden kann, gibt die Funktion ``false'' zurück.

    \begin{algorithm}[H]
        \caption{Berechnung eines Käsequaders aus gegebenen Käsescheiben}
        \begin{algorithmic}
            \Require Die Länge $l$, Höhe $h$ und Tiefe $d$ des Käsequaders, eine Liste $S$ von Käsescheiben, eine leere Liste $O$
            \Ensure Ein boolescher Wert, der angibt, ob die Käsescheiben zu einem vollständigen Käsequader zusammengesetzt werden können,
            und eine Lösungsreihenfolge $O$, in der die Käsescheiben zusammengesetzt werden können.
            \Function{berechne\_quader}{$\textit{l},\ \textit{h},\ \textit{d},\ \textit{S},\ \textit{O}$}
                \If{$l$ = 0 or $h$ = 0 or $d$ = 0}
                    \State \Return wahr
                \EndIf
                \State $R_{entfernt} \gets \emptyset$
                \For{$s \in S$}
                    \State $D_{dimension} \gets \textsc{kann\_entfernt\_werden}$
                    \If{$D_{dimension} \neq UNG\ddot ULTIG$}
                        \State $l_{neu} \gets l - (D_{dimension} = SEITE ? 1 : 0)$
                        \State $h_{neu}  \gets h - (D_{dimension} = OBEN ? 1 : 0)$
                        \State $d_{neu}  \gets d - (D_{dimension} = VORNE ? 1 : 0)$
                        \State $O \gets {scheibe, dimension}$
                        \State $R_{entfernt} \gets s$
                        \State $S \notin s$
                    \EndIf
                    \If{$\textsc{berechne\_quader}(l_{neu},h_{neu},d_{neu},S,O)$}
                        \State \Return wahr
                    \Else
                        \State $O \notin letztes$
                        \For{$r \in R_{entfernt}$}
                            \State $scheiben \gets r$
                        \EndFor
                        \State $R_{entfernt} \gets \emptyset$
                        \EndElse
                    \EndIf
                \EndFor
                \State \Return falsch
            \EndFunction
        \end{algorithmic}\label{alg:pseudo_greedy}
    \end{algorithm}


    \begin{algorithm}[H]
        \caption{Bestimmung der passenden Seite des Quaders zur Scheibe}
        \label{alg:cheese2}
        \begin{algorithmic}
            \Function{kann\_entfernt\_werden}{$l, h, d, s$}
                \If{($s_{l} = l$ \land $s_{h} = h$) \lor ($s_{l} = h$ \land $s_{h} = l$)}
                    \State \Return $VORNE$
                \ElsIf
                        {($s_{l} = l$ \land $s_{d} = d$) \lor ($s_{l} = d$ \land $s_{d} = l$)}
                    \State \Return $OBEN$
                \ElsIf
                        {($s_{h} = h$ \land $s_{d} = d$) \lor ($s_{h} = d$ \land $s_{d} = h$)}
                    \State \Return $SEITE$
                \Else
                    \State \Return $UNG\ddot ULTIG$
                \EndIf
            \EndFunction
        \end{algorithmic}
    \end{algorithm}

    \subsection{Erweiterung}\label{subsec:erweiterung_losungsidee}

    Eine mögliche allgemeinere Fragestellung der Aufgabe könnte sein, dass Antje vergessen hat,
    wie vielen Käsequadern sie am Anfang hatte.
    Daher, zu welcher minimalen Anzahl von Quadern lässt sich die gegebene Menge an Scheiben zusammensetzten.

    Um diese erweiterte Problemstellung zu lösen kann der ursprüngliche Algorithmus weiterhin angewendet werden.
    Allerdings muss dieser nun für alle möglichen Varianten von $n$ Quadern ausgeführt werden.
    Dies lässt sich realisieren, indem zuerst von einem Quader ausgegangen wird, zu dem sich die Scheiben zusammen setzten lassen.
    Wie bei der ersten Problemstellung ermittelt der Algorithmus für eine der Varianten entweder eine Lösung und ist damit beendet,
    sollte er allerdings für keine der Varianten eine Lösung finden, gibt er nun nicht mehr zurück, dass es keine Lösung für die Menge an Scheiben gibt,
    sondern erhöht die Anzahl der möglichen Quader um eins und wiederholt das ganze.
    Eine Zusammensetzung von Varianten für mehrere Quader wird im folgenden Kombination genannt.
    Eine Beispiel für eine Kombination ist:

    \[
        Kombination = \begin{Bmatrix}
                          \begin{Bmatrix}
                              1 & 1 & 1
                          \end{Bmatrix} \\
                          \begin{Bmatrix}
                              3 & 3 & 3
                          \end{Bmatrix} \\
                          \begin{Bmatrix}
                              5 & 5 & 5
                          \end{Bmatrix} \\
        \end{Bmatrix}
    \]

    Das Gesamtvolumen aller Varianten in einer Kombination entspricht dabei immer dem Gesamtvolumen der Menge an Scheiben.
    In dieser Kombination können die Scheiben zu \hyperref[fig:figAB4]{drei Quadern} zusammengesetzt werden.

    Alle möglichen Varianten eines Quaders lassen sich wie in \hyperref[alg:euclid2]{\textttt{finde\_dimensionen}} zu sehen ermitteln.

    \begin{algorithm}[H]
        \caption{Ermittelt alle möglichen Varianten eines Quaders zu den gegebenen Parametern}
        \label{alg:euclid2}
        \begin{algorithmic}[1]
            \Require $volumen$, die minimale Seitenlänge $min$, die maximale Seitenlänge $max$
            \Ensure $dimensionen$
            \Function{finde\_dimensionen}{$volumen, min, max$}
                \State $dimensionen \gets \emptyset$
                \For{$l \gets min$ \textbf{to} $max$}
                    \For{$w \gets min$ \textbf{to} $max$}
                        \For{$h \gets min$ \textbf{to} $max$}
                            \If{$l \times w \times h \leq volumen$}
                                \State $tupel \gets (l, w, h)$
                                \State sortiere $tupel$ asc
                                \State $dimensionen \gets tupel$
                            \EndIf
                        \EndFor
                    \EndFor
                \EndFor
                \State \textbf{return} $dimensionen$
            \EndFunction
        \end{algorithmic}
    \end{algorithm}

    Die Menge an Kombinationen lässt sich dann ermitteln,
    indem rekursiv hyperref[alg:euclid2]{\textttt{finde\_dimensionen}} verwendet wird um jeweils alle Varianten zu bekommen.
    Dies ist in \hyperref[alg:euclid2]{\textttt{finde\_kombinationen}} dargestellt.

    \begin{algorithm}
        \caption{Ermittelt alle möglichen Kombinationen zu den gegebenen Parametern}\label{alg:euclid}
        \begin{algorithmic}[1]
            \Require $volumen, anzahl_{quader}, min, max, kombinationen, kombination_{aktuell}$
            \Function{finde\_kombinationen}{$volumen, anzahl_{quader}, min, max, kombinationen, kombination_{aktuell}$}
                \If{$anzahl_{quader} = 0$ \textbf{and} $volumen = 0$}
                    \State $kombinationen \gets kombination_{aktuell}$
                    \State \textbf{return}
                \EndIf
                \If{$volumen = 0$ \textbf{or} $anzahl_{quader} = 0$}
                    \State \textbf{return}
                \EndIf
                \State $dimensionen_{m\ddot oglich} \gets \textsc{finde\_dimensionen}(volume, min, max)$
                \For{$dimension \gets dimensionen_{m\ddot oglich}$}
                    \State $kombination_{aktuell} \gets dimension$
                    \State $volumen_{neu} = volume - dimension_{x1} \times dimension_{x2} \times dimension_{x3}$
                    \State $\textsc{finde\_kombinationen}(volumen_{neu}, anzahl_{quader} - 1, min, max, kombinationen, kombination_{aktuell})$
                    \State $kombination_{aktuell}$ entferne $dimension$
                \EndFor
            \EndFunction
        \end{algorithmic}
    \end{algorithm}

    Die dafür benötigten Parameter können aus der gegebenen Menge an Scheiben und dem daraus resultierenden Volumen abgeleitet werden.
    Nachdem alle Kombinationen gebildet wurden, ist es sinnvoll die Kombinationen zu entfernen, welche nicht die gewünschte anzahl an Quadern enthalten.

    \begin{algorithm}[H]
        \caption{Finde alle möglichen Kombinationen von Varianten für beliebig viele Quader}
        \label{alg:p}
        \begin{algorithmic}[1]
            \Require{ $volumen, anzahl_{quader}, scheiben$}
            \Ensure{Alle Kombinationen für die gegebenen Parameter}
            \Function{find\_all\_dimensions}{$volume, scheiben, anzahl_{quader}$}
                \State $combinations \gets$ Liste von leeren Mengen von Tupeln
                \State $currentCombination \gets$ leere Menge von Tupeln
                \State $max \gets 0$, $min \gets volume$
                \For{Jede Scheibe $scheibe$ in $scheiben$}
                    \State $side \gets \max(slice.length, slice.height)$
                    \If{$side > \max$}
                        \State $max \gets side$
                    \EndIf
                    \State $side \gets \min(slice.length, slice.height)$
                    \If{$side < min$}
                        \State $min \gets side$
                    \EndIf
                \EndFor
                \State \textbf{find\_combinations}($volume, anzahl_{quader}, \min, \max, combinations, currentCombination$)
                \For{$combination$ in $combinations$}
                    \If{Die Größe von $combination$ ist nicht gleich $Anzahl_{quader}$}
                        \State Entferne $combination$ aus $combinations$
                    \EndIf
                \EndFor
                \State \Return{$combinations$}\label{alg:algorithm}
            \EndFunction
        \end{algorithmic}
    \end{algorithm}

    \newpage


    \section{Umsetzung}\label{sec:umsetzung}

    Das Programm implementiert den beschriebenen Algorithmus, um zu prüfen, ob eine gegebene Menge an Scheiben zu einem Quader zusammengesetzt werden kann.

    Zunächst werden die notwendigen Datenstrukturen definiert, die in der Lösungsidee erwähnt werden.
    Es gibt eine Datenstruktur, um eine Käsescheibe darzustellen, die durch Länge und Höhe beschrieben wird.
    Dann gibt es eine Enumeration, um die verschiedenen Dimensionen eines Quaders zu definieren, sowie eine Funktion, um eine Dimension in einen String zu konvertieren.

    Die Hauptfunktion des Programms ist die Funktion can\_build\_cube, die die gegebene Menge an Scheiben und die Größe des Quaders als Parameter erhält.
    Die Funktion gibt einen boolschen Wert zurück, der angibt, ob die gegebene Menge an Scheiben zu einem Quader der gegebenen Größe zusammengesetzt werden kann.

    Die Implementierung der Funktion can\_build\_cube folgt dem in der Lösungsidee beschriebenen Algorithmus.
    Zunächst wird das Volumen des Quaders berechnet, indem die Länge, Höhe und Tiefe multipliziert werden.
    Anschließend wird die längste Kantenlänge aller Scheiben in der gegebenen Menge bestimmt.

    Danach wird für jede mögliche Variante des Zielquaders (dh für alle möglichen Kombinationen von Länge, Höhe und Tiefe, die das berechnete Volumen enthalten), überprüft, ob sich die gegebene Menge an Scheiben zu dieser Variante zusammensetzen lässt.

    Dazu wird für jede Käsescheibe in der gegebenen Menge getestet, ob sie in den Quader passt.
    Dazu werden die Längen und Höhen der Scheibe mit den Längen und Höhen des Quaders verglichen, um zu bestimmen, ob die Scheibe in der Vorder-, Ober- oder Seitenansicht entfernt werden kann.
    Wenn eine Scheibe in den Quader passt, wird sie aus der Liste der Scheiben entfernt und zur Lösungsliste hinzugefügt.
    Der Quader wird aktualisiert, indem die entsprechende Seitenlänge reduziert wird.

    Wenn eine Lösung gefunden wird, wird die Funktion true zurückgegeben.
    Andernfalls wird zur nächsten Variante des Zielquaders übergegangen und der Algorithmus fortgesetzt.
    Wenn es keine mögliche Variante gibt, zu der sich die gegebene Menge an Scheiben zusammensetzen lässt, wird false zurückgegeben.

    Insgesamt wurde die Lösungsidee gut umgesetzt.
    Der Algorithmus wird klar und verständlich durch den Programmcode widergespiegelt.
    Einige Konzepte, wie die Verwendung von Datenstrukturen, Enumerations und Funktionen, wurden elegant im Code modelliert.
    Der Code ist leicht verständlich und nachvollziehbar.


    Das Programm implementiert die Lösungsidee mithilfe einer Funktion can\_form\_cube, welche prüft, ob die gegebene Menge an Scheiben zu einem Quader zusammengesetzt werden kann.
    Wenn dies der Fall ist, wird true zurückgegeben und die Reihenfolge, in der die Scheiben zusammengesetzt werden können, in out\_slices geschrieben.
    Andernfalls wird false zurückgegeben und out\_slices bleibt leer.

    Die Funktion beginnt damit, das Gesamtvolumen aller Scheiben zu berechnen, um zu prüfen, ob es möglich ist, eine Größe für den Zielquader zu finden, die dieses Volumen aufnehmen kann.
    Wenn das Gesamtvolumen keinem ganzzahligen Quader entspricht, gibt die Funktion false zurück.

    Anschließend wird der größte Wert einer Scheibe in der Länge oder Höhe gesucht, da diese Dimensionen für den Quader maßgeblich sind.
    Daraus können alle möglichen Quadergrößen abgeleitet werden, die das berechnete Volumen aufnehmen können.
    Die möglichen Quadergrößen werden sortiert, um mit den kleinsten anzufangen.

    Die Funktion iteriert nun über alle möglichen Quadergrößen und versucht für jeden Quader, die Scheiben zusammenzusetzen.
    Dazu wird der Quader als aktueller Zustand initialisiert und eine Liste von Scheiben slices wird erzeugt, die alle Scheiben der Eingabe enthält.
    Dann wird eine Schleife gestartet, in der für jede Scheibe in der Liste slices versucht wird, sie in den Quader einzufügen.

    Dies erfolgt, indem die Scheibe in jeder möglichen Dimension des Quaders (vorne, oben und seitlich) getestet wird.
    Wenn eine Scheibe in eine Dimension passt, wird sie entfernt, die Lösungsmenge out\_slices wird um die Scheibe erweitert, und die Abmessungen des Quaders werden entsprechend aktualisiert.
    Wenn es keine weitere Scheibe gibt, die in den aktuellen Quader passt, wird der Algorithmus rekursiv aufgerufen, um zu versuchen, die restlichen Scheiben in den aktuellen Zustand einzufügen.
    Wenn dies erfolgreich ist, wird true zurückgegeben und die aktuelle Lösung wird zurückgegeben.

    Wenn keine der Scheiben in den aktuellen Quader passt, wird die Funktion false zurückgeben und die Funktion wird beendet.
    Wenn die Funktion false zurückgibt, wird die Schleife über alle möglichen Quadergrößen fortgesetzt, bis alle Größen ausprobiert wurden.
    Wenn keine der Größen passt, wird die Funktion false zurückgeben und das Programm endet.

    Die Funktion can\_form\_cube verwendet auch einige Hilfsfunktionen, um den Code übersichtlicher zu gestalten,
    zum Beispiel find\_possible\_cubes, um alle möglichen Quadergrößen zu finden, die das berechnete Volumen aufnehmen können, remove\_slice\_from\_list,
    um eine Scheibe aus einer Liste zu entfernen, und get\_dimension\_for\_slice, um die Dimension des Quaders zu finden, in die eine Scheibe am besten passt.

    Verwendung von Strukturen: Die Lösungsidee sieht vor, dass jede Scheibe durch Länge und Höhe beschrieben wird, während der Quader durch Länge, Breite und Höhe beschrieben wird.
    Im Programm wurde dies durch Verwendung einer Struktur für jede Scheibe und einer separaten Struktur für den Quader umgesetzt.
    Dadurch kann die Scheibenliste einfach an die Funktionen übergeben werden, ohne dass jeder Scheibe separat Länge und Höhe übergeben werden müssen.
    Verwendung von Aufzählungstypen: Die Lösungsidee definiert die Dimensionen "vorne/hinten", "oben/unten" und "links/rechts".
    Im Programm wurden diese Dimensionen durch Verwendung eines Aufzählungstyps (enum) implementiert.
    Dadurch kann sichergestellt werden, dass nur gültige Werte verwendet werden und es leicht ist, zwischen den Dimensionen zu wechseln.
    Verwendung von boolschen Werten: Die Lösungsidee sieht vor, dass die Funktion "TrySlice" entweder true oder false zurückgibt, je nachdem, ob die aktuelle Scheibe in den Quader passt oder nicht.
    Im Programm wird dies durch Verwendung eines boolschen Rückgabewerts umgesetzt.
    Dies macht es einfach zu prüfen, ob eine Lösung gefunden wurde oder nicht.
    Verwendung von Kommentaren: Im Programm wurden Kommentare verwendet, um zu beschreiben, was jeder Teil des Codes tut.
    Dadurch ist es einfacher zu verstehen, wie die Lösungsidee im Code umgesetzt wurde.
    Verwendung von Konstanten: Einige Werte, wie die maximale Anzahl von Varianten, wurden als Konstanten definiert.
    Dadurch kann sichergestellt werden, dass sie im gesamten Code einheitlich verwendet werden und es einfach ist, sie zu ändern, wenn nötig.

    Das Programm implementiert die Lösungsidee, indem es eine Funktion assemble\_slices enthält, die als Eingabe eine Menge von Käsescheiben und als Ausgabe die Reihenfolge der Käsescheiben zurückgibt, die zu einem Quader zusammengesetzt werden können.

    Die Implementierung der Funktion assemble\_slices folgt dem Algorithmus der Lösungsidee:

    Es wird das Volumen des Zielquaders berechnet.
    Es werden alle möglichen Dimensionen des Zielquaders ermittelt, die das berechnete Volumen beinhalten.
    Für jede mögliche Dimension wird überprüft, ob die gegebenen Scheiben zu einem Quader dieser Dimension zusammengesetzt werden können.
    Wenn eine Lösung gefunden wurde, wird die Reihenfolge der Scheiben zurückgegeben.
    Wenn für keine der möglichen Dimensionen eine Lösung gefunden wurde, wird eine leere Liste zurückgegeben.
    Die Funktion assemble\_slices ruft dabei die Funktion assemble\_slices\_helper auf, die den eigentlichen Greedy-Algorithmus implementiert,
    um zu prüfen, ob die gegebenen Käsescheiben zu einem Quader einer bestimmten Dimension zusammengesetzt werden können.
    Die Funktion assemble\_slices\_helper prüft für jede Käsescheibe, ob sie in einer der drei Dimensionen des Quaders platziert werden kann.
    Wenn dies der Fall ist, wird die Käsescheibe aus der Liste der noch nicht platzierten Käsescheiben entfernt und in die Lösungsliste aufgenommen.
    Die Größe des Quaders wird entsprechend angepasst und die Funktion wird rekursiv aufgerufen, um die verbleibenden Käsescheiben zu platzieren.
    Wenn keine der verbleibenden Käsescheiben in den verbleibenden Dimensionen des Quaders platziert werden kann, wird die letzte Käsescheibe aus der Lösungsliste entfernt, die Größe des Quaders wird wiederhergestellt und die Funktion gibt false zurück.

    Das Programm enthält auch die Strukturen Slice und Dimension, die jeweils eine Käsescheibe und eine Dimension des Quaders repräsentieren.
    Die Funktion to\_string gibt eine lesbare Beschreibung einer Dimension zurück, die im Programm für Debugging-Zwecke verwendet wird.

    Das folgende Programm implementiert eine Funktion, die eine gegebene Menge an Scheiben auf ihre Möglichkeit prüft, zu einem Quader zusammengesetzt zu werden.
    Es verwendet Backtracking, um die Lösung zu finden.

    Zunächst gibt es eine Struktur "Slice", die eine Käsescheibe mit Länge und Höhe repräsentiert.
    Dann gibt es eine Aufzählung "Dimension", die die drei Dimensionen Vorne, Oben und Seite sowie eine ungültige Dimension enthält.

    Die Funktion "can\_remove\_slice" prüft, ob eine Scheibe von einem Quader abgeschnitten werden kann, und gibt die Dimension zurück, in der die Scheibe entfernt werden kann.
    Wenn die Scheibe nicht entfernt werden kann, gibt die Funktion -1 zurück.

    Die Hauptfunktion "find\_solution" prüft zuerst, ob es überhaupt möglich ist, einen Quader aus den gegebenen Scheiben zusammenzusetzen.
    Wenn dies der Fall ist, wird für jede mögliche Variante des Zielquaders geprüft, ob die gegebene Menge an Scheiben zu dieser Variante zusammengesetzt werden kann.

    Die möglichen Varianten des Zielquaders werden durch die Kombination der längsten Scheibenlänge und des Quadvolumens berechnet.
    Dann wird für jede mögliche Variante eine Liste "solution" und eine Liste "slices" erstellt.
    Die "solution" Liste enthält die in der aktuellen Lösung enthaltenen Scheiben, während die "slices" Liste die noch nicht betrachteten Scheiben enthält.
    Der Quader wird durch die Längenparameter "length", "height" und "depth" repräsentiert.

    Die Hauptfunktion durchläuft die Liste der Scheiben und versucht, jede Scheibe in jeder möglichen Dimension des Quaders zu entfernen, um zu prüfen, ob eine Lösung gefunden wurde.
    Wenn eine Scheibe entfernt werden kann, wird sie zur "solution" Liste hinzugefügt, und die entsprechende Seitenlänge des Quaders wird reduziert.
    Die Funktion wird dann rekursiv aufgerufen, um die restlichen Scheiben zu entfernen.
    Wenn eine Lösung gefunden wird, wird "true" zurückgegeben, andernfalls werden die Änderungen rückgängig gemacht,
    indem die Scheibe aus der "solution" Liste entfernt und die Seitenlänge des Quaders wieder erhöht wird.

    Wenn alle Scheiben betrachtet wurden und keine Lösung gefunden wurde, wird "false" zurückgegeben.
    Wenn eine Lösung gefunden wurde, gibt die Hauptfunktion die Liste "solution" zurück, die die Scheiben in der Reihenfolge enthält, in der sie zum Aufbau des Quaders verwendet werden sollen.
    Wenn keine Lösung gefunden wurde, gibt die Hauptfunktion eine leere Liste zurück.

    \subsection{Allgemein}\label{subsec:allgemein2}

    \subsubsection{Methoden}\label{subsubsec:methoden}

    \subsubsection{Transformation}\label{subsubsec:transformation}

    \subsubsection{Implementationsdetails}\label{subsubsec:implementationsdetails}

    \subsection{Erweiterung}\label{subsec:erweiterung2}

    \subsubsection{Methoden}\label{subsubsec:methoden_erweiterung}

    Die Methode find\_dimensions verwendet eine Schleife, um alle möglichen Längen, Breiten und Höhen innerhalb der gegebenen Grenzen von min bis max zu durchlaufen.
    Für jede Kombination von Länge, Breite und Höhe wird das Volumen berechnet und mit dem gegebenen Volumen verglichen.
    Wenn das berechnete Volumen kleiner oder gleich dem gegebenen Volumen ist,
    wird ein Tupel aus der Länge, Breite und Höhe erstellt und nach der Sortierung in die Menge dimensions eingefügt.
    Am Ende gibt die Methode dimensions zurück.

    Die Methode find\_combinations verwendet eine rekursive Strategie,
    um alle möglichen Kombinationen von squares Quads mit einem gegebenen volume und min und max Längen, Breiten und Höhen zu finden.
    Die Methode prüft zuerst, ob die aktuelle Kombination vollständig ist und fügt sie, wenn dies der Fall ist, der Vektor combinations hinzu.
    Wenn das Volumen nicht durch die aktuelle Kombination aufgeht oder squares erreicht ist, wird die Methode abgebrochen.
    Andernfalls werden die möglichen Dimensionen für das nächste Quad berechnet, und die Methode wird rekursiv aufgerufen,
    indem jedes mögliche Tupel in die aktuelle Kombination eingefügt wird.
    Wenn die rekursive Aufrufe beendet sind, wird das Tupel aus der aktuellen Kombination entfernt, um Platz für das nächste mögliche Tupel zu machen.
    Am Ende gibt die Methode nichts zurück, da die Ergebnisse im Vektor combinations gespeichert werden.

    Die Methode verwendet die beiden gegebenen Methoden, \textbf{find\_dimensions} und \textbf{find\_combinations},
    um alle möglichen Kombinationen von Abmessungen für beliebig viele Quader zu finden.
    Zunächst werden die maximalen und minimalen Seitenlängen der Quader berechnet, indem die Längen und Höhen der gegebenen Scheiben verglichen werden.
    Dann wird die Methode \textbf{find\_combinations} aufgerufen, um rekursiv alle möglichen Kombinationen von Abmessungen für die gegebenen Parameter zu finden.
    Schließlich werden die Kombinationen, die nicht der erwarteten Anzahl von Quadern entsprechen, aus der Liste entfernt
    und die Liste der verbleibenden Kombinationen zurückgegeben.
    \newpage


    \section{Beispiele}\label{sec:beispiele}

    \begin{figure}[H]
        \centering
        \def\a{3.2}
        \def\b{1.2}
        \begin{subfigure}[b]{0.45\textwidth}
            \centering % b = bottom alignment
            \begin{tikzpicture}[3d view={115}{30},line cap=round,line join=round,scale=0.4]
                \smallSquare{(0,0,0)}{\a}{\b}{0}
            \end{tikzpicture}
            \caption{Zusammengesetzt}\label{fig:figA2}
        \end{subfigure}
        \begin{subfigure}[b]{0.45\textwidth}
            \centering % b = bottom alignment
            \begin{tikzpicture}[3d view={115}{30},line cap=round,line join=round,scale=0.4]
                \smallSquare{(0,9,0)}{\a}{\b}{1}
            \end{tikzpicture}
            \caption{Zerschnitten}\label{fig:figB2}
        \end{subfigure}
        \caption{Figur kleinerWuerfel.txt}\label{fig:figAB2}
    \end{figure}

    \begin{lstlisting}[frame=single, title=Programmausgabe kleinerWuerfel.txt, breaklines=true,label={lst:lstlisting3}]
    Die Scheiben können zu 1 Quader(n) zusammengesetzt werden.
    Quader: 2x2x2
    Slice: (2, 2) Dimension: front
    Slice: (2, 2) Dimension: front
    \end{lstlisting}

    - Simples Beispiel
    - Lösbar für Standard und Erweiterung



    \begin{figure}[H]
        \centering
        \def\a{3.2}
        \def\b{1.2}
        \begin{subfigure}[b]{0.45\textwidth}
            \centering % b = bottom alignment
            \begin{tikzpicture}[3d view={115}{30},line cap=round,line join=round,scale=0.4]
                \unsolvedSquare{(0,0,0)}{\a}{\b}{0}
            \end{tikzpicture}
            \caption{Zusammengesetzt}\label{fig:figA3}
        \end{subfigure}
        \begin{subfigure}[b]{0.45\textwidth}
            \centering % b = bottom alignment
            \begin{tikzpicture}[3d view={115}{30},line cap=round,line join=round,scale=0.4]
                \unsolvedSquare{(0,9,0)}{\a}{\b}{1}
            \end{tikzpicture}
            \caption{Zerschnitten}\label{fig:figB3}
        \end{subfigure}
        \caption{Figur zweiQuader.txt}\label{fig:figAB3}
    \end{figure}

    \begin{lstlisting}[frame=single, title=Programmausgabe zweiQuader.txt, breaklines=true,label={lst:lstlisting4}]
    Die Scheiben können zu 2 Quader(n) zusammengesetzt werden.
    Quader: 1x1x1
    Slice: (1, 1) Dimension: front

    Quader: 2x2x2
    Slice: (2, 2) Dimension: front
    Slice: (2, 2) Dimension: front
    \end{lstlisting}

    - Simples Beispiel
    - Lösbar nur für Erweiterung

    \newpage

    \begin{figure}[H]
        \centering
        \def\a{3.2}
        \def\b{1.2}
        \begin{subfigure}[b]{0.45\textwidth}
            \centering % b = bottom alignment
            \begin{tikzpicture}[3d view={115}{30},line cap=round,line join=round,scale=0.4]
                \begin{scope}
                    \simplecube{(0,     0,      0)}     {1}{1}{1}   {a}{a}{a}   {1}
                \end{scope}
            \end{tikzpicture}
            \begin{tikzpicture}[3d view={115}{30},line cap=round,line join=round,scale=0.4]
                \begin{scope}
                    \simplecube{(0,     0,      0)}     {1}{3}{3}   {a}{a}{a}   {1}
                    \simplecube{(1,   0,      0))}    {1}{3}{3}   {a}{a}{a}   {1}
                    \simplecube{(2,   0,      0))}    {1}{3}{3}   {a}{a}{a}   {1}
                \end{scope}
            \end{tikzpicture}
            \begin{tikzpicture}[3d view={115}{30},line cap=round,line join=round,scale=0.4]
                \begin{scope}
                    \simplecube{(0,     0,      0)}     {1}{5}{5}   {a}{a}{a}   {1}
                    \simplecube{(1,     0,      0)}     {1}{5}{5}   {a}{a}{a}   {1}
                    \simplecube{(2,     0,      0)}     {1}{5}{5}   {a}{a}{a}   {1}
                    \simplecube{(3,     0,      0)}     {1}{5}{5}   {a}{a}{a}   {1}
                    \simplecube{(4,     0,      0)}     {1}{5}{5}   {a}{a}{a}   {1}
                \end{scope}
            \end{tikzpicture}
            \caption{Zusammengesetzt}\label{fig:figA4}
        \end{subfigure}
        \begin{subfigure}[b]{0.45\textwidth}
            \centering % b = bottom alignment
            \begin{tikzpicture}[3d view={115}{30},line cap=round,line join=round,scale=0.4]
                \begin{scope}
                    \simplecube{(0,     0,      0)}     {1}{1}{1}   {a}{a}{a}   {1}
                \end{scope}
            \end{tikzpicture}
            \begin{tikzpicture}[3d view={115}{30},line cap=round,line join=round,scale=0.4]
                \begin{scope}
                    \simplecube{(0,     0,      0)}     {1}{3}{3}   {a}{a}{a}   {1}
                    \simplecube{(1+1,   0,      0))}    {1}{3}{3}   {a}{a}{a}   {1}
                    \simplecube{(2+2*1,   0,      0))}    {1}{3}{3}   {a}{a}{a}   {1}
                \end{scope}
            \end{tikzpicture}
            \begin{tikzpicture}[3d view={115}{30},line cap=round,line join=round,scale=0.4]
                \begin{scope}
                    \simplecube{(0,     0,      0)}     {1}{5}{5}   {a}{a}{a}   {1}
                    \simplecube{(1+1,     0,      0)}     {1}{5}{5}   {a}{a}{a}   {1}
                    \simplecube{(2+2*1,    0,      0)}     {1}{5}{5}   {a}{a}{a}   {1}
                    \simplecube{(3+3*1,     0,      0)}     {1}{5}{5}   {a}{a}{a}   {1}
                    \simplecube{(4+4*1,     0,      0)}     {1}{5}{5}   {a}{a}{a}   {1}
                \end{scope}
            \end{tikzpicture}
            \caption{Zerschnitten}\label{fig:figB4}
        \end{subfigure}
        \caption{Figur dreiQuader.txt}
        \label{fig:figAB4}
    \end{figure}

    \begin{lstlisting}[frame=single, title=Programmausgabe dreiQuader.txt, breaklines=true,label={lst:lstlisting5}]
    Die Scheiben können zu 3 Quader(n) zusammengesetzt werden.
    Quader: 1x1x1
    Slice: (1, 1) Dimension: front

    Quader: 3x3x3
    Slice: (3, 3) Dimension: front
    Slice: (3, 3) Dimension: front
    Slice: (3, 3) Dimension: front

    Quader: 5x5x5
    Slice: (5, 5) Dimension: front
    Slice: (5, 5) Dimension: front
    Slice: (5, 5) Dimension: front
    Slice: (5, 5) Dimension: front
    Slice: (5, 5) Dimension: front
    \end{lstlisting}

    - Zeigt Erweiterung auf beliebige Mengen an Quadern anwendbar

    \newpage

    \begin{figure}[H]
        \centering
        \def\a{3.2}
        \def\b{1.2}
        \begin{subfigure}[b]{0.45\textwidth}
            \centering % b = bottom alignment
            \begin{tikzpicture}[3d view={115}{30},line cap=round,line join=round,scale=0.4]
                \bigSquare{(0,0,0)}{\a}{\b}{0}
            \end{tikzpicture}
            \caption{Zusammengesetzt}\label{fig:figA1}
        \end{subfigure}
        \begin{subfigure}[b]{0.45\textwidth}
            \centering % b = bottom alignment
            \begin{tikzpicture}[3d view={115}{30},line cap=round,line join=round,scale=0.4]
                \bigSquare{(0,9,0)}{\a}{\b}{1}
            \end{tikzpicture}
            \caption{Zerschnitten}\label{fig:figB1}
        \end{subfigure}
        \caption{Figur kaese1.txt}\label{fig:figAB1}
    \end{figure}

    \begin{lstlisting}[frame=single, title=Programmausgabe kaese1.txt, breaklines=true,label={lst:lstlisting6}]
    Die Scheiben können zu einem Quader zusammengesetzt werden.
    Quader: 6x6x6 V(216)

    Slice: (6, 6) Dimension: front
    Slice: (6, 6) Dimension: front
    Slice: (4, 6) Dimension: top
    Slice: (4, 6) Dimension: top
    Slice: (4, 6) Dimension: front
    Slice: (3, 6) Dimension: top
    Slice: (3, 3) Dimension: side
    Slice: (3, 3) Dimension: side
    Slice: (3, 4) Dimension: front
    Slice: (2, 4) Dimension: top
    Slice: (2, 4) Dimension: front
    Slice: (2, 4) Dimension: front
    \end{lstlisting}

    - Beispiel 1 der Eingabedateien
    - erstes beispiel, wo Algorithmus backtracking macht
    - Lösbar nur Erweiterung nur mit sehr viel Zeit
    => betrachtung 3 vs 2 Variablen

    \begin{lstlisting}[frame=single, title=Programmausgabe kaese2.txt, breaklines=true,label={lst:lstlisting7}]
    Die Scheiben können zu einem Quader zusammengesetzt werden.
    Quader: 1000x2x1000 V(2000000)

    Slice: (2, 1000) Dimension: front
    Slice: (2, 1000) Dimension: front
    Slice: (2, 998) Dimension: top
    Slice: (998, 999) Dimension: side
    Slice: (998, 999) Dimension: side
    \end{lstlisting}

    - Beispiel 2 der Eingabedateien

    \newpage


    \begin{lstlisting}[frame=single, title=Programmausgabe kaese3.txt, breaklines=true,label={lst:lstlisting8}]
    Die Scheiben können zu einem Quader zusammengesetzt werden.
    Quader: 1000x10x1000 V(10000000)

    Slice: (1000, 1000) Dimension: side
    Slice: (1000, 1000) Dimension: side
    Slice: (1000, 1000) Dimension: side
    Slice: (7, 1000) Dimension: front
    Slice: (999, 1000) Dimension: side
    Slice: (6, 1000) Dimension: front
    Slice: (998, 1000) Dimension: side
    Slice: (5, 998) Dimension: top
    Slice: (998, 999) Dimension: side
    Slice: (4, 999) Dimension: front
    Slice: (4, 997) Dimension: top
    Slice: (4, 998) Dimension: front
    Slice: (4, 998) Dimension: front
    Slice: (4, 995) Dimension: top
    Slice: (995, 997) Dimension: side
    Slice: (995, 997) Dimension: side
    Slice: (2, 997) Dimension: front
    Slice: (2, 994) Dimension: top
    Slice: (2, 996) Dimension: front
    Slice: (2, 993) Dimension: top
    Slice: (2, 995) Dimension: front
    Slice: (992, 995) Dimension: side
    Slice: (992, 995) Dimension: side
    \end{lstlisting}

    - Beispiel 3 der Eingabedateien

    \begin{lstlisting}[frame=single, title=Ausschnitt der Programmausgabe kaese4.txt, breaklines=true,label={lst:lstlisting9}]
    Die Scheiben können zu einem Quader zusammengesetzt werden.
    Quader: 210x210x210 V(9261000)
    [...]
    \end{lstlisting}

    - Ausschnitt Beispiel 4 der Eingabedateien
    (Anzahl der Scheiben zu groß zum darstellen)

    \begin{lstlisting}[frame=single, title=Ausschnitt der Programmausgabe kaese5.txt, breaklines=true,label={lst:lstlisting10}]
    Die Scheiben können zu einem Quader zusammengesetzt werden.
    Quader: 3570x2310x2730 V(1038654520)
    [...]
    \end{lstlisting}

    - Ausschnitt Beispiel 5 der Eingabedateien
    (Anzahl der Scheiben zu groß zum darstellen)

    \begin{lstlisting}[frame=single, title=Programmausgabe kaese6.txt, breaklines=true,label={lst:lstlisting11}]
    Die Scheiben können zu keinem Quader zusammengesetzt werden.
    \end{lstlisting}

    - Ausschnitt Beispiel 6 der Eingabedateien
    - Algorithmus findet keine Lösung
    - zu groß für Erweiterung

    \begin{lstlisting}[frame=single, title=Programmausgabe kaese7.txt, breaklines=true,label={lst:lstlisting12}]
    Die Scheiben können zu keinem Quader zusammengesetzt werden.
    \end{lstlisting}

    - Ausschnitt Beispiel 7 der Eingabedateien
    - Algorithmus findet keine Lösung
    - zu groß für Erweiterung


    \newpage


    \section{Quellcode}
    \label{sec:quellcode}
    \label{LastPage}
    \lstset{language=C++,
        keywordstyle=\color{magenta},
        stringstyle=\color{red},
        commentstyle=\color{green},
    }

    \subsection{Header-File}\label{subsec:header-file}

    \subsubsection{Implementierung Slice}

    \begin{lstlisting}[frame=single,language=C++,title=Struct Slice,breaklines=true,label={lst:code_slice}]
    /**
     * Repräsentiert eine Käsescheibe
     */
    struct Slice {
        Slice(int p_length, int p_height) {
            this->length = p_length;
            this->height = p_height;
        }

        int length, height;
    };
    \end{lstlisting}

    \subsubsection{Implementierung Dimension}
    \begin{lstlisting}[frame=single,language=C++,title=Enum Dimension,breaklines=true,label={lst:code_dimension}]
    /**
     * Repräsentiert eine Dimension
     */
    enum Dimension{
        FRONT,
        TOP,
        SIDE,
        INVALID
    };
    \end{lstlisting}

    \subsubsection{Implementierung Hilfsmethode}
    \begin{lstlisting}[frame=single,language=C++,title=Methode can\_remove\_slice,breaklines=true,label={lst:code_canRemoveSlice}]
    /**
     * Prüft, ob eine Scheibe, von einem Quader abgeschnitten werden kann
     * @param length die Länge des Quaders
     * @param height die Höhe des Quaders
     * @param depth die Tiefe des Quaders
     * @param slice die Scheibe, auf die getestet wird
     * @return -1, wenn die Scheibe nicht entfernt werden kann,
     * ansonsten wird die Dimension zurückgegeben, wo diese entfernt werden kann
     * Dimensionen:
     * 0 => VORNE
     * 1 => OBEN
     * 2 => SEITE
     */
    Dimension can_remove_slice(int length, int height, int depth, Slice slice) {
        if (slice.length == length && slice.height == height || slice.height == length && slice.length == height) {
            return FRONT;
        } else if (slice.length == length && slice.height == depth || slice.height == length && slice.length == depth) {
            return TOP;
        } else if (slice.length == height && slice.height == depth || slice.height == height && slice.length == depth) {
            return SIDE;
        } else {
            return INVALID;
        }
    }
    \end{lstlisting}

    \newpage

    \subsubsection{Implementierung Algorithmus}
    \begin{lstlisting}[frame=single,language=C++,title=Methode calculate\_square,breaklines=true,label={lst:code_calculateSquare}]
    /**
     * Ermittelt rekursiv durch Backtracking eine Lösungsreihenfolge für die gegebene Menge an Scheiben.
     * @param length die Länge des Quaders
     * @param height die Höhe des Quaders
     * @param depth die Tiefe des Quaders
     * @param order die Lösungsreihenfolge
     * @param slices die noch nicht verwendete Menge an Scheiben
     * @return true, wenn es eine lösung gibt, ansonsten false
     */
    bool calculate_square(int length, int height, int depth, std::vector<std::pair<Slice, Dimension>> &order, std::vector<Slice> &slices) {

        //Wenn mindestens eine der Seiten auf null ist (daher das Volumen des Quaders null ist)
        if (length == 0 || height == 0 || depth == 0) {
            return true;
        }
        std::vector<Slice> removed_slices;
        //Für jede noch nicht verwendete Schiebe
        for (auto it = slices.begin(); it != slices.end(); ++it) {
            Dimension dimension = can_remove_slice(length, height, depth, *it);
            //Wenn die aktuelle Scheibe abgeschnitten werden kann
            if (dimension != INVALID) {
                //Aktualisiere die Maße des Quaders
                int new_length = length - (dimension == SIDE ? 1 : 0);
                int new_height = height - (dimension == TOP ? 1 : 0);
                int new_depth = depth - (dimension == FRONT ? 1 : 0);
                //Füge die Scheibe zur Lösungsmenge hinzu
                order.emplace_back(*it, dimension);
                removed_slices.push_back(*it);
                slices.erase(it);
                //Wenn es eine Lösung mit der aktuellen Scheibe gibt
                if (calculate_square(new_length, new_height, new_depth, order, slices)) {
                    return true;
                } else {
                    //Entferne die aktuelle Scheibe von der Lösungsmenge
                    order.pop_back();
                    for (auto &slice: removed_slices) {
                        slices.push_back(slice);
                    }
                    removed_slices.clear();
                }
            }
        }
        //Wenn keine der noch nicht betrachteten Scheiben verwendet werden kann
        return false;
    }
    \end{lstlisting}

    \newpage

    \subsection{Aufgabe-A}\label{subsec:alles-kaese}

    \subsubsection{Implementierung Obermethode}
    \begin{lstlisting}[frame=single,language=C++,title=Methode main,breaklines=true,label={lst:code_main}]
    /**
     * Liest die Eingabedateien ein und versucht für jede Datei eine Lösung entsprechend der Aufgabenstellung zu finden
     * Die Lösung wird anschließend in die entsprechende Ausgabedatei geschrieben
     * Sollte es keine Lösung geben, wird dies ebenfalls in die Ausgabedatei geschrieben
     * @return 0, wenn es zu keinem RuntimeError oder keiner RuntimeException gekommen ist
     */
    int main() {
        string input_dir = "../LennartProtte/Aufgabe2-Implementierung/Eingabedateien";
        string output_dir = "../LennartProtte/Aufgabe2-Implementierung/Ausgabedateien";

        //Durchläuft alle Dateien im Eingabeordner
        for (const auto &entry: filesystem::directory_iterator(input_dir)) {

            //Liest den Dateinamen aus
            string input_file = entry.path();
            string output_file = output_dir + "/" + entry.path().filename().string();

            //Öffnet die Eingabedatei
            ifstream fin(input_file);

            //Öffnet die Ausgabedatei
            ofstream fout(output_file);

            //Liest die Eingabedatei ein
            vector<Slice> slices;
            int a, b, n;
            fin >> n;
            while (fin >> a >> b) {
                slices.emplace_back(a, b);
            }

            //Berechnet das Volumen des Quaders
            int volume = 0, height = 0;
            for (const auto &slice: slices) {
                volume += slice.length * slice.height;
            }

            //Findet die Seite, wo der größte Wert maximal ist und setzt die Länge auf diesen Wert
            for (const auto &slice: slices) {
                int side = (slice.length > slice.height) ? slice.length : slice.height;
                if (side > height) {
                    height = side;
                }
            }

            //Findet alle anderen möglichen Seiten zu der gesetzten Länge
            vector<pair<int, int>> result;
            int base = volume / height;
            for(int side_a = 1; side_a <= base; side_a++) {
                for(int side_b = 1; side_b <= base; side_b++) {
                    if(side_a * side_b * height == volume) {
                        result.emplace_back(side_a, side_b);
                    }
                }
            }

            //Sortiert die Scheiben nach ihrer Fläche
            vector<pair<Slice, Dimension>> order;
            sort(slices.begin(), slices.end(), [](Slice a, Slice b) {
                     return (a.length * a.height) > (b.length * b.height);
                 }
            );

            auto it = result.begin();
            bool success = false;
            //Für jede mögliche Kombination der Seitenlängen
            while (it != result.end()) {
                int t_height = height;
                vector<Slice> t_slices(slices);
                order.clear();
                //Wenn es eine Lösung gibt
                if (calculate_square( it->first, t_height, it->second, order, t_slices)) {
                    success = true;
                    fout << "Die Scheiben können zu einem Quader zusammengesetzt werden." << endl;
                    fout << "Quader: " << height << "x" << it->first << "x" << it->second << " V(" << volume << ")" << endl
                         << endl;
                    for (auto const &o: order) {
                        fout << "Slice: (" << o.first.length << ", " << o.first.height << ") Dimension: " << to_string(o.second)
                             << endl;
                    }
                    break;
                }
                it++;
            }
            //Wenn es keine Lösung gab
            if (!success) {
                fout << "Die Scheiben können zu keinem Quader zusammengesetzt werden." << endl;
            }
            //Dateien schließen
            fin.close();
            fout.close();
        }
        return 0;
    }
    \end{lstlisting}

    \subsection{Erweiterung}\label{subsec:erweiterung}

    \subsubsection{Implementierung Sortiermethode}

    \begin{lstlisting}[frame=single,language=C++,title=Methode sort\_tupel,breaklines=true,label={lst:code_sortTupel}]
    /**
     * Sortiert Tupel mit drei Elementen in aufsteigender Reihenfolge
     * @param tupel der zu sortierende Tupel
     */
    void sort_tupel(tuple<int, int, int> &tupel) {
        int temp;
        if (get<0>(tupel) > get<1>(tupel)) {
            temp = get<0>(tupel);
            get<0>(tupel) = get<1>(tupel);
            get<1>(tupel) = temp;
        }
        if (get<1>(tupel) > get<2>(tupel)) {
            temp = get<1>(tupel);
            get<1>(tupel) = get<2>(tupel);
            get<2>(tupel) = temp;
            if (get<0>(tupel) > get<1>(tupel)) {
                temp = get<0>(tupel);
                get<0>(tupel) = get<1>(tupel);
                get<1>(tupel) = temp;
            }
        }
    }
    \end{lstlisting}

    \newpage

    \subsubsection{Implementierung Hilfsmethode}
    \begin{lstlisting}[frame=single,language=C++,title=Methode find\_dimensions,breaklines=true,label={lst:code_findDimensions}]
    /**
     * Ermittelt alle möglichen ganzzahligen Kombinationen von Längen, Breiten und Höhen für einen Quader
     * @param volume das maximale Volumen des Quaders
     * @param min die kürzeste Seite des Quaders
     * @param max die längste Seite des Quaders
     * @return alle möglichen Varianten des Quaders
     */
    unordered_set<tuple<int, int, int>, TupelHash> find_dimensions(int volume, const int &min, const int &max) {
        unordered_set<tuple<int, int, int>, TupelHash> dimensions;
        for (int l = min; l <= max; l++) {
            for (int w = min; w <= max; w++) {
                for (int h = min; h <= max; h++) {
                    if (l * w * h <= volume) {
                        tuple<int, int, int> tupel = make_tuple(l, w, h);
                        sort_tupel(tupel);
                        dimensions.insert(tupel);
                    }
                }
            }
        }
        return dimensions;
    }
    \end{lstlisting}

    \subsubsection{Implementierung Algorithmus}
    \begin{lstlisting}[frame=single,language=C++,title=Methode find\_combinations,breaklines=true,label={lst:code_findCombinations}]
    /**
     * Ermittelt rekursiv alle möglichen Kombinationen von Abmessungen für beliebig viele Quader
     * @param volume das Gesamtvolumen für jede Kombination
     * @param squares die maximale Anzahl an Quadern in einer Kombination
     * @param min die minimale Länge einer Seite eines Quaders
     * @param max die maximale Länge einer Seite eines Quaders
     * @param combinations alle Kombinationen
     * @param currentCombination die aktuelle Kombination
     */
    void find_combinations(int volume,
                           int squares,
                           const int &min,
                           const int &max,
                           vector<unordered_set<tuple<int, int, int>, TupelHash>> &combinations,
                           unordered_set<tuple<int, int, int>, TupelHash> &currentCombination) {
        //Wenn die aktuelle Kombination vollständig ist
        if (squares == 0 && volume == 0) {
            combinations.push_back(currentCombination);
            return;
        }
        // Wenn das Volumen die aktuelle Kombination nicht aufgeht
        if (volume == 0 || squares == 0) {
            return;
        }
        //Berechnet rekursiv den nächsten Quader der aktuellen Kombination
        unordered_set<tuple<int, int, int>, TupelHash> possibleDimensions = find_dimensions(volume, min, max);
        for (auto dimension: possibleDimensions) {
            currentCombination.insert(dimension);
            find_combinations(volume - get<0>(dimension) * get<1>(dimension) * get<2>(dimension),
                              squares - 1,
                              min,
                              max,
                              combinations,
                              currentCombination);
            currentCombination.erase(dimension);
        }
    }
    \end{lstlisting}

    \begin{lstlisting}[frame=single,language=C++,title=Methode find\_all\_combinations,breaklines=true,label={lst:code_findAllCombinations}]
    /**
     * Ermittelt alle möglichen Kombinationen von Abmessungen für beliebig viele Quader
     * @param volume das maximale Gesamtvolumen für jede Kombination
     * @param count_of_squares die Anzahl der Quader in jeder Kombination
     * @param slices die gegebene Menge an Scheiben
     * @return alle Kombinationen für die gegebenen Parameter
     */
    vector<unordered_set<tuple<int, int, int>, TupelHash>> find_all_combinations(int volume,
                                                                                 int count_of_squares,
                                                                                 const vector<Slice> &slices) {
        vector<unordered_set<tuple<int, int, int>, TupelHash>> combinations;
        unordered_set<tuple<int, int, int>, TupelHash> currentCombination;

        int max = 0, min = volume;
        //Ermittelt die maximale Seitenlänge eines Quaders
        for (const auto &slice: slices) {
            int side = (slice.length > slice.height) ? slice.length : slice.height;
            if (side > max) {
                max = side;
            }
        }
        //Ermittelt die minimale Seitenlänge eines Quaders
        for (const auto &slice: slices) {
            int side = (slice.length < slice.height) ? slice.length : slice.height;
            if (side < min) {
                min = side;
            }
        }

        find_combinations(volume, count_of_squares, min, max, combinations, currentCombination);
        //Entfernt die Kombinationen, welche nicht der erwarteten Anzahl von Quadern entsprechen
        combinations.erase(
                std::remove_if(combinations.begin(), combinations.end(),
                               [&count_of_squares](auto combination) {
                                   return combination.size() != count_of_squares;
                               }), combinations.end()
        );
        return combinations;
    }
    \end{lstlisting}

    \subsubsection{Implementierung Obermethode}
    \begin{lstlisting}[frame=single,language=C++,title=Methode main,breaklines=true,label={lst:code_main_extended}]
    /**
     * Liest die Eingabedateien ein und versucht für jede Datei eine Lösung entsprechend der Aufgabenstellung zu finden
     * Die Lösung wird anschließend in die entsprechende Ausgabedatei geschrieben
     * Sollte es keine Lösung geben, wird dies ebenfalls in die Ausgabedatei geschrieben
     * @return 0, wenn es zu keinem RuntimeError oder keiner RuntimeException gekommen ist
     */
    int main() {
        string input_dir = "../LennartProtte/Aufgabe2-Implementierung/Eingabedateien_b";
        string output_dir = "../LennartProtte/Aufgabe2-Implementierung/Ausgabedateien_b";

        //Durchläuft alle Dateien im Eingabeordner
        for (const auto &entry: filesystem::directory_iterator(input_dir)) {

            //Liest den Dateinamen aus
            string input_file = entry.path();
            string output_file = output_dir + "/" + entry.path().filename().string();

            //Öffnet die Eingabedatei
            ifstream fin(input_file);

            //Öffnet die Ausgabedatei
            ofstream fout(output_file);

            //Liest die Eingabedatei ein
            vector<Slice> slices;
            int a, b, n;
            fin >> n;
            while (fin >> a >> b) {
                slices.emplace_back(a, b);
            }

            //Berechnet das Volumen des Quaders
            int volume = 0;
            for (const auto &slice: slices) {
                volume += slice.length * slice.height;
            }

            //Sortiert die Scheiben nach ihrer Fläche
            sort(slices.begin(), slices.end(), [](Slice a, Slice b) {
                     return (a.length * a.height) > (b.length * b.height);
                 }
            );

            vector<pair<Slice, Dimension>> order;
            map<tuple<int, int, int>, vector<pair<Slice, Dimension>>> solution;
            bool success = false;
            //Erhöht die Anzahl der Würfel schrittweise, bis es eine Lösung gibt
            for (int count_of_squares = 1; count_of_squares <= slices.size(); count_of_squares++) {
                const auto &combinations = find_all_combinations(volume, count_of_squares, slices);
                for (const auto &combination: combinations) {
                    bool valid = true;
                    vector<Slice> t_slices(slices);
                    for (auto dimension: combination) {
                        order.clear();
                        if (!calculate_square(get<0>(dimension), get<1>(dimension), get<2>(dimension), order, t_slices)) {
                            valid = false;
                        } else {
                            solution.insert(make_pair(dimension, order));
                        }
                    }
                    if (valid && t_slices.empty()) {
                        success = true;
                        goto end;
                    } else {
                        solution.clear();
                    }
                }
            }
            end:
            //Schreibt die Ausgabe
            if (success) {
                fout << "Die Scheiben können zu " << solution.size() << " Quader(n) zusammengesetzt werden." << endl;
                for (auto & it : solution) {
                    fout << "Quader: " << get<0>(it.first) << "x" << get<1>(it.first) << "x" << get<2>(it.first) << endl;
                    for (auto item: it.second) {
                        fout << "Slice: (" << item.first.length << ", " << item.first.height
                             << ") Dimension: " << to_string(item.second) << endl;
                    }
                    fout << endl;
                }
            } else {
                fout << "Die Scheiben können zu keinem Quader zusammengesetzt werden." << endl;
            }

            //Dateien schließen
            fin.close();
            fout.close();
        }
        return 0;
    }
    \end{lstlisting}

\end{document}
