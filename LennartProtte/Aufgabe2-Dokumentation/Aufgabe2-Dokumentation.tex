%! Author = leartpro
%! Date = 04.01.23

\documentclass[a4paper,10pt,ngerman]{scrartcl}
\usepackage{babel}
\usepackage[T1]{fontenc}
\usepackage[utf8x]{inputenc}
\usepackage[a4paper,margin=2.5cm,footskip=0.5cm]{geometry}

% Die nächsten drei Felder bitte anpassen:
\newcommand{\Aufgabe}{Aufgabe 2: Alles Käse} % Aufgabennummer und Aufgabennamen angeben
\newcommand{\TeilnahmeId}{67275}                        % Teilnahme-ID angeben
\newcommand{\Name}{Lennart Protte}                      % Name des Bearbeiter / der Bearbeiterin dieser Aufgabe angeben

% Kopf- und Fußzeilen
\usepackage{scrlayer-scrpage, lastpage}
\setkomafont{pageheadfoot}{\large\textrm}
\lohead{\Aufgabe}
\rohead{Teilnahme-ID: \TeilnahmeId}
\cfoot*{\thepage{}/\pageref{LastPage}}

% Position des Titels
\usepackage{titling}
\setlength{\droptitle}{-1.0cm}

% Für mathematische Befehle und Symbole
\usepackage{amsmath}
\usepackage{amssymb}

% Für Bilder
\usepackage{graphicx}

%Für Überschriften
\usepackage[labelformat=empty]{caption}
\captionsetup[algorithm]{labelformat=empty}

% Für Algorithmen
\usepackage{algpseudocode}
\usepackage{adigraph}
\usepackage{algorithm}
\usepackage{algorithmicx}
\usepackage{tikz}
\usepackage{textcomp}

% Für Beispiele
\usepackage{tikz-3dplot}
\usepackage{subcaption}


% Für Quelltext
\usepackage{listings}
\lstset{
    \usepackage{color}
    keywordstyle=\color{blue},commentstyle=\color{mygreen},
    stringstyle=\color{mymauve},rulecolor=\color{black},
    basicstyle=\footnotesize\ttfamily,numberstyle=\tiny\color{mygray},
    captionpos=b, % sets the caption-position to bottom
    keepspaces=true, % keeps spaces in text
    numbers=left, numbersep=5pt, showspaces=false,showstringspaces=true,
    showtabs=false, stepnumber=2, tabsize=2, title=\lstname ,
    inputencoding = utf8,  % Input encoding
    extendedchars = true,  % Extended ASCII
    literate      =        % Support additional characters
        {á}{{\'a}}1  {é}{{\'e}}1  {í}{{\'i}}1 {ó}{{\'o}}1  {ú}{{\'u}}1
        {Á}{{\'A}}1  {É}{{\'E}}1  {Í}{{\'I}}1 {Ó}{{\'O}}1  {Ú}{{\'U}}1
        {à}{{\`a}}1  {è}{{\`e}}1  {ì}{{\`i}}1 {ò}{{\`o}}1  {ù}{{\`u}}1
        {À}{{\`A}}1  {È}{{\'E}}1  {Ì}{{\`I}}1 {Ò}{{\`O}}1  {Ù}{{\`U}}1
        {ä}{{\"a}}1  {ë}{{\"e}}1  {ï}{{\"i}}1 {ö}{{\"o}}1  {ü}{{\"u}}1
        {Ä}{{\"A}}1  {Ë}{{\"E}}1  {Ï}{{\"I}}1 {Ö}{{\"O}}1  {Ü}{{\"U}}1
        {â}{{\^a}}1  {ê}{{\^e}}1  {î}{{\^i}}1 {ô}{{\^o}}1  {û}{{\^u}}1
        {Â}{{\^A}}1  {Ê}{{\^E}}1  {Î}{{\^I}}1 {Ô}{{\^O}}1  {Û}{{\^U}}1
        {œ}{{\oe}}1  {Œ}{{\OE}}1  {æ}{{\ae}}1 {Æ}{{\AE}}1  {ß}{{\ss}}1
        {ç}{{\c c}}1 {Ç}{{\c C}}1 {ø}{{\o}}1  {Ø}{{\O}}1   {å}{{\r a}}1
        {Å}{{\r A}}1 {ã}{{\~a}}1  {õ}{{\~o}}1 {Ã}{{\~A}}1  {Õ}{{\~O}}1
        {ñ}{{\~n}}1  {Ñ}{{\~N}}1  {¿}{{?`}}1  {¡}{{!`}}1
        {°}{{\textdegree}}1 {º}{{\textordmasculine}}1 {ª}{{\textordfeminine}}1
}

\usetikzlibrary{3d}          % for 'canvas is...' options
\usetikzlibrary{perspective} % for '3d view' option
\tikzset
{
    linea/.style={draw=red},
    lineb/.style={draw=blue},
}

\newcommand{\simplecube}[8]% origin, dimension x, dimension y, dimension z, style x, style y, style z
{
    \begin{scope}[shift={#1}]
        \fill[gray!40,canvas is yz plane at x=#2, opacity=#8] (0,0) rectangle (#3,#4);
        \fill[gray!10,canvas is xz plane at y=#3, opacity=#8] (0,0) rectangle (#2,#4);
        \fill[white  ,canvas is xy plane at z=#4, opacity=#8] (0,0) rectangle (#2,#3);
        \foreach\i/\j in {0/1, 1/1, 1/0}
            {
            \draw[line#5] (0,#3*\i,#4*\j) --++ (#2,0,0);
            \draw[line#6] (#2*\i,0,#4*\j) --++ (0,#3,0);
            \draw[line#7] (#2*\i,#3*\j,0) --++ (0,0,#4);
        }
    \end{scope}
}

\newcommand{\bigSquare}[4]% origin, a, b, separation
{
    \begin{scope}[shift={#1}]
        \simplecube{(0,     0,      0)}     {1}{4}{2}   {a}{a}{a}   {1}
        \simplecube{(1+#4,   0,      0))}    {1}{4}{2}   {a}{a}{a}   {1}
        \simplecube{(0,     0,      2+#4)}   {2}{4}{1}   {a}{a}{a}   {1}
        \simplecube{(2+2*#4,   0,      0)}     {1}{4}{3}   {a}{a}{a}   {0.8}
        \simplecube{(0,     4+#4,    0)}     {3}{1}{3}   {a}{a}{a}   {0.8}
        \simplecube{(0,     5+2*#4,    0)}     {3}{1}{3}   {a}{a}{a}   {0.5}
        \simplecube{(0,     0,      3+2*#4)}   {3}{6}{1}   {a}{a}{a}   {0.8}
        \simplecube{(3+3*#4,   0,      0))}    {1}{6}{4}   {a}{a}{a}   {0.5}
        \simplecube{(0,     0,      4+3*#4)}   {4}{6}{1}   {a}{a}{a}   {0.5}
        \simplecube{(0,     0,      5+4*#4)}   {4}{6}{1}   {a}{a}{a}   {0.2}
        \simplecube{(4+4*#4,   0,      0))}    {1}{6}{6}   {a}{a}{a}   {0.2}
        \simplecube{(5+5*#4,   0,      0))}    {1}{6}{6}   {a}{a}{a}   {0}
    \end{scope}
}

\newcommand{\smallSquare}[4]% origin, a, b, separation
{
    \begin{scope}[shift={#1}]
        \simplecube{(0,     0,      0)}     {1}{2}{2}   {a}{a}{a}   {1}
        \simplecube{(1+#4,   0,      0))}    {1}{2}{2}   {a}{a}{a}   {1}
    \end{scope}
}

\newcommand{\unsolvedSquare}[4]% origin, a, b, separation
{
    \begin{scope}[shift={#1}]
        \simplecube{(0,     0,      0)}     {1}{2}{2}   {a}{a}{a}   {1}
        \simplecube{(1+#4,   0,      0))}    {1}{2}{2}   {a}{a}{a}   {1}
        \simplecube{(2+2*#4,   0,      0))}    {1}{1}{1}   {a}{a}{a}   {1}
    \end{scope}
}

\newcommand{\threeSquares}[4]% origin, a, b, separation
{
    \begin{scope}[shift={0}]
        \simplecube{(0,     0,      0)}     {1}{1}{1}   {a}{a}{a}   {1}
    \end{scope}
    \begin{scope}[shift={5}]
        \simplecube{(0,     0,      0)}     {1}{3}{3}   {a}{a}{a}   {1}
        \simplecube{(1+#4,   0,      0))}    {1}{3}{3}   {a}{a}{a}   {1}
        \simplecube{(2+2*#4,   0,      0))}    {1}{3}{3}   {a}{a}{a}   {1}
    \end{scope}
    \begin{scope}[shift={7}]
        \simplecube{(0,     0,      0)}     {1}{5}{5}   {a}{a}{a}   {1}
        \simplecube{(1+#4,,     0,      0)}     {1}{5}{5}   {a}{a}{a}   {1}
        \simplecube{(2+2*#4,    0,      0)}     {1}{5}{5}   {a}{a}{a}   {1}
        \simplecube{(3+3*#4,    0,      0)}     {1}{5}{5}   {a}{a}{a}   {1}
        \simplecube{(4+4*#4,    0,      0)}     {1}{5}{5}   {a}{a}{a}   {1}
    \end{scope}
}

% Diese beiden Pakete müssen zuletzt geladen werden
\usepackage{hyperref} % Anklickbare Links im Dokument
\usepackage{cleveref}

% Daten für die Titelseite
\title{\textbf{\Huge\Aufgabe}}
\author{\LARGE Teilnahme-ID: \LARGE \TeilnahmeId \\\\
\LARGE Bearbeiter dieser Aufgabe: \\
\LARGE \Name\\\\}
\date{\LARGE\today}

\begin{document}

    \maketitle
    \tableofcontents
    \vspace{0.5cm}
    \newpage


    \section{Lösungsidee}\label{sec:losungsidee}

    \subsection{Allgemein}\label{subsec:allgemein}

    Der Aufgabenstellung ist entnehmen, dass eine gegebene Menge an Scheiben, wobei zu jede Scheibe durch
    eine Länge und eine Höhe beschreiben wird, geprüft werden soll, ob diese zu einem Quader zusammengesetzt werden können.
    Wenn dies der Fall ist, soll auch die Reihenfolge angegeben werden, in der die Scheiben zu einem, Quader zusammengesetzt werden können.,
    Dabei müssen in der Lösungsreihenfolge alle Scheiben der gegebenen Menge vorkommen.
    Die Aufgabenstellung stellt dabei keine Ansprüche, an die Form des Zielquaders, allerdings erscheint es nicht im Sinne der Aufgabenstellung,
    dass der Zielquader Lücken, beziehungsweise Hohlräume aufweist.

    Beim Zusammensetzten der Scheiben zu einem Quader ist zu beachten, dass diese nicht zwangsläufig alle von der gleichen Seite des Quaders stammen müssen,
    daher müssen alle Dimensionen des Quaders betrachtet werden.
    Aufgrund der mathematischen Eigenschaften eines Quaders, ist es ausreichend, wenn nur drei der sechs Seiten betrachtet werden,
    da parallele Seiten bei einem Quader identisch sind.

    Aus der Menge der gegebenen Scheiben lässt sich das Volumen des Quaders berechnen.
    Aus diesem und der längsten Seite, die eine Scheibe der gegebenen Menge besitzt, lassen sich alle möglichen Varianten des Zielquaders bilden.
    Aufgrund der ganzzahligen Scheibendimensionen, kann angenommen werden, dass auch die Dimensionen des Zielquaders ganzzahlig sind.
    Es müssen daher nur Quader mit ganzzahligen Dimensionen betrachtet werden, welche das berechnete Volumen beinhalten.

    Nun kann für jede mögliche Variante des Zielquaders, mit einem Greedy-Algorithmus, geprüft werden,
    ob sich die gegebene Menge an Scheiben zu dieser Variante zusammenzusetzen lässt.
    Sobald dies der Fall ist, ist eine Lösung gefunden.
    Sollte dies nicht der Fall sein, wird die nächste Variante betrachtet.
    Sollte es für keine der möglichen Varianten eine Lösung geben, ist es nicht möglich die gegebene Menge an Scheiben
    zu einem Quader zusammenzusetzen.

    Um zu prüfen, ob eine Variante in die gegebene Menge an Schieben zerlegt werden kann, wird jede
    Scheibe in drei möglichen Dimensionen des Quaders (Vorne, Oben, Seite) getestet.
    Wenn eine der beiden Seitenlängen der Scheibe mit einer der Seitenlängen des Quaders übereinstimmt,
    kann die Scheibe in der entsprechenden Dimension entfernt werden.
    Wenn dies der Fall ist, wird sie zur Lösungsmenge $O$ hinzugefügt und aus der Liste $S$ entfernt.
    Der Quader wird aktualisiert, indem die entsprechende Seitenlänge reduziert wird.
    Wenn es eine Lösung mit der aktuellen Scheibe gibt, gibt die Funktion ``true'' zurück.
    Andernfalls werden die Änderungen rückgängig gemacht, indem die Scheibe wieder zur Liste $S$ hinzugefügt wird
    und die letzte Scheibe aus der Lösungsmenge $O$ entfernt wird.
    Wenn keine der noch nicht betrachteten Scheiben entfernt werden kann, gibt die Funktion ``false'' zurück.

    \begin{algorithm}[H]
        \caption{Berechnung eines Käsequaders aus gegebenen Käsescheiben}
        \begin{algorithmic}
            \Require Die Länge $l$, Höhe $h$ und Tiefe $d$ des Käsequaders, eine Liste $S$ von Käsescheiben, eine leere Liste $O$
            \Ensure Ein boolescher Wert, der angibt, ob die Käsescheiben zu einem vollständigen Käsequader zusammengesetzt werden können,
            und eine Lösungsreihenfolge $O$, in der die Käsescheiben zusammengesetzt werden können.
            \Function{calculate\_square}{$\textit{l},\ \textit{h},\ \textit{d},\ \textit{S},\ \textit{O}$}
                \If{$l$ = 0 or $h$ = 0 or $d$ = 0}
                    \State \Return true
                \EndIf
                \State $R_{removed} \gets \emptyset$
                \For{$s \in S$}
                    \State $D_{dimension} \gets \textsc{can\_remove\_slice}$
                    \If{$D_{dimension} \neq UNG\ddot ULTIG$}
                        \State $l_{new} \gets l - (D_{dimension} = SIDE ? 1 : 0)$
                        \State $h_{new}  \gets h - (D_{dimension} = TOP ? 1 : 0)$
                        \State $d_{new}  \gets d - (D_{dimension} = FRONT ? 1 : 0)$
                        \State $O \gets {slice, dimension}$
                        \State $R_{removed} \gets s$
                        \State $S \notin s$
                    \EndIf
                    \If{$\textsc{calculate\_square}(l_{new},h_{new},d_{new},S,O)$}
                        \State \Return true
                    \Else
                        \State $O \notin last$
                        \For{$r \in R_{removed}$}
                            \State $slices \gets r$
                        \EndFor
                        \State $R_{removed} \gets \emptyset$
                        \EndElse
                    \EndIf
                \EndFor
                \State \Return false
            \EndFunction
        \end{algorithmic}\label{alg:pseudo_greedy}
    \end{algorithm}


    \begin{algorithm}[H]
        \caption{Bestimmung der passenden Seite des Quaders zur Scheibe}
        \label{alg:cheese2}
        \begin{algorithmic}
            \Function{can\_remove\_slice}{$l, h, d, s$}
                \If{($s_{l} = l$ \land $s_{h} = h$) \lor ($s_{l} = h$ \land $s_{h} = l$)}
                    \State \Return $VORNE$
                \ElsIf
                        {($s_{l} = l$ \land $s_{d} = d$) \lor ($s_{l} = d$ \land $s_{d} = l$)}
                    \State \Return $OBEN$
                \ElsIf
                        {($s_{h} = h$ \land $s_{d} = d$) \lor ($s_{h} = d$ \land $s_{d} = h$)}
                    \State \Return $SEITE$
                \Else
                    \State \Return $UNG\ddot ULTIG$
                \EndIf
            \EndFunction
        \end{algorithmic}
    \end{algorithm}

    \subsection{Erweiterung}\label{subsec:erweiterung_losungsidee}

    \newpage


    \section{Umsetzung}\label{sec:umsetzung}

    - wie die Lösungsidee im Programm tatsächlich umgesetzt wurde
    - auch Implementierungsdetails
    - Transformation von Idee ins Programm
    - Bezug zwischen der Lösungsidee und der Implementation
    - in welchen Methoden die Algorithmen aus der Lösungsidee implementiert sind
    - wie bestimmte Konzepte der Lösungsidee elegant im Programm modelliert wurden

    \subsection{Allgemein}\label{subsec:allgemein2}

    \subsubsection{Methoden}\label{subsubsec:methoden}
    \subsubsection{Transformation}\label{subsubsec:transformation}
    \subsubsection{Implementationsdetails}\label{subsubsec:implementationsdetails}

    \subsection{Erweiterung}\label{subsec:erweiterung2}

    \subsubsection{Methoden}\label{subsubsec:methoden_erweiterung}
    \subsubsection{Transformation}\label{subsubsec:transformation_erweiterung}
    \subsubsection{Implementationsdetails}\label{subsubsec:implementationsdetails_erweiterung}

    \newpage


    \section{Beispiele}\label{sec:beispiele}

    \begin{figure}[H]
        \centering
        \def\a{3.2}
        \def\b{1.2}
        \begin{subfigure}[b]{0.45\textwidth}
            \centering % b = bottom alignment
            \begin{tikzpicture}[3d view={115}{30},line cap=round,line join=round,scale=0.4]
                \smallSquare{(0,0,0)}{\a}{\b}{0}
            \end{tikzpicture}
            \caption{Zusammengesetzt}\label{fig:figA2}
        \end{subfigure}
        \begin{subfigure}[b]{0.45\textwidth}
            \centering % b = bottom alignment
            \begin{tikzpicture}[3d view={115}{30},line cap=round,line join=round,scale=0.4]
                \smallSquare{(0,9,0)}{\a}{\b}{1}
            \end{tikzpicture}
            \caption{Zerschnitten}\label{fig:figB2}
        \end{subfigure}
        \caption{Figur kleinerWuerfel.txt}\label{fig:figAB2}
    \end{figure}

    \begin{lstlisting}[frame=single, title=Programmausgabe kleinerWuerfel.txt, breaklines=true,label={lst:lstlisting3}]
Die Scheiben können zu 1 Quader(n) zusammengesetzt werden.
Quader: 2x2x2
Slice: (2, 2) Dimension: front
Slice: (2, 2) Dimension: front
    \end{lstlisting}

    - Simples Beispiel
    - Lösbar für Standard und Erweiterung


    \begin{figure}[H]
        \centering
        \def\a{3.2}
        \def\b{1.2}
        \begin{subfigure}[b]{0.45\textwidth}
            \centering % b = bottom alignment
            \begin{tikzpicture}[3d view={115}{30},line cap=round,line join=round,scale=0.4]
                \unsolvedSquare{(0,0,0)}{\a}{\b}{0}
            \end{tikzpicture}
            \caption{Zusammengesetzt}\label{fig:figA3}
        \end{subfigure}
        \begin{subfigure}[b]{0.45\textwidth}
            \centering % b = bottom alignment
            \begin{tikzpicture}[3d view={115}{30},line cap=round,line join=round,scale=0.4]
                \unsolvedSquare{(0,9,0)}{\a}{\b}{1}
            \end{tikzpicture}
            \caption{Zerschnitten}\label{fig:figB3}
        \end{subfigure}
        \caption{Figur zweiQuader.txt}\label{fig:figAB3}
    \end{figure}

    \begin{lstlisting}[frame=single, title=Programmausgabe zweiQuader.txt, breaklines=true,label={lst:lstlisting4}]
Die Scheiben können zu 2 Quader(n) zusammengesetzt werden.
Quader: 1x1x1
Slice: (1, 1) Dimension: front

Quader: 2x2x2
Slice: (2, 2) Dimension: front
Slice: (2, 2) Dimension: front
    \end{lstlisting}

    - Simples Beispiel
    - Lösbar nur für Erweiterung

    \newpage

    \begin{figure}[H]
        \centering
        \def\a{3.2}
        \def\b{1.2}
        \begin{subfigure}[b]{0.45\textwidth}
            \centering % b = bottom alignment
            \begin{tikzpicture}[3d view={115}{30},line cap=round,line join=round,scale=0.4]
                \begin{scope}
                    \simplecube{(0,     0,      0)}     {1}{1}{1}   {a}{a}{a}   {1}
                \end{scope}
            \end{tikzpicture}
            \begin{tikzpicture}[3d view={115}{30},line cap=round,line join=round,scale=0.4]
                \begin{scope}
                    \simplecube{(0,     0,      0)}     {1}{3}{3}   {a}{a}{a}   {1}
                    \simplecube{(1,   0,      0))}    {1}{3}{3}   {a}{a}{a}   {1}
                    \simplecube{(2,   0,      0))}    {1}{3}{3}   {a}{a}{a}   {1}
                \end{scope}
            \end{tikzpicture}
            \begin{tikzpicture}[3d view={115}{30},line cap=round,line join=round,scale=0.4]
                \begin{scope}
                    \simplecube{(0,     0,      0)}     {1}{5}{5}   {a}{a}{a}   {1}
                    \simplecube{(1,     0,      0)}     {1}{5}{5}   {a}{a}{a}   {1}
                    \simplecube{(2,     0,      0)}     {1}{5}{5}   {a}{a}{a}   {1}
                    \simplecube{(3,     0,      0)}     {1}{5}{5}   {a}{a}{a}   {1}
                    \simplecube{(4,     0,      0)}     {1}{5}{5}   {a}{a}{a}   {1}
                \end{scope}
            \end{tikzpicture}
            \caption{Zusammengesetzt}\label{fig:figA4}
        \end{subfigure}
        \begin{subfigure}[b]{0.45\textwidth}
            \centering % b = bottom alignment
            \begin{tikzpicture}[3d view={115}{30},line cap=round,line join=round,scale=0.4]
                \begin{scope}
                    \simplecube{(0,     0,      0)}     {1}{1}{1}   {a}{a}{a}   {1}
                \end{scope}
            \end{tikzpicture}
            \begin{tikzpicture}[3d view={115}{30},line cap=round,line join=round,scale=0.4]
                \begin{scope}
                    \simplecube{(0,     0,      0)}     {1}{3}{3}   {a}{a}{a}   {1}
                    \simplecube{(1+1,   0,      0))}    {1}{3}{3}   {a}{a}{a}   {1}
                    \simplecube{(2+2*1,   0,      0))}    {1}{3}{3}   {a}{a}{a}   {1}
                \end{scope}
            \end{tikzpicture}
            \begin{tikzpicture}[3d view={115}{30},line cap=round,line join=round,scale=0.4]
                \begin{scope}
                    \simplecube{(0,     0,      0)}     {1}{5}{5}   {a}{a}{a}   {1}
                    \simplecube{(1+1,     0,      0)}     {1}{5}{5}   {a}{a}{a}   {1}
                    \simplecube{(2+2*1,    0,      0)}     {1}{5}{5}   {a}{a}{a}   {1}
                    \simplecube{(3+3*1,     0,      0)}     {1}{5}{5}   {a}{a}{a}   {1}
                    \simplecube{(4+4*1,     0,      0)}     {1}{5}{5}   {a}{a}{a}   {1}
                \end{scope}
            \end{tikzpicture}
            \caption{Zerschnitten}\label{fig:figB4}
        \end{subfigure}
        \caption{Figur dreiQuader.txt}\label{fig:figAB4}
    \end{figure}

    \begin{lstlisting}[frame=single, title=Programmausgabe dreiQuader.txt, breaklines=true,label={lst:lstlisting5}]
Die Scheiben können zu 3 Quader(n) zusammengesetzt werden.
Quader: 1x1x1
Slice: (1, 1) Dimension: front

Quader: 3x3x3
Slice: (3, 3) Dimension: front
Slice: (3, 3) Dimension: front
Slice: (3, 3) Dimension: front

Quader: 5x5x5
Slice: (5, 5) Dimension: front
Slice: (5, 5) Dimension: front
Slice: (5, 5) Dimension: front
Slice: (5, 5) Dimension: front
Slice: (5, 5) Dimension: front
    \end{lstlisting}

    - Zeigt Erweiterung auf beliebige Mengen an Quadern anwendbar

    \newpage

    \begin{figure}[H]
        \centering
        \def\a{3.2}
        \def\b{1.2}
        \begin{subfigure}[b]{0.45\textwidth}
            \centering % b = bottom alignment
            \begin{tikzpicture}[3d view={115}{30},line cap=round,line join=round,scale=0.4]
                \bigSquare{(0,0,0)}{\a}{\b}{0}
            \end{tikzpicture}
            \caption{Zusammengesetzt}\label{fig:figA1}
        \end{subfigure}
        \begin{subfigure}[b]{0.45\textwidth}
            \centering % b = bottom alignment
            \begin{tikzpicture}[3d view={115}{30},line cap=round,line join=round,scale=0.4]
                \bigSquare{(0,9,0)}{\a}{\b}{1}
            \end{tikzpicture}
            \caption{Zerschnitten}\label{fig:figB1}
        \end{subfigure}
        \caption{Figur kaese1.txt}\label{fig:figAB1}
    \end{figure}

    \begin{lstlisting}[frame=single, title=Programmausgabe kaese1.txt, breaklines=true,label={lst:lstlisting6}]
Die Scheiben können zu einem Quader zusammengesetzt werden.
Quader: 6x6x6 V(216)

Slice: (6, 6) Dimension: front
Slice: (6, 6) Dimension: front
Slice: (4, 6) Dimension: top
Slice: (4, 6) Dimension: top
Slice: (4, 6) Dimension: front
Slice: (3, 6) Dimension: top
Slice: (3, 3) Dimension: side
Slice: (3, 3) Dimension: side
Slice: (3, 4) Dimension: front
Slice: (2, 4) Dimension: top
Slice: (2, 4) Dimension: front
Slice: (2, 4) Dimension: front

    \end{lstlisting}

    - Beispiel 1 der Eingabedateien
    - erstes beispiel, wo Algorithmus backtracking macht
    - Lösbar nur Erweiterung nur mit sehr viel Zeit
    => betrachtung 3 vs 2 Variablen

    \begin{lstlisting}[frame=single, title=Programmausgabe kaese2.txt, breaklines=true,label={lst:lstlisting7}]
Die Scheiben können zu einem Quader zusammengesetzt werden.
Quader: 1000x2x1000 V(2000000)

Slice: (2, 1000) Dimension: front
Slice: (2, 1000) Dimension: front
Slice: (2, 998) Dimension: top
Slice: (998, 999) Dimension: side
Slice: (998, 999) Dimension: side

    \end{lstlisting}

    - Beispiel 2 der Eingabedateien

    \newpage


    \begin{lstlisting}[frame=single, title=Programmausgabe kaese3.txt, breaklines=true,label={lst:lstlisting8}]
Die Scheiben können zu einem Quader zusammengesetzt werden.
Quader: 1000x10x1000 V(10000000)

Slice: (1000, 1000) Dimension: side
Slice: (1000, 1000) Dimension: side
Slice: (1000, 1000) Dimension: side
Slice: (7, 1000) Dimension: front
Slice: (999, 1000) Dimension: side
Slice: (6, 1000) Dimension: front
Slice: (998, 1000) Dimension: side
Slice: (5, 998) Dimension: top
Slice: (998, 999) Dimension: side
Slice: (4, 999) Dimension: front
Slice: (4, 997) Dimension: top
Slice: (4, 998) Dimension: front
Slice: (4, 998) Dimension: front
Slice: (4, 995) Dimension: top
Slice: (995, 997) Dimension: side
Slice: (995, 997) Dimension: side
Slice: (2, 997) Dimension: front
Slice: (2, 994) Dimension: top
Slice: (2, 996) Dimension: front
Slice: (2, 993) Dimension: top
Slice: (2, 995) Dimension: front
Slice: (992, 995) Dimension: side
Slice: (992, 995) Dimension: side
    \end{lstlisting}

    - Beispiel 3 der Eingabedateien

    \begin{lstlisting}[frame=single, title=Ausschnitt der Programmausgabe kaese4.txt, breaklines=true,label={lst:lstlisting9}]
Die Scheiben können zu einem Quader zusammengesetzt werden.
Quader: 210x210x210 V(9261000)
[...]
    \end{lstlisting}

    - Ausschnitt Beispiel 4 der Eingabedateien
    (Anzahl der Scheiben zu groß zum darstellen)

    \begin{lstlisting}[frame=single, title=Ausschnitt der Programmausgabe kaese5.txt, breaklines=true,label={lst:lstlisting10}]
Die Scheiben können zu einem Quader zusammengesetzt werden.
Quader: 3570x2310x2730 V(1038654520)
[...]
    \end{lstlisting}

    - Ausschnitt Beispiel 5 der Eingabedateien
    (Anzahl der Scheiben zu groß zum darstellen)

    \begin{lstlisting}[frame=single, title=Programmausgabe kaese6.txt, breaklines=true,label={lst:lstlisting11}]
Die Scheiben können zu keinem Quader zusammengesetzt werden.
    \end{lstlisting}

    - Ausschnitt Beispiel 6 der Eingabedateien
    - Algorithmus findet keine Lösung
    - zu groß für Erweiterung

    \begin{lstlisting}[frame=single, title=Programmausgabe kaese7.txt, breaklines=true,label={lst:lstlisting12}]
Die Scheiben können zu keinem Quader zusammengesetzt werden.
    \end{lstlisting}

    - Ausschnitt Beispiel 7 der Eingabedateien
    - Algorithmus findet keine Lösung
    - zu groß für Erweiterung


    \newpage


    \section{Quellcode}
    \label{sec:quellcode}
    \label{LastPage}
    \lstset{language=C++,
        keywordstyle=\color{magenta},
        stringstyle=\color{red},
        commentstyle=\color{green},
    }

    \subsection{Header-File}\label{subsec:header-file}

    \subsubsection{Implementierung Slice}

    \begin{lstlisting}[frame=single,language=C++,title=Struct Slice,breaklines=true,label={lst:code_slice}]
    /**
     * Repräsentiert eine Käsescheibe
     */
    struct Slice {
        Slice(int p_length, int p_height) {
            this->length = p_length;
            this->height = p_height;
        }

        int length, height;
    };
    \end{lstlisting}

    \subsubsection{Implementierung Dimension}
    \begin{lstlisting}[frame=single,language=C++,title=Enum Dimension,breaklines=true,label={lst:code_dimension}]
    /**
     * Repräsentiert eine Dimension
     */
    enum Dimension{
        FRONT,
        TOP,
        SIDE,
        INVALID
    };
    \end{lstlisting}

    \subsubsection{Implementierung Hilfsmethode}
    \begin{lstlisting}[frame=single,language=C++,title=Methode can\_remove\_slice,breaklines=true,label={lst:code_canRemoveSlice}]
    /**
     * Prüft, ob eine Scheibe, von einem Quader abgeschnitten werden kann
     * @param length die Länge des Quaders
     * @param height die Höhe des Quaders
     * @param depth die Tiefe des Quaders
     * @param slice die Scheibe, auf die getestet wird
     * @return -1, wenn die Scheibe nicht entfernt werden kann,
     * ansonsten wird die Dimension zurückgegeben, wo diese entfernt werden kann
     * Dimensionen:
     * 0 => VORNE
     * 1 => OBEN
     * 2 => SEITE
     */
    Dimension can_remove_slice(int length, int height, int depth, Slice slice) {
        if (slice.length == length && slice.height == height || slice.height == length && slice.length == height) {
            return FRONT;
        } else if (slice.length == length && slice.height == depth || slice.height == length && slice.length == depth) {
            return TOP;
        } else if (slice.length == height && slice.height == depth || slice.height == height && slice.length == depth) {
            return SIDE;
        } else {
            return INVALID;
        }
    }
    \end{lstlisting}

    \newpage

    \subsubsection{Implementierung Algorithmus}
    \begin{lstlisting}[frame=single,language=C++,title=Methode calculate\_square,breaklines=true,label={lst:code_calculateSquare}]
    /**
     * Ermittelt rekursiv durch Backtracking eine Lösungsreihenfolge für die gegebene Menge an Scheiben.
     * @param length die Länge des Quaders
     * @param height die Höhe des Quaders
     * @param depth die Tiefe des Quaders
     * @param order die Lösungsreihenfolge
     * @param slices die noch nicht verwendete Menge an Scheiben
     * @return true, wenn es eine lösung gibt, ansonsten false
     */
    bool calculate_square(int length, int height, int depth, std::vector<std::pair<Slice, Dimension>> &order, std::vector<Slice> &slices) {

        //Wenn mindestens eine der Seiten auf null ist (daher das Volumen des Quaders null ist)
        if (length == 0 || height == 0 || depth == 0) {
            return true;
        }
        std::vector<Slice> removed_slices;
        //Für jede noch nicht verwendete Schiebe
        for (auto it = slices.begin(); it != slices.end(); ++it) {
            Dimension dimension = can_remove_slice(length, height, depth, *it);
            //Wenn die aktuelle Scheibe abgeschnitten werden kann
            if (dimension != INVALID) {
                //Aktualisiere die Maße des Quaders
                int new_length = length - (dimension == SIDE ? 1 : 0);
                int new_height = height - (dimension == TOP ? 1 : 0);
                int new_depth = depth - (dimension == FRONT ? 1 : 0);
                //Füge die Scheibe zur Lösungsmenge hinzu
                order.emplace_back(*it, dimension);
                removed_slices.push_back(*it);
                slices.erase(it);
                //Wenn es eine Lösung mit der aktuellen Scheibe gibt
                if (calculate_square(new_length, new_height, new_depth, order, slices)) {
                    return true;
                } else {
                    //Entferne die aktuelle Scheibe von der Lösungsmenge
                    order.pop_back();
                    for (auto &slice: removed_slices) {
                        slices.push_back(slice);
                    }
                    removed_slices.clear();
                }
            }
        }
        //Wenn keine der noch nicht betrachteten Scheiben verwendet werden kann
        return false;
    }
    \end{lstlisting}

    \newpage

    \subsection{Aufgabe-A}\label{subsec:alles-kaese}

    \subsubsection{Implementierung Obermethode}
    \begin{lstlisting}[frame=single,language=C++,title=Methode main,breaklines=true,label={lst:code_main}]
    /**
     * Liest die Eingabedateien ein und versucht für jede Datei eine Lösung entsprechend der Aufgabenstellung zu finden
     * Die Lösung wird anschließend in die entsprechende Ausgabedatei geschrieben
     * Sollte es keine Lösung geben, wird dies ebenfalls in die Ausgabedatei geschrieben
     * @return 0, wenn es zu keinem RuntimeError oder keiner RuntimeException gekommen ist
     */
    int main() {
        string input_dir = "../LennartProtte/Aufgabe2-Implementierung/Eingabedateien";
        string output_dir = "../LennartProtte/Aufgabe2-Implementierung/Ausgabedateien";

        //Durchläuft alle Dateien im Eingabeordner
        for (const auto &entry: filesystem::directory_iterator(input_dir)) {

            //Liest den Dateinamen aus
            string input_file = entry.path();
            string output_file = output_dir + "/" + entry.path().filename().string();

            //Öffnet die Eingabedatei
            ifstream fin(input_file);

            //Öffnet die Ausgabedatei
            ofstream fout(output_file);

            //Liest die Eingabedatei ein
            vector<Slice> slices;
            int a, b, n;
            fin >> n;
            while (fin >> a >> b) {
                slices.emplace_back(a, b);
            }

            //Berechnet das Volumen des Quaders
            int volume = 0, height = 0;
            for (const auto &slice: slices) {
                volume += slice.length * slice.height;
            }

            //Findet die Seite, wo der größte Wert maximal ist und setzt die Länge auf diesen Wert
            for (const auto &slice: slices) {
                int side = (slice.length > slice.height) ? slice.length : slice.height;
                if (side > height) {
                    height = side;
                }
            }

            //Findet alle anderen möglichen Seiten zu der gesetzten Länge
            vector<pair<int, int>> result;
            int base = volume / height;
            for(int side_a = 1; side_a <= base; side_a++) {
                for(int side_b = 1; side_b <= base; side_b++) {
                    if(side_a * side_b * height == volume) {
                        result.emplace_back(side_a, side_b);
                    }
                }
            }

            //Sortiert die Scheiben nach ihrer Fläche
            vector<pair<Slice, Dimension>> order;
            sort(slices.begin(), slices.end(), [](Slice a, Slice b) {
                     return (a.length * a.height) > (b.length * b.height);
                 }
            );

            auto it = result.begin();
            bool success = false;
            //Für jede mögliche Kombination der Seitenlängen
            while (it != result.end()) {
                int t_height = height;
                vector<Slice> t_slices(slices);
                order.clear();
                //Wenn es eine Lösung gibt
                if (calculate_square( it->first, t_height, it->second, order, t_slices)) {
                    success = true;
                    fout << "Die Scheiben können zu einem Quader zusammengesetzt werden." << endl;
                    fout << "Quader: " << height << "x" << it->first << "x" << it->second << " V(" << volume << ")" << endl
                         << endl;
                    for (auto const &o: order) {
                        fout << "Slice: (" << o.first.length << ", " << o.first.height << ") Dimension: " << to_string(o.second)
                             << endl;
                    }
                    break;
                }
                it++;
            }
            //Wenn es keine Lösung gab
            if (!success) {
                fout << "Die Scheiben können zu keinem Quader zusammengesetzt werden." << endl;
            }
            //Dateien schließen
            fin.close();
            fout.close();
        }
        return 0;
    }
    \end{lstlisting}

    \subsection{Erweiterung}\label{subsec:erweiterung}

    \subsubsection{Implementierung Sortiermethode}

    \begin{lstlisting}[frame=single,language=C++,title=Methode sort\_tupel,breaklines=true,label={lst:code_sortTupel}]
    /**
     * Sortiert Tupel mit drei Elementen in aufsteigender Reihenfolge
     * @param tupel der zu sortierende Tupel
     */
    void sort_tupel(tuple<int, int, int> &tupel) {
        int temp;
        if (get<0>(tupel) > get<1>(tupel)) {
            temp = get<0>(tupel);
            get<0>(tupel) = get<1>(tupel);
            get<1>(tupel) = temp;
        }
        if (get<1>(tupel) > get<2>(tupel)) {
            temp = get<1>(tupel);
            get<1>(tupel) = get<2>(tupel);
            get<2>(tupel) = temp;
            if (get<0>(tupel) > get<1>(tupel)) {
                temp = get<0>(tupel);
                get<0>(tupel) = get<1>(tupel);
                get<1>(tupel) = temp;
            }
        }
    }
    \end{lstlisting}

    \newpage

    \subsubsection{Implementierung Hilfsmethode}
    \begin{lstlisting}[frame=single,language=C++,title=Methode find\_dimensions,breaklines=true,label={lst:code_findDimensions}]
    /**
     * Ermittelt alle möglichen ganzzahligen Kombinationen von Längen, Breiten und Höhen für einen Quader
     * @param volume das maximale Volumen des Quaders
     * @param min die kürzeste Seite des Quaders
     * @param max die längste Seite des Quaders
     * @return alle möglichen Varianten des Quaders
     */
    unordered_set<tuple<int, int, int>, TupelHash> find_dimensions(int volume, const int &min, const int &max) {
        unordered_set<tuple<int, int, int>, TupelHash> dimensions;
        for (int l = min; l <= max; l++) {
            for (int w = min; w <= max; w++) {
                for (int h = min; h <= max; h++) {
                    if (l * w * h <= volume) {
                        tuple<int, int, int> tupel = make_tuple(l, w, h);
                        sort_tupel(tupel);
                        dimensions.insert(tupel);
                    }
                }
            }
        }
        return dimensions;
    }
    \end{lstlisting}

    \subsubsection{Implementierung Algorithmus}
    \begin{lstlisting}[frame=single,language=C++,title=Methode find\_combinations,breaklines=true,label={lst:code_findCombinations}]
    /**
     * Ermittelt rekursiv alle möglichen Kombinationen von Abmessungen für beliebig viele Quader
     * @param volume das Gesamtvolumen für jede Kombination
     * @param squares die maximale Anzahl an Quadern in einer Kombination
     * @param min die minimale Länge einer Seite eines Quaders
     * @param max die maximale Länge einer Seite eines Quaders
     * @param combinations alle Kombinationen
     * @param currentCombination die aktuelle Kombination
     */
    void find_combinations(int volume,
                           int squares,
                           const int &min,
                           const int &max,
                           vector<unordered_set<tuple<int, int, int>, TupelHash>> &combinations,
                           unordered_set<tuple<int, int, int>, TupelHash> &currentCombination) {
        //Wenn die aktuelle Kombination vollständig ist
        if (squares == 0 && volume == 0) {
            combinations.push_back(currentCombination);
            return;
        }
        // Wenn das Volumen die aktuelle Kombination nicht aufgeht
        if (volume == 0 || squares == 0) {
            return;
        }
        //Berechnet rekursiv den nächsten Quader der aktuellen Kombination
        unordered_set<tuple<int, int, int>, TupelHash> possibleDimensions = find_dimensions(volume, min, max);
        for (auto dimension: possibleDimensions) {
            currentCombination.insert(dimension);
            find_combinations(volume - get<0>(dimension) * get<1>(dimension) * get<2>(dimension),
                              squares - 1,
                              min,
                              max,
                              combinations,
                              currentCombination);
            currentCombination.erase(dimension);
        }
    }
    \end{lstlisting}

    \begin{lstlisting}[frame=single,language=C++,title=Methode find\_all\_combinations,breaklines=true,label={lst:code_findAllCombinations}]
    /**
     * Ermittelt alle möglichen Kombinationen von Abmessungen für beliebig viele Quader
     * @param volume das maximale Gesamtvolumen für jede Kombination
     * @param count_of_squares die Anzahl der Quader in jeder Kombination
     * @param slices die gegebene Menge an Scheiben
     * @return alle Kombinationen für die gegebenen Parameter
     */
    vector<unordered_set<tuple<int, int, int>, TupelHash>> find_all_combinations(int volume,
                                                                                 int count_of_squares,
                                                                                 const vector<Slice> &slices) {
        vector<unordered_set<tuple<int, int, int>, TupelHash>> combinations;
        unordered_set<tuple<int, int, int>, TupelHash> currentCombination;

        int max = 0, min = volume;
        //Ermittelt die maximale Seitenlänge eines Quaders
        for (const auto &slice: slices) {
            int side = (slice.length > slice.height) ? slice.length : slice.height;
            if (side > max) {
                max = side;
            }
        }
        //Ermittelt die minimale Seitenlänge eines Quaders
        for (const auto &slice: slices) {
            int side = (slice.length < slice.height) ? slice.length : slice.height;
            if (side < min) {
                min = side;
            }
        }

        find_combinations(volume, count_of_squares, min, max, combinations, currentCombination);
        //Entfernt die Kombinationen, welche nicht der erwarteten Anzahl von Quadern entsprechen
        combinations.erase(
                std::remove_if(combinations.begin(), combinations.end(),
                               [&count_of_squares](auto combination) {
                                   return combination.size() != count_of_squares;
                               }), combinations.end()
        );
        return combinations;
    }
    \end{lstlisting}

    \subsubsection{Implementierung Obermethode}
    \begin{lstlisting}[frame=single,language=C++,title=Methode main,breaklines=true,label={lst:code_main_extended}]
    /**
     * Liest die Eingabedateien ein und versucht für jede Datei eine Lösung entsprechend der Aufgabenstellung zu finden
     * Die Lösung wird anschließend in die entsprechende Ausgabedatei geschrieben
     * Sollte es keine Lösung geben, wird dies ebenfalls in die Ausgabedatei geschrieben
     * @return 0, wenn es zu keinem RuntimeError oder keiner RuntimeException gekommen ist
     */
    int main() {
        string input_dir = "../LennartProtte/Aufgabe2-Implementierung/Eingabedateien_b";
        string output_dir = "../LennartProtte/Aufgabe2-Implementierung/Ausgabedateien_b";

        //Durchläuft alle Dateien im Eingabeordner
        for (const auto &entry: filesystem::directory_iterator(input_dir)) {

            //Liest den Dateinamen aus
            string input_file = entry.path();
            string output_file = output_dir + "/" + entry.path().filename().string();

            //Öffnet die Eingabedatei
            ifstream fin(input_file);

            //Öffnet die Ausgabedatei
            ofstream fout(output_file);

            //Liest die Eingabedatei ein
            vector<Slice> slices;
            int a, b, n;
            fin >> n;
            while (fin >> a >> b) {
                slices.emplace_back(a, b);
            }

            //Berechnet das Volumen des Quaders
            int volume = 0;
            for (const auto &slice: slices) {
                volume += slice.length * slice.height;
            }

            //Sortiert die Scheiben nach ihrer Fläche
            sort(slices.begin(), slices.end(), [](Slice a, Slice b) {
                     return (a.length * a.height) > (b.length * b.height);
                 }
            );

            vector<pair<Slice, Dimension>> order;
            map<tuple<int, int, int>, vector<pair<Slice, Dimension>>> solution;
            bool success = false;
            //Erhöht die Anzahl der Würfel schrittweise, bis es eine Lösung gibt
            for (int count_of_squares = 1; count_of_squares <= slices.size(); count_of_squares++) {
                const auto &combinations = find_all_combinations(volume, count_of_squares, slices);
                for (const auto &combination: combinations) {
                    bool valid = true;
                    vector<Slice> t_slices(slices);
                    for (auto dimension: combination) {
                        order.clear();
                        if (!calculate_square(get<0>(dimension), get<1>(dimension), get<2>(dimension), order, t_slices)) {
                            valid = false;
                        } else {
                            solution.insert(make_pair(dimension, order));
                        }
                    }
                    if (valid && t_slices.empty()) {
                        success = true;
                        goto end;
                    } else {
                        solution.clear();
                    }
                }
            }
            end:
            //Schreibt die Ausgabe
            if (success) {
                fout << "Die Scheiben können zu " << solution.size() << " Quader(n) zusammengesetzt werden." << endl;
                for (auto & it : solution) {
                    fout << "Quader: " << get<0>(it.first) << "x" << get<1>(it.first) << "x" << get<2>(it.first) << endl;
                    for (auto item: it.second) {
                        fout << "Slice: (" << item.first.length << ", " << item.first.height
                             << ") Dimension: " << to_string(item.second) << endl;
                    }
                    fout << endl;
                }
            } else {
                fout << "Die Scheiben können zu keinem Quader zusammengesetzt werden." << endl;
            }

            //Dateien schließen
            fin.close();
            fout.close();
        }
        return 0;
    }
    \end{lstlisting}

\end{document}
