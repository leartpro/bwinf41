%! Author = leartpro
%! Date = 04.01.23

\documentclass[a4paper,10pt,ngerman]{scrartcl}
\usepackage{babel}
\usepackage[T1]{fontenc}
\usepackage[utf8x]{inputenc}
\usepackage[a4paper,margin=2.5cm,footskip=0.5cm]{geometry}

% Die nächsten drei Felder bitte anpassen:
\newcommand{\Aufgabe}{Aufgabe 1: Weniger krumme Touren} % Aufgabennummer und Aufgabennamen angeben
\newcommand{\TeilnahmeId}{?????}                        % Teilnahme-ID angeben
\newcommand{\Name}{Lennart Protte}                      % Name des Bearbeiter / der Bearbeiterin dieser Aufgabe angeben


% Kopf- und Fußzeilen
\usepackage{scrlayer-scrpage, lastpage}
\setkomafont{pageheadfoot}{\large\textrm}
\lohead{\Aufgabe}
\rohead{Teilnahme-ID: \TeilnahmeId}
\cfoot*{\thepage{}/\pageref{LastPage}}

% Position des Titels
\usepackage{titling}
\setlength{\droptitle}{-1.0cm}

% Für mathematische Befehle und Symbole
\usepackage{amsmath}
\usepackage{amssymb}

% Für Bilder
\usepackage{graphicx}

%Für Überschriften
\usepackage[labelformat=empty]{caption}
\captionsetup[algorithm]{labelformat=empty}

% Für Algorithmen
\usepackage{algpseudocode}
\usepackage{adigraph}
\usepackage{algorithm}
\usepackage{algorithmicx}

% Für Quelltext
\usepackage{listings}
\usepackage{color}
\usepackage{textcomp}
\definecolor{mygreen}{rgb}{0,0.6,0}
\definecolor{mygray}{rgb}{0.5,0.5,0.5}
\definecolor{mymauve}{rgb}{0.58,0,0.82}
\lstset{
    keywordstyle=\color{blue},commentstyle=\color{mygreen},
    stringstyle=\color{mymauve},rulecolor=\color{black},
    basicstyle=\footnotesize\ttfamily,numberstyle=\tiny\color{mygray},
    captionpos=b, % sets the caption-position to bottom
    keepspaces=true, % keeps spaces in text
    numbers=left, numbersep=5pt, showspaces=false,showstringspaces=true,
    showtabs=false, stepnumber=2, tabsize=2, title=\lstname ,
    inputencoding = utf8,  % Input encoding
    extendedchars = true,  % Extended ASCII
    literate      =        % Support additional characters
        {á}{{\'a}}1  {é}{{\'e}}1  {í}{{\'i}}1 {ó}{{\'o}}1  {ú}{{\'u}}1
        {Á}{{\'A}}1  {É}{{\'E}}1  {Í}{{\'I}}1 {Ó}{{\'O}}1  {Ú}{{\'U}}1
        {à}{{\`a}}1  {è}{{\`e}}1  {ì}{{\`i}}1 {ò}{{\`o}}1  {ù}{{\`u}}1
        {À}{{\`A}}1  {È}{{\'E}}1  {Ì}{{\`I}}1 {Ò}{{\`O}}1  {Ù}{{\`U}}1
        {ä}{{\"a}}1  {ë}{{\"e}}1  {ï}{{\"i}}1 {ö}{{\"o}}1  {ü}{{\"u}}1
        {Ä}{{\"A}}1  {Ë}{{\"E}}1  {Ï}{{\"I}}1 {Ö}{{\"O}}1  {Ü}{{\"U}}1
        {â}{{\^a}}1  {ê}{{\^e}}1  {î}{{\^i}}1 {ô}{{\^o}}1  {û}{{\^u}}1
        {Â}{{\^A}}1  {Ê}{{\^E}}1  {Î}{{\^I}}1 {Ô}{{\^O}}1  {Û}{{\^U}}1
        {œ}{{\oe}}1  {Œ}{{\OE}}1  {æ}{{\ae}}1 {Æ}{{\AE}}1  {ß}{{\ss}}1
        {ç}{{\c c}}1 {Ç}{{\c C}}1 {ø}{{\o}}1  {Ø}{{\O}}1   {å}{{\r a}}1
        {Å}{{\r A}}1 {ã}{{\~a}}1  {õ}{{\~o}}1 {Ã}{{\~A}}1  {Õ}{{\~O}}1
        {ñ}{{\~n}}1  {Ñ}{{\~N}}1  {¿}{{?`}}1  {¡}{{!`}}1
        {°}{{\textdegree}}1 {º}{{\textordmasculine}}1 {ª}{{\textordfeminine}}1
}

% Diese beiden Pakete müssen zuletzt geladen werden
\usepackage{hyperref} % Anklickbare Links im Dokument
\usepackage{cleveref}

% Daten für die Titelseite
\title{\textbf{\Huge\Aufgabe}}
\author{\LARGE Teilnahme-ID: \LARGE \TeilnahmeId \\\\
\LARGE Bearbeiter/-in dieser Aufgabe: \\
\LARGE \Name\\\\}
\date{\LARGE\today}

\begin{document}

    \maketitle
    \tableofcontents
    \vspace{0.5cm}

    \textbf{Anleitung:} Trage oben in den Zeilen 8 bis 10 die Aufgabennummer,
    die Teilnahme-ID und die/den Bearbeiterin/Bearbeiter dieser Aufgabe mit Vor- und Nachnamen ein.
    Vergiss nicht, auch den Aufgabennamen anzupassen (statt \("`\)\LaTeX-Dokument\("'\))!

    Dann kannst du dieses Dokument mit deiner \LaTeX-Umgebung übersetzen.

    Die Texte, die hier bereits stehen, geben ein paar Hinweise zur
    Einsendung.
    Du solltest sie aber in deiner Einsendung wieder entfernen!

    \newpage
    \section{Lösungsidee}\label{sec:losungsidee}
    Die Idee der Lösung sollte hieraus vollkommen ersichtlich werden,
    ohne dass auf die eigentliche Implementierung Bezug genommen wird.
    
    

    \newpage
    \section{Umsetzung}\label{sec:umsetzung}
    Hier wird kurz erläutert, wie die Lösungsidee im Programm tatsächlich umgesetzt wurde.
    Hier können auch Implementierungsdetails erwähnt werden.

    \begin{algorithm}
        \begin{algorithmic}[1]
            \Function{CrossAngle}{$from_node,over_node,to_node$}
                \State $p \gets \Call{MakePair}{over_node.first-from_node.first,over_node.second-from_node.second}$
                \State $q \gets \Call{MakePair}{to_node.first-over_node.first,to_node.second-over_node.second}$
                \State $angle \gets acos(\frac{p.firstq.first+p.secondq.second}{\sqrt{p.first^2+p.second^2}\sqrt{q.first^2+q.second^2}})*180/\pi$
                \If {$angle>180$}
                    \State $angle \gets 180-angle$
                \EndIf
                \State \textbf{return} $angle$
            \EndFunction
            \State
            \Function{Solve}{$route,coordinates$}
                \If {$\Call{Size}{route} = \Call{Size}{coordinates}$}
                    \State \textbf{return} $true$
                \EndIf
                \If {NOT $\Call{IsEmpty}{route}$}
                    \State $p \gets \Call{Back}{route}$
                    \State $\Call{Sort}{coordinates,lambda (lhs,rhs) \gets \sqrt{(p.first-lhs.first)^2+(p.second-lhs.second)^2}<\sqrt{(p.first-rhs.first)^2+(p.second-rhs.second)^2}}$
                \EndIf
                \For {$i \gets 1$ \textbf{to} $\Call{Size}{coordinates}$}
                    \If {$coordinates[i] \in route$}
                        \State \textbf{continue}
                    \EndIf
                    \State $angle \gets -1$
                    \If {$\Call{Size}{route}\geq 2$}
                        %!\State $angle \gets \Call{CrossAngle}{route[\Call{Size{route}-1],route[\Call{Size}{route}],coordinates[i]}$
                    \EndIf
                    \If {$\Call{IsEmpty}{route}$ OR ((NOT $coordinates[i] \in route$) AND ($\Call{Size}{route} < 2$ OR $angle\geq90$ OR $angle = 0$))}
                        \State $\Call{PushBack}{route,coordinates[i]}$
                        \If {$\Call{Solve}{route,coordinates}$}
                            \State \textbf{return} $true$
                        \Else
                            \State $\Call{PopBack}{route}$
                        \EndIf
                    \EndIf
                \EndFor
                \State \textbf{return} $false$
            \EndFunction
\end{algorithmic}\label{alg:algorithm}
    \end{algorithm}

    \newpage
    \section{Beispiele}\label{sec:beispiele}
    
	\begin{figure}[!h]\centering
\NewAdigraph{wenigerkrummThree}{
0:8.4641,9.33333:start;
    1:8.80423,8.31476:168° ;
    2:8.97809,7.22415:167.9° ;
    3:8.97809,6.10918:167.9° ;
    4:8.80423,5.01858:168° ;
    5:8.4641,4:147.3° ;
    6:7.97258,3.5687:123.3° ;
    7:7.97258,3.09797:123.3° ;
    8:8.4641,2.66667:147.3° ;
    9:8.80423,1.64809:168° ;
    10:8.97809,0.557485:167.9° ;
    11:8.97809,-0.557485:167.9° ;
    12:8.80423,-1.64809:168° ;
    13:8.4641,-2.66667:168° ;
    14:7.97258,-3.5687:167.9° ;
    15:7.35114,-4.31476:167.9° ;
    16:6.62695,-4.87224:167.9° ;
    17:5.83165,-5.21679:168° ;
    18:5,-5.33333:167.9° ;
    19:4.16835,-5.21679:168° ;
    20:3.37305,-4.87224:167.9° ;
    21:2.64886,-4.31476:150° ;
    22:2.35114,-4.31476:120° ;
    23:2.02742,-3.5687:174° ;
    24:1.5359,-2.66667:168° ;
    25:1.19577,-1.64809:168° ;
    26:1.02191,-0.557485:167.9° ;
    27:1.02191,0.557485:164.1° ;
    28:0.831647,1.44988:96.3° ;
    29:1.19577,1.64809:172° ;
    30:1.62695,1.79442:96.3° ;
    31:1.5359,2.66667:115.4° ;
    32:2.02742,3.09797:123.3° ;
    33:2.02742,3.5687:123.3° ;
    34:1.5359,4:115.4° ;
    35:1.62695,4.87224:96.3° ;
    36:1.19577,5.01858:172° ;
    37:0.831647,5.21679:96.3° ;
    38:1.02191,6.10918:164.1° ;
    39:1.02191,7.22415:167.9° ;
    40:1.19577,8.31476:168° ;
    41:1.5359,9.33333:168° ;
    42:2.02742,10.2354:174° ;
    43:2.35114,10.9814:120° ;
    44:2.64886,10.9814:120° ;
    45:2.97258,10.2354:174° ;
    46:3.4641,9.33333:168° ;
    47:3.80423,8.31476:168° ;
    48:3.97809,7.22415:167.9° ;
    49:3.97809,6.10918:164.1° ;
    50:4.16835,5.21679:96.3° ;
    51:3.80423,5.01858:172° ;
    52:3.37305,4.87224:96.3° ;
    53:3.4641,4:115.4° ;
    54:2.97258,3.5687:123.3° ;
    55:2.97258,3.09797:123.3° ;
    56:3.4641,2.66667:115.4° ;
    57:3.37305,1.79442:96.3° ;
    58:3.80423,1.64809:172° ;
    59:4.16835,1.44988:96.3° ;
    60:3.97809,0.557485:164.1° ;
    61:3.97809,-0.557485:167.9° ;
    62:3.80423,-1.64809:168° ;
    63:3.4641,-2.66667:168° ;
    64:2.97258,-3.5687:161.9° ;
    65:1.62695,-4.87224:161.9° ;
    66:0.831647,-5.21679:168° ;
    67:0,-5.33333:167.9° ;
    68:-0.831647,-5.21679:168° ;
    69:-1.62695,-4.87224:167.9° ;
    70:-2.35114,-4.31476:167.9° ;
    71:-2.97258,-3.5687:167.9° ;
    72:-3.4641,-2.66667:168° ;
    73:-3.80423,-1.64809:168° ;
    74:-3.97809,-0.557485:167.9° ;
    75:-3.97809,0.557485:167.9° ;
    76:-3.80423,1.64809:168° ;
    77:-3.4641,2.66667:147.3° ;
    78:-2.97258,3.09797:123.3° ;
    79:-2.97258,3.5687:123.3° ;
    80:-3.4641,4:147.3° ;
    81:-3.80423,5.01858:168° ;
    82:-3.97809,6.10918:167.9° ;
    83:-3.97809,7.22415:167.9° ;
    84:-3.80423,8.31476:168° ;
    85:-3.4641,9.33333:168° ;
    86:-2.97258,10.2354:167.9° ;
    87:-2.35114,10.9814:167.9° ;
    88:-1.62695,11.5389:167.9° ;
    89:-0.831647,11.8835:168° ;
    90:0,12:167.9° ;
    91:0.831647,11.8835:168° ;
    92:1.62695,11.5389:162° ;
    93:3.37305,11.5389:162° ;
    94:4.16835,11.8835:168° ;
    95:5,12:167.9° ;
    96:5.83165,11.8835:168° ;
    97:6.62695,11.5389:167.9° ;
    98:7.35114,10.9814:167.9° ;
    99:7.97258,10.2354:113.5° ;
    100:6.62695,4.87224:90.4° ;
    101:5.83165,5.21679:168° ;
    102:5,5.33333:156° ;
    103:2.64886,4.31476:162° ;
    104:2.35114,4.31476:90° ;
    105:2.35114,2.35191:90° ;
    106:2.64886,2.35191:162° ;
    107:5,1.33333:156° ;
    108:5.83165,1.44988:168° ;
    109:6.62695,1.79442:167.9° ;
    110:7.35114,2.35191:119.9° ;
    111:7.35114,4.31476:95.9° ;
    112:0,5.33333:168° ;
    113:-0.831647,5.21679:168° ;
    114:-1.62695,4.87224:167.9° ;
    115:-2.35114,4.31476:119.9° ;
    116:-2.35114,2.35191:119.9° ;
    117:-1.62695,1.79442:167.9° ;
    118:-0.831647,1.44988:168° ;
    119:0,1.33333:end;
}{
0,1;
1,2;
2,3;
3,4;
4,5;
5,6;
6,7;
7,8;
8,9;
9,10;
10,11;
11,12;
12,13;
13,14;
14,15;
15,16;
16,17;
17,18;
18,19;
19,20;
20,21;
21,22;
22,23;
23,24;
24,25;
25,26;
26,27;
27,28;
28,29;
29,30;
30,31;
31,32;
32,33;
33,34;
34,35;
35,36;
36,37;
37,38;
38,39;
39,40;
40,41;
41,42;
42,43;
43,44;
44,45;
45,46;
46,47;
47,48;
48,49;
49,50;
50,51;
51,52;
52,53;
53,54;
54,55;
55,56;
56,57;
57,58;
58,59;
59,60;
60,61;
61,62;
62,63;
63,64;
64,65;
65,66;
66,67;
67,68;
68,69;
69,70;
70,71;
71,72;
72,73;
73,74;
74,75;
75,76;
76,77;
77,78;
78,79;
79,80;
80,81;
81,82;
82,83;
83,84;
84,85;
85,86;
86,87;
87,88;
88,89;
89,90;
90,91;
91,92;
92,93;
93,94;
94,95;
95,96;
96,97;
97,98;
98,99;
99,100;
100,101;
101,102;
102,103;
103,104;
104,105;
105,106;
106,107;
107,108;
108,109;
109,110;
110,111;
111,112;
112,113;
113,114;
114,115;
115,116;
116,117;
117,118;
118,119;
}
\wenigerkrummThree{
0,1,2,3,4,5,6,7,8,9,10,11,12,13,14,15,16,17,18,19,20,21,22,23,24,25,26,27,28,29,30,31,32,33,34,35,36,37,38,39,40,41,42,43,44,45,46,47,48,49,50,51,52,53,54,55,56,57,58,59,60,61,62,63,64,65,66,67,68,69,70,71,72,73,74,75,76,77,78,79,80,81,82,83,84,85,86,87,88,89,90,91,92,93,94,95,96,97,98,99,100,101,102,103,104,105,106,107,108,109,110,111,112,113,114,115,116,117,118,119::red;
}
\caption{Figur: wenigerkrumm3}
\label{fig:wenigerkrumm3}
\end{figure}
\newpage
\begin{figure}[!h]\centering
\NewAdigraph{wenigerkrummFour}{
    0:0.404243,3.12027;
    1:0.667594,2.00322:160.5° ;
    2:0.578274,1.174:151.1° ;
    3:1.02016,0.115392:149.6° ;
    4:0.842755,-1.2064:116.9° ;
    5:-1.65728,-2.08347:106° ;
    6:-1.97521,-1.63541:176.6° ;
    7:-2.74635,-0.402939:113.4° ;
    8:-3.83434,-0.56721:179.8° ;
    9:-4.423,-0.657251:105.5° ;
    10:-4.79696,0.173428:154.9° ;
    11:-4.78828,0.808542:131.9° ;
    12:-3.08177,2.30045:164.4° ;
    13:-2.38053,3.36907:91.8° ;
    14:2.02998,0.669684:170.2° ;
    15:2.78893,0.00466476:97.6° ;
    16:1.8958,-1.34175:145° ;
    17:-2.58209,-3.10083:96.1° ;
    18:-4.80738,1.14852:127.7° ;
    19:-4.38297,2.07371:150.8° ;
    20:-2.15977,3.70347:105.5° ;
    21:3.0626,-0.407218:108.4° ;
    22:2.89666,-0.869526:98.9° ;
    23:-0.334463,-0.253791:131.8° ;
    24:-0.419424,-0.112742;
}{
0,1;
1,2;
2,3;
3,4;
4,5;
5,6;
6,7;
7,8;
8,9;
9,10;
10,11;
11,12;
12,13;
13,14;
14,15;
15,16;
16,17;
17,18;
18,19;
19,20;
20,21;
21,22;
22,23;
23,24;
}
\wenigerkrummFour{
0,1,2,3,4,5,6,7,8,9,10,11,12,13,14,15,16,17,18,19,20,21,22,23,24::red;
}
\caption{Figur: wenigerkrumm4}
\label{fig:wenigerkrumm4}
\end{figure}
\newpage
\begin{figure}[!h]\centering
\NewAdigraph{wenigerkrummOne}{
    0:4,0;
    1:4.2,0:180° ;
    2:4.4,0:180° ;
    3:4.6,0:180° ;
    4:4.8,0:180° ;
    5:5,0:180° ;
    6:5.2,0:180° ;
    7:5.4,0:180° ;
    8:5.6,0:180° ;
    9:5.8,0:180° ;
    10:6,0:180° ;
    11:6.2,0:180° ;
    12:6.4,0:180° ;
    13:6.6,0:180° ;
    14:6.8,0:180° ;
    15:7,0:180° ;
    16:7.2,0:180° ;
    17:7.4,0:180° ;
    18:7.6,0:180° ;
    19:7.8,0:180° ;
    20:8,0:108.4° ;
    21:8.1,0.3:143.1° ;
    22:8,0.6:108.4° ;
    23:7.8,0.6:180° ;
    24:7.6,0.6:180° ;
    25:7.4,0.6:180° ;
    26:7.2,0.6:180° ;
    27:7,0.6:180° ;
    28:6.8,0.6:180° ;
    29:6.6,0.6:180° ;
    30:6.4,0.6:180° ;
    31:6.2,0.6:180° ;
    32:6,0.6:180° ;
    33:5.8,0.6:180° ;
    34:5.6,0.6:180° ;
    35:5.4,0.6:180° ;
    36:5.2,0.6:180° ;
    37:5,0.6:180° ;
    38:4.8,0.6:180° ;
    39:4.6,0.6:180° ;
    40:4.4,0.6:180° ;
    41:4.2,0.6:180° ;
    42:4,0.6:180° ;
    43:3.8,0.6:180° ;
    44:3.6,0.6:180° ;
    45:3.4,0.6:180° ;
    46:3.2,0.6:180° ;
    47:3,0.6:180° ;
    48:2.8,0.6:180° ;
    49:2.6,0.6:180° ;
    50:2.4,0.6:180° ;
    51:2.2,0.6:180° ;
    52:2,0.6:180° ;
    53:1.8,0.6:180° ;
    54:1.6,0.6:180° ;
    55:1.4,0.6:180° ;
    56:1.2,0.6:180° ;
    57:1,0.6:180° ;
    58:0.8,0.6:180° ;
    59:0.6,0.6:180° ;
    60:0.4,0.6:180° ;
    61:0.2,0.6:180° ;
    62:0,0.6:108.4° ;
    63:-0.1,0.3:143.1° ;
    64:0,0:108.4° ;
    65:0.2,0:180° ;
    66:0.4,0:180° ;
    67:0.6,0:180° ;
    68:0.8,0:180° ;
    69:1,0:180° ;
    70:1.2,0:180° ;
    71:1.4,0:180° ;
    72:1.6,0:180° ;
    73:1.8,0:180° ;
    74:2,0:180° ;
    75:2.2,0:180° ;
    76:2.4,0:180° ;
    77:2.6,0:180° ;
    78:2.8,0:180° ;
    79:3,0:180° ;
    80:3.2,0:180° ;
    81:3.4,0:180° ;
    82:3.6,0:180° ;
    83:3.8,0;
}{
0,1;
1,2;
2,3;
3,4;
4,5;
5,6;
6,7;
7,8;
8,9;
9,10;
10,11;
11,12;
12,13;
13,14;
14,15;
15,16;
16,17;
17,18;
18,19;
19,20;
20,21;
21,22;
22,23;
23,24;
24,25;
25,26;
26,27;
27,28;
28,29;
29,30;
30,31;
31,32;
32,33;
33,34;
34,35;
35,36;
36,37;
37,38;
38,39;
39,40;
40,41;
41,42;
42,43;
43,44;
44,45;
45,46;
46,47;
47,48;
48,49;
49,50;
50,51;
51,52;
52,53;
53,54;
54,55;
55,56;
56,57;
57,58;
58,59;
59,60;
60,61;
61,62;
62,63;
63,64;
64,65;
65,66;
66,67;
67,68;
68,69;
69,70;
70,71;
71,72;
72,73;
73,74;
74,75;
75,76;
76,77;
77,78;
78,79;
79,80;
80,81;
81,82;
82,83;
}
\wenigerkrummOne{
0,1,2,3,4,5,6,7,8,9,10,11,12,13,14,15,16,17,18,19,20,21,22,23,24,25,26,27,28,29,30,31,32,33,34,35,36,37,38,39,40,41,42,43,44,45,46,47,48,49,50,51,52,53,54,55,56,57,58,59,60,61,62,63,64,65,66,67,68,69,70,71,72,73,74,75,76,77,78,79,80,81,82,83::red;
}
\caption{Figur: wenigerkrumm1}
\label{fig:wenigerkrumm1}
\end{figure}
\newpage
\begin{figure}[!h]\centering
\NewAdigraph{wenigerkrummSeven}{
    0:-0.945331,-1.33968;
    1:-0.928061,-0.275116:174.1° ;
    2:-0.967088,0.181828:163.9° ;
    3:-0.854094,0.75359:136.9° ;
    4:-0.558239,0.966535:111.4° ;
    5:-0.0886984,0.663298:174.2° ;
    6:0.0630423,0.542078:112.6° ;
    7:-0.191617,-0.350333:99.2° ;
    8:-0.00401872,-0.438553:165.5° ;
    9:0.940227,-0.617744:131.7° ;
    10:1.11102,-0.901799:117.1° ;
    11:1.77638,-0.85669:125° ;
    12:2.3798,-1.60407:132.7° ;
    13:2.53808,-1.61467:144.9° ;
    14:3.44778,-1.06267:99° ;
    15:3.17105,-0.385088:149.9° ;
    16:2.82867,-0.120462:91° ;
    17:2.71562,-0.261069:174.7° ;
    18:1.37822,-1.64247:170.1° ;
    19:0.699183,-2.13685:163.3° ;
    20:-0.347132,-2.50508:154.7° ;
    21:-2.67462,-2.26612:139.2° ;
    22:-3.04261,-1.87689:148.7° ;
    23:-3.44756,-1.76596:140.9° ;
    24:-4.05657,-2.034:93.6° ;
    25:-3.7827,-2.78157:108.5° ;
    26:-3.11303,-2.76304:93.9° ;
    27:-2.94726,1.19216:126.1° ;
    28:-3.71298,1.80289:142.1° ;
    29:-3.78124,2.08571:135.8° ;
    30:-4.30228,2.41481:94.3° ;
    31:-4.01542,2.95482:146.6° ;
    32:-4.0297,3.1055:150° ;
    33:-3.85362,3.49046:133.1° ;
    34:-3.0626,3.75635:133.6° ;
    35:-2.40773,3.41179:96.5° ;
    36:-2.6997,2.6589:126.5° ;
    37:-3.14847,2.53601:153.2° ;
    38:-4.1533,1.62821:139.7° ;
    39:-4.96339,1.60264:141.1° ;
    40:-5.68258,2.14505:151.7° ;
    41:-5.74117,2.272:123.5° ;
    42:-5.37478,2.86553:173.3° ;
    43:-4.80499,3.5867:98.1° ;
    44:-4.44985,3.38067:138.6° ;
    45:-2.28292,3.81231:113.5° ;
    46:-1.69254,2.96433:99.6° ;
    47:-2.53138,2.1393:163.3° ;
    48:-3.4271,0.509261:175.9° ;
    49:-3.68185,0.114746:126.9° ;
    50:-4.53575,0.0531772:103.2° ;
    51:-4.89919,-2.22093:132.9° ;
    52:-3.62418,-3.85246:122.8° ;
    53:1.19654,-3.41427:148.1° ;
    54:1.84596,-2.92339:147.3° ;
    55:2.13199,-2.14868:150° ;
    56:4.11436,-0.49953:112.7° ;
    57:4.49199,-0.69597:111.9° ;
    58:4.79233,-0.438883:102.5° ;
    59:5.52553,-0.988973:129.2° ;
    60:5.56211,-1.87544:173.9° ;
    61:5.51587,-2.58831:113.5° ;
    62:4.71654,-2.87678:131.7° ;
    63:4.35199,-3.78516:93.3° ;
    64:-4.34565,-0.866332:103.8° ;
    65:-4.22858,0.555408:117.4° ;
    66:-1.13829,1.85002:166.1° ;
    67:0.331465,2.08042:124.3° ;
    68:0.817943,1.56305:158.9° ;
    69:1.09104,1.43134:152.6° ;
    70:1.1278,1.43237:97.5° ;
    71:1.15111,1.28835:125.1° ;
    72:2.58049,0.594034:106.1° ;
    73:3.17484,1.25238:142.2° ;
    74:3.25846,2.34931:107.9° ;
    75:2.96217,2.47117:130.7° ;
    76:2.536,2.25455:126.8° ;
    77:2.4075,2.31779:140.3° ;
    78:2.00088,3.2259:166.4° ;
    79:1.91894,3.66556:99° ;
    80:2.6171,3.9139:119.3° ;
    81:3.41029,3.2234:126.7° ;
    82:3.39981,3.08521:94.4° ;
    83:1.09533,3.08108:92.7° ;
    84:1.06873,2.51366:151.8° ;
    85:1.49775,1.61173:138.2° ;
    86:3.56397,0.74064:108.5° ;
    87:5.42602,2.85048:90° ;
    88:4.43616,3.72482:135.7° ;
    89:-0.260605,3.49396:94.8° ;
    90:-0.366321,0.555117:96.7° ;
    91:3.78774,-0.0893045:90.8° ;
    92:4.33382,3.12629:174.1° ;
    93:4.34438,3.2859:93.5° ;
    94:0.681581,3.75462:108.8° ;
    95:0.172873,2.71815:121.2° ;
    96:1.1087,1.25458:96.8° ;
    97:4.17185,2.73237:102.8° ;
    98:5.93824,0.516231:112.2° ;
    99:5.12952,-0.931828;
}{
0,1;
1,2;
2,3;
3,4;
4,5;
5,6;
6,7;
7,8;
8,9;
9,10;
10,11;
11,12;
12,13;
13,14;
14,15;
15,16;
16,17;
17,18;
18,19;
19,20;
20,21;
21,22;
22,23;
23,24;
24,25;
25,26;
26,27;
27,28;
28,29;
29,30;
30,31;
31,32;
32,33;
33,34;
34,35;
35,36;
36,37;
37,38;
38,39;
39,40;
40,41;
41,42;
42,43;
43,44;
44,45;
45,46;
46,47;
47,48;
48,49;
49,50;
50,51;
51,52;
52,53;
53,54;
54,55;
55,56;
56,57;
57,58;
58,59;
59,60;
60,61;
61,62;
62,63;
63,64;
64,65;
65,66;
66,67;
67,68;
68,69;
69,70;
70,71;
71,72;
72,73;
73,74;
74,75;
75,76;
76,77;
77,78;
78,79;
79,80;
80,81;
81,82;
82,83;
83,84;
84,85;
85,86;
86,87;
87,88;
88,89;
89,90;
90,91;
91,92;
92,93;
93,94;
94,95;
95,96;
96,97;
97,98;
98,99;
}
\wenigerkrummSeven{
0,1,2,3,4,5,6,7,8,9,10,11,12,13,14,15,16,17,18,19,20,21,22,23,24,25,26,27,28,29,30,31,32,33,34,35,36,37,38,39,40,41,42,43,44,45,46,47,48,49,50,51,52,53,54,55,56,57,58,59,60,61,62,63,64,65,66,67,68,69,70,71,72,73,74,75,76,77,78,79,80,81,82,83,84,85,86,87,88,89,90,91,92,93,94,95,96,97,98,99::red;
}
\caption{Figur: wenigerkrumm7}
\label{fig:wenigerkrumm7}
\end{figure}
\newpage
\begin{figure}[!h]\centering
\NewAdigraph{wenigerkrummTwo}{
	0:3.38947,7.61288:start;
    1:4.89821,6.74181:168° ;
    2:6.19287,5.57609:168° ;
    3:7.21688,4.16667:167.9° ;
    4:7.92547,2.57514:168° ;
    5:8.28768,0.871071:167.9° ;
    6:8.28768,-0.871071:167.9° ;
    7:7.92547,-2.57514:168° ;
    8:7.21688,-4.16667:167.9° ;
    9:6.19287,-5.57609:168° ;
    10:4.89821,-6.74181:168° ;
    11:3.38947,-7.61288:167.9° ;
    12:1.7326,-8.15123:168° ;
    13:0,-8.33333:167.9° ;
    14:-1.7326,-8.15123:168° ;
    15:-3.38947,-7.61288:167.9° ;
    16:-4.89821,-6.74181:168° ;
    17:-6.19287,-5.57609:168° ;
    18:-7.21688,-4.16667:167.9° ;
    19:-7.92547,-2.57514:168° ;
    20:-8.28768,-0.871071:167.9° ;
    21:-8.28768,0.871071:167.9° ;
    22:-7.92547,2.57514:168° ;
    23:-7.21688,4.16667:167.9° ;
    24:-6.19287,5.57609:168° ;
    25:-4.89821,6.74181:168° ;
    26:-3.38947,7.61288:167.9° ;
    27:-1.7326,8.15123:168° ;
    28:0,8.33333:167.9° ;
    29:1.7326,8.15123:114.3° ;
    30:2.5421,5.70966:138.3° ;
    31:3.67366,5.05636:168° ;
    32:4.64466,4.18207:168° ;
    33:5.41266,3.125:167.9° ;
    34:5.9441,1.93136:168° ;
    35:6.21576,0.653303:167.9° ;
    36:6.21576,-0.653303:167.9° ;
    37:5.9441,-1.93136:168° ;
    38:5.41266,-3.125:167.9° ;
    39:4.64466,-4.18207:168° ;
    40:3.67366,-5.05636:168° ;
    41:2.5421,-5.70966:167.9° ;
    42:1.29945,-6.11342:168° ;
    43:0,-6.25:167.9° ;
    44:-1.29945,-6.11342:168° ;
    45:-2.5421,-5.70966:167.9° ;
    46:-3.67366,-5.05636:168° ;
    47:-4.64466,-4.18207:168° ;
    48:-5.41266,-3.125:167.9° ;
    49:-5.9441,-1.93136:168° ;
    50:-6.21576,-0.653303:167.9° ;
    51:-6.21576,0.653303:167.9° ;
    52:-5.9441,1.93136:168° ;
    53:-5.41266,3.125:167.9° ;
    54:-4.64466,4.18207:168° ;
    55:-3.67366,5.05636:168° ;
    56:-2.5421,5.70966:167.9° ;
    57:-1.29945,6.11342:168° ;
    58:0,6.25:167.9° ;
    59:1.29945,6.11342:end;
}{
0,1;
1,2;
2,3;
3,4;
4,5;
5,6;
6,7;
7,8;
8,9;
9,10;
10,11;
11,12;
12,13;
13,14;
14,15;
15,16;
16,17;
17,18;
18,19;
19,20;
20,21;
21,22;
22,23;
23,24;
24,25;
25,26;
26,27;
27,28;
28,29;
29,30;
30,31;
31,32;
32,33;
33,34;
34,35;
35,36;
36,37;
37,38;
38,39;
39,40;
40,41;
41,42;
42,43;
43,44;
44,45;
45,46;
46,47;
47,48;
48,49;
49,50;
50,51;
51,52;
52,53;
53,54;
54,55;
55,56;
56,57;
57,58;
58,59;
}
\wenigerkrummTwo{
0,1,2,3,4,5,6,7,8,9,10,11,12,13,14,15,16,17,18,19,20,21,22,23,24,25,26,27,28,29,30,31,32,33,34,35,36,37,38,39,40,41,42,43,44,45,46,47,48,49,50,51,52,53,54,55,56,57,58,59::red;
}
\caption{Figur: wenigerkrumm2}
\label{fig:wenigerkrumm2}
\end{figure}
\newpage
\begin{figure}[!h]\centering
\NewAdigraph{wenigerkrummSix}{
    0:2.05819,1.20216;
    1:2.00014,1.53159:101.8° ;
    2:2.43322,1.70535:110.9° ;
    3:2.41813,2.63598:107.1° ;
    4:2.65589,2.71363:173.7° ;
    5:3.09741,2.80655:104.7° ;
    6:3.51356,1.97859:149.2° ;
    7:3.45674,1.18368:96.8° ;
    8:2.42785,1.13388:165.5° ;
    9:1.66012,1.29294:147.5° ;
    10:1.29118,1.65135:115.9° ;
    11:1.4738,2.20449:101.8° ;
    12:1.70088,2.17893:103.4° ;
    13:1.93563,2.82742:90.8° ;
    14:-1.10183,3.97654:112.4° ;
    15:-1.78908,3.24475:164.2° ;
    16:-2.01138,2.81617:141.8° ;
    17:-1.94783,2.48241:158.6° ;
    18:-2.054,1.91264:146° ;
    19:-2.79483,1.15873:105.1° ;
    20:-3.50236,1.55685:159.5° ;
    21:-3.89974,2.02727:103.5° ;
    22:-3.35989,2.76391:121.3° ;
    23:0.421349,2.44329:93.4° ;
    24:0.404366,1.76062:118.5° ;
    25:1.13433,1.33919:124.3° ;
    26:1.16039,0.998758:177.9° ;
    27:1.17433,0.656719:124.1° ;
    28:0.806553,0.38432:164.9° ;
    29:0.280112,-0.280307:177.1° ;
    30:-0.0918131,-0.801345:172.5° ;
    31:-0.634908,-1.38416:102.4° ;
    32:-1.02688,-1.1531:152.9° ;
    33:-1.45131,-0.485636:119.4° ;
    34:-1.8032,-0.504017:174.4° ;
    35:-2.5314,-0.612904:121.6° ;
    36:-2.89776,-1.46991:141.7° ;
    37:-3.79977,-1.96088:119.6° ;
    38:-3.82433,-3.25378:164.3° ;
    39:-3.74971,-3.54062:101.3° ;
    40:0.423533,-3.30843:129.4° ;
    41:0.933486,-3.8618:132.8° ;
    42:3.04205,-3.86763:96.2° ;
    43:3.15178,-2.88402:113.5° ;
    44:2.86228,-2.71733:133.9° ;
    45:2.69385,-2.04306:117.3° ;
    46:3.01052,-1.7646:140° ;
    47:3.10811,-1.12876:136.7° ;
    48:2.76274,-0.626976:109.6° ;
    49:2.25667,-0.761152:124.7° ;
    50:1.53294,-0.145794:114.2° ;
    51:1.63481,0.205525:164° ;
    52:3.02864,2.42855:104.9° ;
    53:2.04447,3.48404:109.1° ;
    54:0.981838,3.01358:135.1° ;
    55:-0.370148,-0.458105:132° ;
    56:0.395306,-1.98473:141° ;
    57:1.29887,-2.39569:111.6° ;
    58:1.84512,-1.87028:174.6° ;
    59:3.10684,-0.405056:166.5° ;
    60:4.9083,0.895881:106.3° ;
    61:4.6389,1.65922:103° ;
    62:4.57859,1.6525:107.7° ;
    63:3.75339,-2.45312:90.5° ;
    64:4.20373,-2.54807:143.9° ;
    65:4.42057,-2.7887:112.3° ;
    66:4.77167,-2.66287:124.7° ;
    67:5.78598,1.12103:151.6° ;
    68:5.55643,2.08525:129.9° ;
    69:5.1027,2.3119:94.9° ;
    70:4.93243,2.03412:155.1° ;
    71:4.33652,-3.04048:111.9° ;
    72:4.90042,-3.34898:110° ;
    73:4.85621,-3.64109:108.4° ;
    74:4.04694,-3.78139:155.6° ;
    75:-0.3862,-2.63622:162.9° ;
    76:-2.14394,-1.55585:97.5° ;
    77:-3.08452,-2.71044:144.8° ;
    78:-5.77488,-3.467:107° ;
    79:-5.87667,-3.3088;
}{
0,1;
1,2;
2,3;
3,4;
4,5;
5,6;
6,7;
7,8;
8,9;
9,10;
10,11;
11,12;
12,13;
13,14;
14,15;
15,16;
16,17;
17,18;
18,19;
19,20;
20,21;
21,22;
22,23;
23,24;
24,25;
25,26;
26,27;
27,28;
28,29;
29,30;
30,31;
31,32;
32,33;
33,34;
34,35;
35,36;
36,37;
37,38;
38,39;
39,40;
40,41;
41,42;
42,43;
43,44;
44,45;
45,46;
46,47;
47,48;
48,49;
49,50;
50,51;
51,52;
52,53;
53,54;
54,55;
55,56;
56,57;
57,58;
58,59;
59,60;
60,61;
61,62;
62,63;
63,64;
64,65;
65,66;
66,67;
67,68;
68,69;
69,70;
70,71;
71,72;
72,73;
73,74;
74,75;
75,76;
76,77;
77,78;
78,79;
}
\wenigerkrummSix{
0,1,2,3,4,5,6,7,8,9,10,11,12,13,14,15,16,17,18,19,20,21,22,23,24,25,26,27,28,29,30,31,32,33,34,35,36,37,38,39,40,41,42,43,44,45,46,47,48,49,50,51,52,53,54,55,56,57,58,59,60,61,62,63,64,65,66,67,68,69,70,71,72,73,74,75,76,77,78,79::red;
}
\caption{Figur: wenigerkrumm6}
\label{fig:wenigerkrumm6}
\end{figure}
\newpage

	
    \section{Quellcode}
    \label{sec:quellcode}
    \label{LastPage}
    \lstset{language=C++,
                keywordstyle=\color{magenta},
                stringstyle=\color{red},
                commentstyle=\color{green},
	}
\begin{lstlisting}[frame=single,language=C++,title=Methode graphFromLines,breaklines=true]
    /**
 * Liest die Eingabedateien ein 
 * und versucht für jede Datei eine Lösung entsprechend der Aufgabenstellung zu finden
 * Die Lösung wird anschließend in die entsprechende Ausgabedatei geschrieben
 * Sollte es keine Lösung geben, wird dies ebenfalls in die Ausgabedatei geschrieben
 * @return 0, wenn es zu keinem RuntimeError oder keiner RuntimeException gekommen ist
 */
int main() {
    string input_dir = "../LennartProtte/Aufgabe1-Implementierung/Eingabedateien";
    string output_dir = "../LennartProtte/Aufgabe1-Implementierung/Ausgabedateien";

    //Durchläuft alle Dateien im Eingabeordner
    for (const std::filesystem::directory_entry &entry: filesystem::directory_iterator(input_dir)) {

        //Liest den Dateinamen aus
        string input_file = entry.path();
        string output_file = output_dir + "/" + entry.path().filename().string();

        //Öffnet die Eingabedatei
        ifstream fin(input_file);

        //Öffnet die Ausgabedatei
        ofstream fout(output_file);

        //Liest die Eingabedatei ein
        vector<pair<double, double> > coordinates;
        double x, y;
        while (fin >> x >> y) {
            coordinates.emplace_back(x, y);
        }

        //Berechnet die Lösung
        vector<pair<double, double> > result;
        if (solve(result, coordinates)) {
            fout << "Es konnte eine Flugstrecke durch alle Außenposten ermittelt werden" << endl;
            for (int i = 0; i < result.size(); i++) {
                if (i != 0 && i != result.size() - 1) {
                    fout << cross_angle(result[i - 1], result[i], result[i + 1]) << "° ";
                }
                fout << "[(" << result[i].first << ", " << result[i].second << ")] -> " << endl;
            }
        } else {
            fout << "Es konnte keine Flugstrecke durch alle Außenposten ermittelt werde" << endl;
        }
    }
    return 0;
}
\end{lstlisting}
	
	    \newpage
\begin{lstlisting}[frame=single,language=C++,title=Methode graphFromLines,breaklines=true]
    /**
 * Versucht rekursiv mit backtracking eine möglichst kurze Route durch den Graphen zu finden,
 * welche die Kriterien der Aufgabenstellung erfüllt.
 * @param route die aktuelle Route
 * @param coordinates eine Menge aller eingelesenen Koordinaten
 * @return true, wenn alle Knoten in der Lösungsmenge (route) enthalten sind, sonst false
 */
bool solve(vector<pair<double, double> > &route, vector<pair<double, double> > &coordinates) {
    //Wenn alle Knoten in der Lösungsmenge sind
    if (route.size() == coordinates.size()) {
        return true;
    }
    //Sortiere nach dem nächsten Knoten
    if (!route.empty()) {
        const auto &p = route.back();
        sort(coordinates.begin(), coordinates.end(),
             [p](const auto &lhs, const auto &rhs) {
                 return sqrt(pow((p.first - lhs.first), 2.0) + (pow((p.second - lhs.second), 2.0)))
                 < sqrt(pow((p.first - rhs.first), 2.0) + (pow((p.second - rhs.second), 2.0)));
             });
    }
    //Für jeden Knoten
    for (int i = 0; i < coordinates.size(); i++) {
        //Wenn dieser Knoten bereits in der Lösungsmenge existiert, überspringe diesen
        if (std::find(route.begin(), route.end(), coordinates[i]) != route.end()) {
            continue;
        }
        double angle = -1;
        if(route.size() >= 2) {
            angle = cross_angle(route[route.size() - 2], route.back(), coordinates[i]);
        }
        if (route.empty() ||
            (std::find(route.begin(), route.end(), coordinates[i]) == route.end() &&
             (route.size() < 2 || angle >= 90 || angle == 0))
                ) {
            //Füge den Knoten hinzu
            route.push_back(coordinates[i]);
            //Wenn es eine Lösung mit der aktuellen Route gibt
            if (solve(route, coordinates)) {
                return true;
            } else {
                route.pop_back();
            }
        }
    }
//Wenn es mit der aktuellen Route keine Lösung geben kann
return false;
}
\end{lstlisting}
	
	\newpage
\begin{lstlisting}[frame=single,language=C++,title=Methode graphFromLines,breaklines=true]
    /**
 * Berechnet den Winkel zwischen den Vektoren von from_node nach over_node und over_node nach to_node
 * @param over_node der zweite Knoten
 * @param to_node der dritte Knoten (Zielknoten)
 * @param from_node der erste Knoten
 * @return false, wenn der Winkel der Kanten kleiner als 90° beträgt, sonst true
 */
double cross_angle(const pair<double, double> &from_node,
                   const pair<double, double> &over_node,
                   const pair<double, double> &to_node) {
    pair<double, double> p, q;
    p = make_pair(over_node.first - from_node.first,
                  over_node.second - from_node.second);
    q = make_pair(to_node.first - over_node.first,
                  to_node.second - over_node.second);
    double angle = acos(
            (p.first * q.first + p.second * q.second) / (
                    sqrt(pow(p.first, 2.0) + (pow(p.second, 2.0))) *
                    sqrt(pow(q.first, 2.0) + (pow(q.second, 2.0))
                    )
            )
    ) * 180 / M_PI; //Umrechnung von Radian nach Grad
    if(angle > 180) {
        angle = 180 - angle;
    }
    return angle;
}
	\end{lstlisting}
\end{document}