%! Author = leartpro
%! Date = 04.01.23

\documentclass[a4paper,10pt,ngerman]{scrartcl}
\usepackage{babel}
\usepackage[T1]{fontenc}
\usepackage[utf8x]{inputenc}
\usepackage[a4paper,margin=2.5cm,footskip=0.5cm]{geometry}

% Die nächsten drei Felder bitte anpassen:
\newcommand{\Aufgabe}{Aufgabe 1: Weniger krumme Touren} % Aufgabennummer und Aufgabennamen angeben
\newcommand{\TeilnahmeId}{?????}                        % Teilnahme-ID angeben
\newcommand{\Name}{Lennart Protte}                      % Name des Bearbeiter / der Bearbeiterin dieser Aufgabe angeben


% Kopf- und Fußzeilen
\usepackage{scrlayer-scrpage, lastpage}
\setkomafont{pageheadfoot}{\large\textrm}
\lohead{\Aufgabe}
\rohead{Teilnahme-ID: \TeilnahmeId}
\cfoot*{\thepage{}/\pageref{LastPage}}

% Position des Titels
\usepackage{titling}
\setlength{\droptitle}{-1.0cm}

% Für mathematische Befehle und Symbole
\usepackage{amsmath}
\usepackage{amssymb}

% Für Bilder
\usepackage{graphicx}

%Für Überschriften
\usepackage[labelformat=empty]{caption}
\captionsetup[algorithm]{labelformat=empty}

% Für Algorithmen
\usepackage{algpseudocode}
\usepackage{adigraph}
\usepackage{algorithm}
\usepackage{algorithmicx}

% Für Quelltext
\usepackage{listings}
\usepackage{color}
\definecolor{mygreen}{rgb}{0,0.6,0}
\definecolor{mygray}{rgb}{0.5,0.5,0.5}
\definecolor{mymauve}{rgb}{0.58,0,0.82}
\lstset{
    keywordstyle=\color{blue},commentstyle=\color{mygreen},
    stringstyle=\color{mymauve},rulecolor=\color{black},
    basicstyle=\footnotesize\ttfamily,numberstyle=\tiny\color{mygray},
    captionpos=b, % sets the caption-position to bottom
    keepspaces=true, % keeps spaces in text
    numbers=left, numbersep=5pt, showspaces=false,showstringspaces=true,
    showtabs=false, stepnumber=2, tabsize=2, title=\lstname,
    inputencoding = utf8,  % Input encoding
    extendedchars = true,  % Extended ASCII
    literate      =        % Support additional characters
        {á}{{\'a}}1  {é}{{\'e}}1  {í}{{\'i}}1 {ó}{{\'o}}1  {ú}{{\'u}}1
        {Á}{{\'A}}1  {É}{{\'E}}1  {Í}{{\'I}}1 {Ó}{{\'O}}1  {Ú}{{\'U}}1
        {à}{{\`a}}1  {è}{{\`e}}1  {ì}{{\`i}}1 {ò}{{\`o}}1  {ù}{{\`u}}1
        {À}{{\`A}}1  {È}{{\'E}}1  {Ì}{{\`I}}1 {Ò}{{\`O}}1  {Ù}{{\`U}}1
        {ä}{{\"a}}1  {ë}{{\"e}}1  {ï}{{\"i}}1 {ö}{{\"o}}1  {ü}{{\"u}}1
        {Ä}{{\"A}}1  {Ë}{{\"E}}1  {Ï}{{\"I}}1 {Ö}{{\"O}}1  {Ü}{{\"U}}1
        {â}{{\^a}}1  {ê}{{\^e}}1  {î}{{\^i}}1 {ô}{{\^o}}1  {û}{{\^u}}1
        {Â}{{\^A}}1  {Ê}{{\^E}}1  {Î}{{\^I}}1 {Ô}{{\^O}}1  {Û}{{\^U}}1
        {œ}{{\oe}}1  {Œ}{{\OE}}1  {æ}{{\ae}}1 {Æ}{{\AE}}1  {ß}{{\ss}}1
        {ç}{{\c c}}1 {Ç}{{\c C}}1 {ø}{{\o}}1  {Ø}{{\O}}1   {å}{{\r a}}1
        {Å}{{\r A}}1 {ã}{{\~a}}1  {õ}{{\~o}}1 {Ã}{{\~A}}1  {Õ}{{\~O}}1
        {ñ}{{\~n}}1  {Ñ}{{\~N}}1  {¿}{{?`}}1  {¡}{{!`}}1
        {°}{{\textdegree}}1 {º}{{\textordmasculine}}1 {ª}{{\textordfeminine}}1
}

% Diese beiden Pakete müssen zuletzt geladen werden
\usepackage{hyperref} % Anklickbare Links im Dokument
\usepackage{cleveref}

% Daten für die Titelseite
\title{\textbf{\Huge\Aufgabe}}
\author{\LARGE Teilnahme-ID: \LARGE \TeilnahmeId \\\\
\LARGE Bearbeiter/-in dieser Aufgabe: \\
\LARGE \Name\\\\}
\date{\LARGE\today}

\begin{document}

    \maketitle
    \tableofcontents
    \vspace{0.5cm}

    \textbf{Anleitung:} Trage oben in den Zeilen 8 bis 10 die Aufgabennummer,
    die Teilnahme-ID und die/den Bearbeiterin/Bearbeiter dieser Aufgabe mit Vor- und Nachnamen ein.
    Vergiss nicht, auch den Aufgabennamen anzupassen (statt \("`\)\LaTeX-Dokument\("'\))!

    Dann kannst du dieses Dokument mit deiner \LaTeX-Umgebung übersetzen.

    Die Texte, die hier bereits stehen, geben ein paar Hinweise zur
    Einsendung.
    Du solltest sie aber in deiner Einsendung wieder entfernen!

    \newpage
    \section{Lösungsidee}\label{sec:losungsidee}
    Die Idee der Lösung sollte hieraus vollkommen ersichtlich werden,
    ohne dass auf die eigentliche Implementierung Bezug genommen wird.

    \newpage
    \section{Umsetzung}\label{sec:umsetzung}
    Hier wird kurz erläutert, wie die Lösungsidee im Programm tatsächlich umgesetzt wurde.
    Hier können auch Implementierungsdetails erwähnt werden.

    \newpage
    \section{Beispiele}\label{sec:beispiele}
    Genügend Beispiele einbinden!
    Die Beispiele von der BwInf-Webseite sollten hier diskutiert werden,
    aber auch eigene Beispiele sind sehr gut – besonders wenn sie Spezialfälle abdecken.
    Aber bitte nicht 30 Seiten Programmausgabe hier einfügen!

    \newpage
    \section{Quellcode}\label{sec:quellcode}\label{LastPage}
    Unwichtige Teile des Programms sollen hier nicht abgedruckt werden.
    Dieser Teil sollte nicht mehr als 2–3 Seiten umfassen, maximal 10.



\end{document}
