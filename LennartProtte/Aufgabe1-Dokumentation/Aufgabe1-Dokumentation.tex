%! Author = leartpro
%! Date = 04.01.23

\documentclass[a4paper,10pt,ngerman]{scrartcl}
\usepackage{babel}
\usepackage[T1]{fontenc}
\usepackage[utf8x]{inputenc}
\usepackage[a4paper,margin=2.5cm,footskip=0.5cm]{geometry}

% Die nächsten drei Felder bitte anpassen:
\newcommand{\Aufgabe}{Aufgabe 1: Weniger krumme Touren} % Aufgabennummer und Aufgabennamen angeben
\newcommand{\TeilnahmeId}{?????}                        % Teilnahme-ID angeben
\newcommand{\Name}{Lennart Protte}                      % Name des Bearbeiter / der Bearbeiterin dieser Aufgabe angeben


% Kopf- und Fußzeilen
\usepackage{scrlayer-scrpage, lastpage}
\setkomafont{pageheadfoot}{\large\textrm}
\lohead{\Aufgabe}
\rohead{Teilnahme-ID: \TeilnahmeId}
\cfoot*{\thepage{}/\pageref{LastPage}}

% Position des Titels
\usepackage{titling}
\setlength{\droptitle}{-1.0cm}

% Für mathematische Befehle und Symbole
\usepackage{amsmath}
\usepackage{amssymb}

% Für Bilder
\usepackage{graphicx}

%Für Überschriften
\usepackage[labelformat=empty]{caption}
\captionsetup[algorithm]{labelformat=empty}

% Für Algorithmen
\usepackage{algpseudocode}
\usepackage{adigraph}
\usepackage{algorithm}
\usepackage{algorithmicx}

% Für Quelltext
\usepackage{listings}
\usepackage{color}
\usepackage{textcomp}
\definecolor{mygreen}{rgb}{0,0.6,0}
\definecolor{mygray}{rgb}{0.5,0.5,0.5}
\definecolor{mymauve}{rgb}{0.58,0,0.82}
\lstset{
    keywordstyle=\color{blue},commentstyle=\color{mygreen},
    stringstyle=\color{mymauve},rulecolor=\color{black},
    basicstyle=\footnotesize\ttfamily,numberstyle=\tiny\color{mygray},
    captionpos=b, % sets the caption-position to bottom
    keepspaces=true, % keeps spaces in text
    numbers=left, numbersep=5pt, showspaces=false,showstringspaces=true,
    showtabs=false, stepnumber=2, tabsize=2, title=\lstname ,
    inputencoding = utf8,  % Input encoding
    extendedchars = true,  % Extended ASCII
    literate      =        % Support additional characters
        {á}{{\'a}}1  {é}{{\'e}}1  {í}{{\'i}}1 {ó}{{\'o}}1  {ú}{{\'u}}1
        {Á}{{\'A}}1  {É}{{\'E}}1  {Í}{{\'I}}1 {Ó}{{\'O}}1  {Ú}{{\'U}}1
        {à}{{\`a}}1  {è}{{\`e}}1  {ì}{{\`i}}1 {ò}{{\`o}}1  {ù}{{\`u}}1
        {À}{{\`A}}1  {È}{{\'E}}1  {Ì}{{\`I}}1 {Ò}{{\`O}}1  {Ù}{{\`U}}1
        {ä}{{\"a}}1  {ë}{{\"e}}1  {ï}{{\"i}}1 {ö}{{\"o}}1  {ü}{{\"u}}1
        {Ä}{{\"A}}1  {Ë}{{\"E}}1  {Ï}{{\"I}}1 {Ö}{{\"O}}1  {Ü}{{\"U}}1
        {â}{{\^a}}1  {ê}{{\^e}}1  {î}{{\^i}}1 {ô}{{\^o}}1  {û}{{\^u}}1
        {Â}{{\^A}}1  {Ê}{{\^E}}1  {Î}{{\^I}}1 {Ô}{{\^O}}1  {Û}{{\^U}}1
        {œ}{{\oe}}1  {Œ}{{\OE}}1  {æ}{{\ae}}1 {Æ}{{\AE}}1  {ß}{{\ss}}1
        {ç}{{\c c}}1 {Ç}{{\c C}}1 {ø}{{\o}}1  {Ø}{{\O}}1   {å}{{\r a}}1
        {Å}{{\r A}}1 {ã}{{\~a}}1  {õ}{{\~o}}1 {Ã}{{\~A}}1  {Õ}{{\~O}}1
        {ñ}{{\~n}}1  {Ñ}{{\~N}}1  {¿}{{?`}}1  {¡}{{!`}}1
        {°}{{\textdegree}}1 {º}{{\textordmasculine}}1 {ª}{{\textordfeminine}}1
}

% Diese beiden Pakete müssen zuletzt geladen werden
\usepackage{hyperref} % Anklickbare Links im Dokument
\usepackage{cleveref}

% Daten für die Titelseite
\title{\textbf{\Huge\Aufgabe}}
\author{\LARGE Teilnahme-ID: \LARGE \TeilnahmeId \\\\
\LARGE Bearbeiter/-in dieser Aufgabe: \\
\LARGE \Name\\\\}
\date{\LARGE\today}

\begin{document}

    \maketitle
    \tableofcontents
    \vspace{0.5cm}

    \textbf{Anleitung:} Trage oben in den Zeilen 8 bis 10 die Aufgabennummer,
    die Teilnahme-ID und die/den Bearbeiterin/Bearbeiter dieser Aufgabe mit Vor- und Nachnamen ein.
    Vergiss nicht, auch den Aufgabennamen anzupassen (statt \("`\)\LaTeX-Dokument\("'\))!

    Dann kannst du dieses Dokument mit deiner \LaTeX-Umgebung übersetzen.

    Die Texte, die hier bereits stehen, geben ein paar Hinweise zur
    Einsendung.
    Du solltest sie aber in deiner Einsendung wieder entfernen!

    \newpage
    \section{Lösungsidee}\label{sec:losungsidee}
    Die Idee der Lösung sollte hieraus vollkommen ersichtlich werden,
    ohne dass auf die eigentliche Implementierung Bezug genommen wird.
    
    

    \newpage
    \section{Umsetzung}\label{sec:umsetzung}
    Hier wird kurz erläutert, wie die Lösungsidee im Programm tatsächlich umgesetzt wurde.
    Hier können auch Implementierungsdetails erwähnt werden.

    \begin{algorithm}
        \begin{algorithmic}[1]
            \Function{CrossAngle}{$from_node,over_node,to_node$}
                \State $p \gets \Call{MakePair}{over_node.first-from_node.first,over_node.second-from_node.second}$
                \State $q \gets \Call{MakePair}{to_node.first-over_node.first,to_node.second-over_node.second}$
                \State $angle \gets acos(\frac{p.firstq.first+p.secondq.second}{\sqrt{p.first^2+p.second^2}\sqrt{q.first^2+q.second^2}})*180/\pi$
                \If {$angle>180$}
                    \State $angle \gets 180-angle$
                \EndIf
                \State \textbf{return} $angle$
            \EndFunction
            \State
            \Function{Solve}{$route,coordinates$}
                \If {$\Call{Size}{route} = \Call{Size}{coordinates}$}
                    \State \textbf{return} $true$
                \EndIf
                \If {NOT $\Call{IsEmpty}{route}$}
                    \State $p \gets \Call{Back}{route}$
                    \State $\Call{Sort}{coordinates,lambda (lhs,rhs) \gets \sqrt{(p.first-lhs.first)^2+(p.second-lhs.second)^2}<\sqrt{(p.first-rhs.first)^2+(p.second-rhs.second)^2}}$
                \EndIf
                \For {$i \gets 1$ \textbf{to} $\Call{Size}{coordinates}$}
                    \If {$coordinates[i] \in route$}
                        \State \textbf{continue}
                    \EndIf
                    \State $angle \gets -1$
                    \If {$\Call{Size}{route}\geq 2$}
                        %!\State $angle \gets \Call{CrossAngle}{route[\Call{Size{route}-1],route[\Call{Size}{route}],coordinates[i]}$
                    \EndIf
                    \If {$\Call{IsEmpty}{route}$ OR ((NOT $coordinates[i] \in route$) AND ($\Call{Size}{route} < 2$ OR $angle\geq90$ OR $angle = 0$))}
                        \State $\Call{PushBack}{route,coordinates[i]}$
                        \If {$\Call{Solve}{route,coordinates}$}
                            \State \textbf{return} $true$
                        \Else
                            \State $\Call{PopBack}{route}$
                        \EndIf
                    \EndIf
                \EndFor
                \State \textbf{return} $false$
            \EndFunction
\end{algorithmic}\label{alg:algorithm}
    \end{algorithm}

    \newpage
    \section{Beispiele}\label{sec:beispiele}
    Genügend Beispiele einbinden!
    Die Beispiele von der BwInf-Webseite sollten hier diskutiert werden,
    aber auch eigene Beispiele sind sehr gut – besonders wenn sie Spezialfälle abdecken.
    Aber bitte nicht 30 Seiten Programmausgabe hier einfügen!
    \newpage
\begin{figure}\centering
\NewAdigraph{wenigerkrummFour}{
    0:0.404243,3.12027:0;
    1:0.667594,2.00322:1;
    2:0.578274,1.174:2;
    3:1.02016,0.115392:3;
    4:0.842755,-1.2064:4;
    5:-1.65728,-2.08347:5;
    6:-1.97521,-1.63541:6;
    7:-2.74635,-0.402939:7;
    8:-3.83434,-0.56721:8;
    9:-4.423,-0.657251:9;
    10:-4.79696,0.173428:10;
    11:-4.78828,0.808542:11;
    12:-3.08177,2.30045:12;
    13:-2.38053,3.36907:13;
    14:2.02998,0.669684:14;
    15:2.78893,0.00466476:15;
    16:1.8958,-1.34175:16;
    17:-2.58209,-3.10083:17;
    18:-4.80738,1.14852:18;
    19:-4.38297,2.07371:19;
    20:-2.15977,3.70347:20;
    21:3.0626,-0.407218:21;
    22:2.89666,-0.869526:22;
    23:-0.334463,-0.253791:23;
    24:-0.419424,-0.112742:24;
}{
0,1;
1,2;
2,3;
3,4;
4,5;
5,6;
6,7;
7,8;
8,9;
9,10;
10,11;
11,12;
12,13;
13,14;
14,15;
15,16;
16,17;
17,18;
18,19;
19,20;
20,21;
21,22;
22,23;
23,24;
}
\wenigerkrummFour{
0,1,2,3,4,5,6,7,8,9,10,11,12,13,14,15,16,17,18,19,20,21,22,23,24::red;
}
\caption{Figur: wenigerkrummFour}
        \label{fig:wenigerkrummFour}
    \end{figure}
\newpage
\begin{figure}\centering
\NewAdigraph{wenigerkrummOne}{
    0:6.66667,0:0;
    1:7,0:1;
    2:7.33333,0:2;
    3:7.66667,0:3;
    4:8,0:4;
    5:8.33333,0:5;
    6:8.66667,0:6;
    7:9,0:7;
    8:9.33333,0:8;
    9:9.66667,0:9;
    10:10,0:10;
    11:10.3333,0:11;
    12:10.6667,0:12;
    13:11,0:13;
    14:11.3333,0:14;
    15:11.6667,0:15;
    16:12,0:16;
    17:12.3333,0:17;
    18:12.6667,0:18;
    19:13,0:19;
    20:13.3333,0:20;
    21:13.5,0.5:21;
    22:13.3333,1:22;
    23:13,1:23;
    24:12.6667,1:24;
    25:12.3333,1:25;
    26:12,1:26;
    27:11.6667,1:27;
    28:11.3333,1:28;
    29:11,1:29;
    30:10.6667,1:30;
    31:10.3333,1:31;
    32:10,1:32;
    33:9.66667,1:33;
    34:9.33333,1:34;
    35:9,1:35;
    36:8.66667,1:36;
    37:8.33333,1:37;
    38:8,1:38;
    39:7.66667,1:39;
    40:7.33333,1:40;
    41:7,1:41;
    42:6.66667,1:42;
    43:6.33333,1:43;
    44:6,1:44;
    45:5.66667,1:45;
    46:5.33333,1:46;
    47:5,1:47;
    48:4.66667,1:48;
    49:4.33333,1:49;
    50:4,1:50;
    51:3.66667,1:51;
    52:3.33333,1:52;
    53:3,1:53;
    54:2.66667,1:54;
    55:2.33333,1:55;
    56:2,1:56;
    57:1.66667,1:57;
    58:1.33333,1:58;
    59:1,1:59;
    60:0.666667,1:60;
    61:0.333333,1:61;
    62:0,1:62;
    63:-0.166667,0.5:63;
    64:0,0:64;
    65:0.333333,0:65;
    66:0.666667,0:66;
    67:1,0:67;
    68:1.33333,0:68;
    69:1.66667,0:69;
    70:2,0:70;
    71:2.33333,0:71;
    72:2.66667,0:72;
    73:3,0:73;
    74:3.33333,0:74;
    75:3.66667,0:75;
    76:4,0:76;
    77:4.33333,0:77;
    78:4.66667,0:78;
    79:5,0:79;
    80:5.33333,0:80;
    81:5.66667,0:81;
    82:6,0:82;
    83:6.33333,0:83;
}{
0,1;
1,2;
2,3;
3,4;
4,5;
5,6;
6,7;
7,8;
8,9;
9,10;
10,11;
11,12;
12,13;
13,14;
14,15;
15,16;
16,17;
17,18;
18,19;
19,20;
20,21;
21,22;
22,23;
23,24;
24,25;
25,26;
26,27;
27,28;
28,29;
29,30;
30,31;
31,32;
32,33;
33,34;
34,35;
35,36;
36,37;
37,38;
38,39;
39,40;
40,41;
41,42;
42,43;
43,44;
44,45;
45,46;
46,47;
47,48;
48,49;
49,50;
50,51;
51,52;
52,53;
53,54;
54,55;
55,56;
56,57;
57,58;
58,59;
59,60;
60,61;
61,62;
62,63;
63,64;
64,65;
65,66;
66,67;
67,68;
68,69;
69,70;
70,71;
71,72;
72,73;
73,74;
74,75;
75,76;
76,77;
77,78;
78,79;
79,80;
80,81;
81,82;
82,83;
}
\wenigerkrummOne{
0,1,2,3,4,5,6,7,8,9,10,11,12,13,14,15,16,17,18,19,20,21,22,23,24,25,26,27,28,29,30,31,32,33,34,35,36,37,38,39,40,41,42,43,44,45,46,47,48,49,50,51,52,53,54,55,56,57,58,59,60,61,62,63,64,65,66,67,68,69,70,71,72,73,74,75,76,77,78,79,80,81,82,83::red;
}
\caption{Figur: wenigerkrummOne}
        \label{fig:wenigerkrummOne}
    \end{figure}
\newpage
\begin{figure}\centering
\NewAdigraph{test}{
    0:8,2:0;
    1:6,3:1;
    2:4,4:2;
    3:3,6:3;
    4:2,8:4;
    5:3,10:5;
    6:4,12:6;
    7:12,12:7;
    8:13,10:8;
    9:14,8:9;
    10:13,6:10;
    11:8,8:11;
    12:6,13:12;
    13:8,14:13;
    14:10,13:14;
    15:12,4:15;
    16:10,3:16;
}{
0,1;
1,2;
2,3;
3,4;
4,5;
5,6;
6,7;
7,8;
8,9;
9,10;
10,11;
11,12;
12,13;
13,14;
14,15;
15,16;
}
\test{
0,1,2,3,4,5,6,7,8,9,10,11,12,13,14,15,16::red;
}
\caption{Figur: test}
        \label{fig:test}
    \end{figure}
\newpage
\begin{figure}\centering
\NewAdigraph{wenigerkrummTwo}{
    0:1.62695,3.65418:0;
    1:2.35114,3.23607:1;
    2:2.97258,2.67652:2;
    3:3.4641,2:3;
    4:3.80423,1.23607:4;
    5:3.97809,0.418114:5;
    6:3.97809,-0.418114:6;
    7:3.80423,-1.23607:7;
    8:3.4641,-2:8;
    9:2.97258,-2.67652:9;
    10:2.35114,-3.23607:10;
    11:1.62695,-3.65418:11;
    12:0.831647,-3.91259:12;
    13:0,-4:13;
    14:-0.831647,-3.91259:14;
    15:-1.62695,-3.65418:15;
    16:-2.35114,-3.23607:16;
    17:-2.97258,-2.67652:17;
    18:-3.4641,-2:18;
    19:-3.80423,-1.23607:19;
    20:-3.97809,-0.418114:20;
    21:-3.97809,0.418114:21;
    22:-3.80423,1.23607:22;
    23:-3.4641,2:23;
    24:-2.97258,2.67652:24;
    25:-2.35114,3.23607:25;
    26:-1.62695,3.65418:26;
    27:-0.831647,3.91259:27;
    28:0,4:28;
    29:0.831647,3.91259:29;
    30:1.22021,2.74064:30;
    31:1.76336,2.42705:31;
    32:2.22943,2.00739:32;
    33:2.59808,1.5:33;
    34:2.85317,0.927051:34;
    35:2.98357,0.313585:35;
    36:2.98357,-0.313585:36;
    37:2.85317,-0.927051:37;
    38:2.59808,-1.5:38;
    39:2.22943,-2.00739:39;
    40:1.76336,-2.42705:40;
    41:1.22021,-2.74064:41;
    42:0.623735,-2.93444:42;
    43:0,-3:43;
    44:-0.623735,-2.93444:44;
    45:-1.22021,-2.74064:45;
    46:-1.76336,-2.42705:46;
    47:-2.22943,-2.00739:47;
    48:-2.59808,-1.5:48;
    49:-2.85317,-0.927051:49;
    50:-2.98357,-0.313585:50;
    51:-2.98357,0.313585:51;
    52:-2.85317,0.927051:52;
    53:-2.59808,1.5:53;
    54:-2.22943,2.00739:54;
    55:-1.76336,2.42705:55;
    56:-1.22021,2.74064:56;
    57:-0.623735,2.93444:57;
    58:0,3:58;
    59:0.623735,2.93444:59;
}{
0,1;
1,2;
2,3;
3,4;
4,5;
5,6;
6,7;
7,8;
8,9;
9,10;
10,11;
11,12;
12,13;
13,14;
14,15;
15,16;
16,17;
17,18;
18,19;
19,20;
20,21;
21,22;
22,23;
23,24;
24,25;
25,26;
26,27;
27,28;
28,29;
29,30;
30,31;
31,32;
32,33;
33,34;
34,35;
35,36;
36,37;
37,38;
38,39;
39,40;
40,41;
41,42;
42,43;
43,44;
44,45;
45,46;
46,47;
47,48;
48,49;
49,50;
50,51;
51,52;
52,53;
53,54;
54,55;
55,56;
56,57;
57,58;
58,59;
}
\wenigerkrummTwo{
0,1,2,3,4,5,6,7,8,9,10,11,12,13,14,15,16,17,18,19,20,21,22,23,24,25,26,27,28,29,30,31,32,33,34,35,36,37,38,39,40,41,42,43,44,45,46,47,48,49,50,51,52,53,54,55,56,57,58,59::red;
}
\caption{Figur: wenigerkrummTwo}
        \label{fig:wenigerkrummTwo}
    \end{figure}
\newpage
\begin{figure}\centering
\NewAdigraph{wenigerkrummSix}{
    0:2.05819,1.20216:0;
    1:2.00014,1.53159:1;
    2:2.43322,1.70535:2;
    3:2.41813,2.63598:3;
    4:2.65589,2.71363:4;
    5:3.09741,2.80655:5;
    6:3.51356,1.97859:6;
    7:3.45674,1.18368:7;
    8:2.42785,1.13388:8;
    9:1.66012,1.29294:9;
    10:1.29118,1.65135:10;
    11:1.4738,2.20449:11;
    12:1.70088,2.17893:12;
    13:1.93563,2.82742:13;
    14:-1.10183,3.97654:14;
    15:-1.78908,3.24475:15;
    16:-2.01138,2.81617:16;
    17:-1.94783,2.48241:17;
    18:-2.054,1.91264:18;
    19:-2.79483,1.15873:19;
    20:-3.50236,1.55685:20;
    21:-3.89974,2.02727:21;
    22:-3.35989,2.76391:22;
    23:0.421349,2.44329:23;
    24:0.404366,1.76062:24;
    25:1.13433,1.33919:25;
    26:1.16039,0.998758:26;
    27:1.17433,0.656719:27;
    28:0.806553,0.38432:28;
    29:0.280112,-0.280307:29;
    30:-0.0918131,-0.801345:30;
    31:-0.634908,-1.38416:31;
    32:-1.02688,-1.1531:32;
    33:-1.45131,-0.485636:33;
    34:-1.8032,-0.504017:34;
    35:-2.5314,-0.612904:35;
    36:-2.89776,-1.46991:36;
    37:-3.79977,-1.96088:37;
    38:-3.82433,-3.25378:38;
    39:-3.74971,-3.54062:39;
    40:0.423533,-3.30843:40;
    41:0.933486,-3.8618:41;
    42:3.04205,-3.86763:42;
    43:3.15178,-2.88402:43;
    44:2.86228,-2.71733:44;
    45:2.69385,-2.04306:45;
    46:3.01052,-1.7646:46;
    47:3.10811,-1.12876:47;
    48:2.76274,-0.626976:48;
    49:2.25667,-0.761152:49;
    50:1.53294,-0.145794:50;
    51:1.63481,0.205525:51;
    52:3.02864,2.42855:52;
    53:2.04447,3.48404:53;
    54:0.981838,3.01358:54;
    55:-0.370148,-0.458105:55;
    56:0.395306,-1.98473:56;
    57:1.29887,-2.39569:57;
    58:1.84512,-1.87028:58;
    59:3.10684,-0.405056:59;
    60:4.9083,0.895881:60;
    61:4.6389,1.65922:61;
    62:4.57859,1.6525:62;
    63:3.75339,-2.45312:63;
    64:4.20373,-2.54807:64;
    65:4.42057,-2.7887:65;
    66:4.77167,-2.66287:66;
    67:5.78598,1.12103:67;
    68:5.55643,2.08525:68;
    69:5.1027,2.3119:69;
    70:4.93243,2.03412:70;
    71:4.33652,-3.04048:71;
    72:4.90042,-3.34898:72;
    73:4.85621,-3.64109:73;
    74:4.04694,-3.78139:74;
    75:-0.3862,-2.63622:75;
    76:-2.14394,-1.55585:76;
    77:-3.08452,-2.71044:77;
    78:-5.77488,-3.467:78;
    79:-5.87667,-3.3088:79;
}{
0,1;
1,2;
2,3;
3,4;
4,5;
5,6;
6,7;
7,8;
8,9;
9,10;
10,11;
11,12;
12,13;
13,14;
14,15;
15,16;
16,17;
17,18;
18,19;
19,20;
20,21;
21,22;
22,23;
23,24;
24,25;
25,26;
26,27;
27,28;
28,29;
29,30;
30,31;
31,32;
32,33;
33,34;
34,35;
35,36;
36,37;
37,38;
38,39;
39,40;
40,41;
41,42;
42,43;
43,44;
44,45;
45,46;
46,47;
47,48;
48,49;
49,50;
50,51;
51,52;
52,53;
53,54;
54,55;
55,56;
56,57;
57,58;
58,59;
59,60;
60,61;
61,62;
62,63;
63,64;
64,65;
65,66;
66,67;
67,68;
68,69;
69,70;
70,71;
71,72;
72,73;
73,74;
74,75;
75,76;
76,77;
77,78;
78,79;
}
\wenigerkrummSix{
0,1,2,3,4,5,6,7,8,9,10,11,12,13,14,15,16,17,18,19,20,21,22,23,24,25,26,27,28,29,30,31,32,33,34,35,36,37,38,39,40,41,42,43,44,45,46,47,48,49,50,51,52,53,54,55,56,57,58,59,60,61,62,63,64,65,66,67,68,69,70,71,72,73,74,75,76,77,78,79::red;
}
\caption{Figur: wenigerkrummSix}
        \label{fig:wenigerkrummSix}
    \end{figure}
\newpage


\begin{figure}[h!]
        \centering
        \NewAdigraph{FigurVier}{
            1:3,0;
            2:-3.5,-1;
            3:-1.5,0;
            4:4,0;
            5:0,3;
            6:-3.5,1;
            7:1.5,3;
            8:3.5,1.5;
            9:3,-1.5;
            10:0,0;
            11:0,-3;
            12:-1.5,1.5;
            13:0,1.5;
            14:1.5,0;
            15:-3,3;
            16:0,-1.5;
            17:2,-3;
            18:1.5,1.5;
            19:-2.5,-3;
            20:-1.5,-1.5;
        }{
            1,18:1;
            1,8;
            1,4;
            2,3;
            2,19:1;
            3,19;
            3,6;
            5,15;
            5,12;
            5,13;
            6,2;
            6,12;
            7,5;
            7,8;
            8,18;
            8,4;
            9,4;
            9,1;
            9,14;
            10,16;
            10,3;
            10,12;
            10,18;
            11,19;
            12,3;
            13,7;
            13,10:3;
            14,17;
            14,16;
            14,10;
            14,18;
            14,1;
            15,12;
            16,11;
            16,20;
            16,17;
            16,19;
            17,11;
            17,9;
            18,13:2;
            18,7;
            19,20:2;
            20,3;
            20,10:3;
        }
        \FigurVier{
            1,18,13,10::blue;
            2,19,20,10::red;
        }
        \caption{Figur 4: huepfburg0.txt Parcours}
        \label{fig:Figure4}
    \end{figure}

    \begin{table}[h!]
        \centering
        \begin{tabular}{lll}
            \textbf{Schritt} & \textbf{Spieler 1 Knotenmenge} & \textbf{Spieler 2 Knotenmenge} \\
            0.               & \{ 1 \}                        & \{ 2 \}                        \\
            1.               & \{ 4, 8, 18 \}                 & \{ 3, 19 \}                    \\
            2.               & \{ 4, 7, 13, 18 \}             & \{ 6, 19, 20 \}                \\
            3.               & \{ 5, 7, 10, 13 \}             & \{ 2, 3, 10, 12, 20 \}         \\
        \end{tabular}
        \caption{Tabelle 3: Schrittfolge für \hyperref[fig:Figure4]{Figur 4.}}
        \label{tab:Table3}
    \end{table}

	Bei \hyperref[fig:Figure4]{Figur 4} handelt es sich um den Beispielparcours aus der Aufgabenstellung
	beziehungsweise um den Parcours aus huepfburg0.txt aus den Eingabedateien.
	Es handelt sich daher um einen lösbaren Parcours, 
	in dem sich beide Spieler im dritten Schritt auf Feld zehn treffen.
	Wie \hyperref[tab:Table3]{Tabelle 3} zu entnehmen ist, ergibt sich im dritten Schritt der Knoten zehn aus
	der Schnittmenge der zwei Breitensuchen.
	In der Methode \hyperref[lst:sameTargetRoute]{\texttt{int[][] sameTargetRoute(\ldots)}} 
	liefert die Bedingung in der \texttt{do-while Schleife} in Schritt drei \texttt{false}.
	Daher wird im Anschluss der Zielknoten durch die Methode 
	\hyperref[lst:firstSameTargetOfTimelines]{\texttt{int firstSameTargetOfTimelines(\ldots)} }
	gesetzt und die Wege mit \hyperref[lst:findSingleRouteInTimeline]{\texttt{int[] findSingleRouteInTimeline(\ldots)}} gesetzt und anschließend wie folgt ausgegeben ausgegeben.
	
	\begin{lstlisting}[frame=single, title=Programmausgabe Figur 4., breaklines=true]
  Ergebnis für huepfburg0.txt
	Der Parcours hat folgende Lösung:
	Zielfeld: 10
	Anzahl an Schritten: 4
	Sasha's Weg: 
	1 -> 18 -> 13 -> 10
	Mika's Weg: 
	2 -> 19 -> 20 -> 10
    \end{lstlisting}
    \newpage
    \section{Quellcode}
    \label{sec:quellcode}
    \label{LastPage}
    \lstset{language=C++,
                keywordstyle=\color{magenta},
                stringstyle=\color{red},
                commentstyle=\color{green},
	}
	
\begin{lstlisting}[frame=single,language=C++,title=Methode graphFromLines,breaklines=true]
    /**
 * Liest die Eingabedateien ein 
 * und versucht für jede Datei eine Lösung entsprechend der Aufgabenstellung zu finden
 * Die Lösung wird anschließend in die entsprechende Ausgabedatei geschrieben
 * Sollte es keine Lösung geben, wird dies ebenfalls in die Ausgabedatei geschrieben
 * @return 0, wenn es zu keinem RuntimeError oder keiner RuntimeException gekommen ist
 */
int main() {
    string input_dir = "../LennartProtte/Aufgabe1-Implementierung/Eingabedateien";
    string output_dir = "../LennartProtte/Aufgabe1-Implementierung/Ausgabedateien";

    //Durchläuft alle Dateien im Eingabeordner
    for (const std::filesystem::directory_entry &entry: filesystem::directory_iterator(input_dir)) {

        //Liest den Dateinamen aus
        string input_file = entry.path();
        string output_file = output_dir + "/" + entry.path().filename().string();

        //Öffnet die Eingabedatei
        ifstream fin(input_file);

        //Öffnet die Ausgabedatei
        ofstream fout(output_file);

        //Liest die Eingabedatei ein
        vector<pair<double, double> > coordinates;
        double x, y;
        while (fin >> x >> y) {
            coordinates.emplace_back(x, y);
        }

        //Berechnet die Lösung
        vector<pair<double, double> > result;
        if (solve(result, coordinates)) {
            fout << "Es konnte eine Flugstrecke durch alle Außenposten ermittelt werden" << endl;
            for (int i = 0; i < result.size(); i++) {
                if (i != 0 && i != result.size() - 1) {
                    fout << cross_angle(result[i - 1], result[i], result[i + 1]) << "° ";
                }
                fout << "[(" << result[i].first << ", " << result[i].second << ")] -> " << endl;
            }
        } else {
            fout << "Es konnte keine Flugstrecke durch alle Außenposten ermittelt werde" << endl;
        }
    }
    return 0;
}
\end{lstlisting}
	
	    \newpage
\begin{lstlisting}[frame=single,language=C++,title=Methode graphFromLines,breaklines=true]
    /**
 * Versucht rekursiv mit backtracking eine möglichst kurze Route durch den Graphen zu finden,
 * welche die Kriterien der Aufgabenstellung erfüllt.
 * @param route die aktuelle Route
 * @param coordinates eine Menge aller eingelesenen Koordinaten
 * @return true, wenn alle Knoten in der Lösungsmenge (route) enthalten sind, sonst false
 */
bool solve(vector<pair<double, double> > &route, vector<pair<double, double> > &coordinates) {
    //Wenn alle Knoten in der Lösungsmenge sind
    if (route.size() == coordinates.size()) {
        return true;
    }
    //Sortiere nach dem nächsten Knoten
    if (!route.empty()) {
        const auto &p = route.back();
        sort(coordinates.begin(), coordinates.end(),
             [p](const auto &lhs, const auto &rhs) {
                 return sqrt(pow((p.first - lhs.first), 2.0) + (pow((p.second - lhs.second), 2.0)))
                 < sqrt(pow((p.first - rhs.first), 2.0) + (pow((p.second - rhs.second), 2.0)));
             });
    }
    //Für jeden Knoten
    for (int i = 0; i < coordinates.size(); i++) {
        //Wenn dieser Knoten bereits in der Lösungsmenge existiert, überspringe diesen
        if (std::find(route.begin(), route.end(), coordinates[i]) != route.end()) {
            continue;
        }
        double angle = -1;
        if(route.size() >= 2) {
            angle = cross_angle(route[route.size() - 2], route.back(), coordinates[i]);
        }
        if (route.empty() ||
            (std::find(route.begin(), route.end(), coordinates[i]) == route.end() &&
             (route.size() < 2 || angle >= 90 || angle == 0))
                ) {
            //Füge den Knoten hinzu
            route.push_back(coordinates[i]);
            //Wenn es eine Lösung mit der aktuellen Route gibt
            if (solve(route, coordinates)) {
                return true;
            } else {
                route.pop_back();
            }
        }
    }
//Wenn es mit der aktuellen Route keine Lösung geben kann
return false;
}
\end{lstlisting}
	
	\newpage
\begin{lstlisting}[frame=single,language=C++,title=Methode graphFromLines,breaklines=true]
    /**
 * Berechnet den Winkel zwischen den Vektoren von from_node nach over_node und over_node nach to_node
 * @param over_node der zweite Knoten
 * @param to_node der dritte Knoten (Zielknoten)
 * @param from_node der erste Knoten
 * @return false, wenn der Winkel der Kanten kleiner als 90° beträgt, sonst true
 */
double cross_angle(const pair<double, double> &from_node,
                   const pair<double, double> &over_node,
                   const pair<double, double> &to_node) {
    pair<double, double> p, q;
    p = make_pair(over_node.first - from_node.first,
                  over_node.second - from_node.second);
    q = make_pair(to_node.first - over_node.first,
                  to_node.second - over_node.second);
    double angle = acos(
            (p.first * q.first + p.second * q.second) / (
                    sqrt(pow(p.first, 2.0) + (pow(p.second, 2.0))) *
                    sqrt(pow(q.first, 2.0) + (pow(q.second, 2.0))
                    )
            )
    ) * 180 / M_PI; //Umrechnung von Radian nach Grad
    if(angle > 180) {
        angle = 180 - angle;
    }
    return angle;
}
	\end{lstlisting}
\end{document}